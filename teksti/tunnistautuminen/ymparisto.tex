Yritysten ja yhteisöjen web-sovellusten halutaan usein olevan avoimia kaikkia Internetin käyttäjiä pienemmälle osajoukolle, esimerkiksi yrityksen työntekijöille. Erilaisin palomuuri-asetuksin pääsy intranet-järjestelmään voidaan rajata vain yrityksen sisäverkkoon, mikä ei poista tunnistautumisen tarvetta. Aivan kuten Internet-sovelluksissa yleensä, myös intranet-järjestelmässä halutaan tietää, kuka yrityksen työntekijöistä sitä kulloinkin käyttää, jotta työntekijälle osataan näyttää vain häntä koskevaa dataa.

Joissakin tapauksissa intranet-järjestelmä pitää sisällään myös käyttäjähallinnan. Tällöin ei ylläpidetä erillistä tietokantaa käyttäjistä, vaan jokaiselle työntekijälle luodaan erillinen tunnus intranetin käyttöä varten. Toisissa tapauksissa taas halutaan hyödyntää erillistä käyttäjähallintaa, jolloin käyttäjän tiedot ovat esimerkiksi erillisellä AD-palvelimella (Active Directory), josta intranet-järjestelmä tunnistaa käyttäjät \cite{active_directory}.

Palvelusuuntautuneissa arkkitehtuureissa samaa käyttäjätietokantaa käyttäviä sovelluksia tai palveluita voi olla useita. Tällöin ns. pääkäyttäjäkannasta voidaan luoda oma paikallinen kopio jokaista palvelua varten. Tästä seuraa monenlaisia synkronointiongelmia \cite{synkronointi}. Esimerkiksi työntekijän irtisanoutuessa joudutaan tunnus poistamaan kaikista tietokannoista erikseen. Myös osoitteiden ja muiden tietojen muutokset täytyy päivittää kaikkiin tietokantoihin. Lisäksi käyttäjälle syntyy saman järjestelmän sisällä useita tunnuksia, joihin saattaa liittyä erilliset salasanat. Käyttäjän kannalta on ikävää kirjautua jokaiseen osapalveluun erikseen.

Koko käyttäjäkantaa ei kuitenkaan tarvitse kopioida jokaiselle sovellukselle, vaan yksittäiset sovellukset voivat tunnistautua erillistä tunnistautumispalvelua käyttäen \cite{facebook}. Tällöin käyttääkseen yrityksen intranet-palvelua työntekijä käy tunnistautumassa tunnistautumispalvelussa. Monet web-sovellukset toimivat juuri ulkoisen tunnistautumispalvelun kautta. Esimerkiksi viihdesivustoille voi luoda tunnuksen kirjautumalla Facebookin tai LinkedInin kaltaisten sivustojen kautta \cite{facebook}. Kelan sivuston käyttöä varten taas tunnistaudutaan pankkitunnuksilla Tupas-järjestelmän avulla \cite{tupas}.