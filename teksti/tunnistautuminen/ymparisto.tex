Yrityksen tai yhteisön web-sovellukset ovat avoimia pienemmälle osajoukolle käyttäjiä, esimerkiksi yrityksen intranet-järjestelmään on pääsy vain yrityksen työntekijöillä, jotka on kirjattu tietokantaan. Erilaisin palomuuri-asetuksin pääsy intranet-järjestelmään voidaan rajata vain yrityksen sisäverkkoon, mutta sekään ei poista tunnistautumisen tarvetta. Aivan kuin avoimissa sovelluksissa, myös intranet-järjestelmässä halutaan tietää kuka yrityksen työntekijöistä sitä kulloinkin käyttää, jotta työntekijälle osataan näyttää vain häntä koskevaa dataa.

Joissakin tapauksissa intranet-järjestelmä pitää sisällään myös käyttäjähallinnan, yrityksessä ei ole erillistä tietokantaa käyttäjille, vaan jokaiselle työntekijälle luodaan erillinen tunnus intranetiin. Toisissa tapauksissa taas halutaan hyödyntää erillistä käyttäjähallintaa, jolloin käyttäjän tiedot ovat  esimerkiksi erillisellä LDAP-palvelimella, jota vasten intranet-järjestelmä tunnistaa käyttäjät. Erilaisia käyttäjähallinnan teknologioita käsitellään seuraavassa luvussa.