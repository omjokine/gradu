Yrityksen tai yhteisön web-sovellukset ovat avoimia pienemmälle osajoukolle käyttäjiä, esimerkiksi yrityksen intranet-järjestelmään on pääsy vain yrityksen työntekijöillä, jotka on kirjattu tietokantaan. Erilaisin palomuuri-asetuksin pääsy intranet-järjestelmään voidaan rajata vain yrityksen sisäverkkoon, mutta sekään ei poista tunnistautumisen tarvetta. Aivan kuin avoimissa sovelluksissa, myös intranet-järjestelmässä halutaan tietää kuka yrityksen työntekijöistä sitä kulloinkin käyttää, jotta työntekijälle osataan näyttää vain häntä koskevaa dataa.

Joissakin tapauksissa intranet-järjestelmä pitää sisällään myös käyttäjähallinnan, jolloin yrityksessä ei ole erillistä tietokantaa käyttäjille, vaan jokaiselle työntekijälle luodaan erillinen tunnus intranetiin. Toisissa tapauksissa taas halutaan hyödyntää erillistä käyttäjähallintaa, jolloin käyttäjän tiedot ovat  esimerkiksi erillisellä LDAP-palvelimella, jota vasten intranet-järjestelmä tunnistaa käyttäjät.

Palvelusuuntautuneissa arkkitehtuureissa samaa käyttäjätietokantaa käyttäviä sovelluksia, tai palveluita, voi olla useita. Tällöin ns. pääkäyttäjäkannasta voidaan luoda oma paikallinen kopio jokaista palvelua varten. Tästä seuraa monenlaisia synkronointiongelmia [TODO: lähde]. Esimerkiksi työntekijän irtisanoutuessa joudutaan tunnus poistamaan kaikista tietokannoista erikseen. Myös osoitteen yms. tietojen muutokset täytyy päivittää kaikkiin tietokantoihin. Lisäksi käyttäjälle syntyy saman järjestelmän sisällä monia tunnuksia, joihin saattaa liittyä erilliset salasanat. Käyttäjän kannalta on myöskin ikävää kirjautua jokaiseen osapalveluun erikseen.

Koko käyttäjäkantaa ei ole kuitenkaan tarve kopioida jokaiselle sovellukselle, vaan yksittäiset sovellukset voivat tunnistautua erillistä tunnistautumispalvelua vasten [TODO: lähde]. Tällöin käyttääkseen yrityksen intranet-palvelua, täytyy käydä tunnistautumassa tunnistautumispalvelussa. Nykypäivänä monet web-sovellukset toimivat juuri ulkoisen tunnistaumispalvelun kautta. Viihdesivustoille voi luoda tunnuksen kirjautumalla Facebookin tai LinkedInin kaltaisten sivustojen kautta [TODO: lähde?]. Myös esimerkiksi Kelan sivuston käyttöä varten tunnistaudutaan pankkitunnuksilla Tupas-järjestelmän avulla [TODO: lähde].

Palvelusuuntauneen arkkitehtuurin kannalta tunnistautumisen keskittäminen yhdelle palvelulle on kiinnostava idea. Tällöin yksittäinen palvelu käyttää tunnistautumiseen erillistä palvelua, joka integroituu yrityksen tai yhteisön käyttäjäkantaan. Keskitettyä tunnistautumista käsitellään seuraavassa luvussa.