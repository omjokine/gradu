Lähteet: Howes, T. A., The Lightweight Directory Access Protocol: X.500 Lite. CITI
Technical Report 95–8, University of Michigan, 1995. \cite{howes} \\
rfc:t 4510-4513 (ainakin 4513 "Authentication Methods and Security Mechanisms" kiinnostaa)

Lightweight Directory Access Protocol (LDAP) on X.500 OSI-standardiin perustuva hakemistopalvelu, jota käytetään yleisesti käyttäjätiedon tallennukseen [TODO: lähde]. 1990-luvulla TCP/IP-mallin syrjäytettyä OSI-mallin, myös DAP kävi vanhanaikaiseksi \cite{howes}. Korvaajaksi on noussut LDAP, josta käytetään myös nimeä X.500 Lite \cite{howes}.

LDAP:ssa asiakassovellukset (directory user agent, DUA) keskustelevat puumalliin perustuvan hakemistopalvelimen (directory system agent, DSA) kanssa käyttäen määriteltyä protokollaa (directory access protocol, DAP) \cite{howes}. Asiakassovellukset voivat hakea hakemistopalvelimesta tietoa suodattimiin (filter) perustuvalla lukuoperaatiolla. Suodattimessa voidaan määritellä raja-arvot attribuutin arvolle tai hakea avainsanoilla attribuuteista.

LDAP-tietuille voidaan määritellä pakollisten attribuuttien (esim. etu- ja sukunimi) lisäksi valinnaisia attribuutteja. Tietueet on järjestetty puuhun niiden yksilöivän nimen (distinguished name, DN) mukaan ja ne voi olla hajautettu usealle palvelimelle. Suhteellinen nimi (relative distinguished name, RDN) identifioi tietueen omalla hierarkiatasollaan.

LDAP-tietueella voi olla tunnus sekä salasana ja LDAP-palvelinta voidaan käyttää käyttäjän tunnistautumiseen [TODO: lähde, ehkä rfc4513].

TODO: lisää tekstiä
