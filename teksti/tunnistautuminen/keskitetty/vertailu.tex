Etuna keskitetystä tunnistautumispalvelusta on järjestelmän tietoturvan ja tunnistautumisen luotettavuuden paraneminen. Keskitetyssä palvelussa käyttäjän tunnistautumisen luotettavuutta voidaan parantaa esimerkiksi vaatimalla tavallista web-sovellusta vahvempia salasanoja. On havaittu, että käyttäjät valitsevat vahvempia salasanoja palveluihin, jotka he kokevat tärkeiksi verrattuna vähemmän tärkeisiin web-palveluihin \cite{password_habits}.

Käyttäjien tunnistautumista voidaan vahvistaa myös lisävarmistuksilla, kuten puhelimen kautta tehtävällä todennuksella tai biometriikalla \cite{nisti}. Yksittäisen palvelun kohdalla vaikkapa matkapuhelinvarmennuksen käyttöönotto on hankalaa, mutta kun tunnusta käytetään useampaan palveluun, sen käyttö on helpommin perusteltavissa. Tällöin tunnistautumiseen tehtävät parannukset eivät vaadi muutoksia yksittäiseen web-sovellukseen, koska web-sovelluksen ja tunnistautumispalvelun välinen rajapinta ei muutu.

Keskitetyn ratkaisun myötä erillisten web-sovellusten ei tarvitse integroitua suoraan intranetin omistavan organisaation käyttäjähallintajärjestelmiin, vaan ainoastaan tunnistautumispalvelulla on pääsy sinne. Tällöin pelko käyttäjiä koskevan datan joutumisesta vääriin käsiin vähenee. Myös kriittisen käyttäjähallinnan integraatiopisteiden määrä vähenee, kun web-sovellukset eivät ota suoraan yhteyttä käyttäjähallintaan.

Jos käyttäjähallinta on toteutettu esimerkiksi SQL-tietokantana, joka ei tue tunnistautumista, keskitetyllä tunnistautumispalvelulla ratkaistaan tilanne, jossa käyttäjädata ja käyttäjiin liittyviä tunnistautumistietoja käyttäjätunnuksineen ja salasanoineen on kopioitu jokaiseen erilliseen web-sovellukseen \cite{billion_keys}. Tällöin aiemmin mainittuja synkronointiongelmia ei synny esimerkiksi henkilön jättäessä intranet-palvelua hallinnoivan organisaation.

Keskitettyyn tunnistautumispalveluun on määritelty web-palvelut, jotka käyttävät sitä tunnistamiseen. Tunnistautumispalvelu varmistaa pelkästään käyttäjän identiteetin, mutta sillä ei ole mitään tietoa yksittäisen web-palvelun toiminnasta. Esimerkiksi Facebook-tunnistautumista käyttävä valokuvien jakopalvelu ei suinkaan tallenna valokuvia Facebookiin, vaan omaan tietokantaansa \cite{web_resources}. Tällöin web-palvelussa täytyy olla käyttäjätietokanta, joka pitää sisällään käyttäjän id-numeron ulkopuolisessa tunnistautumispalvelussa ja yksittäiseen käyttäjään liittyviä sisäisiä resursseja, kuten valokuvia. Kun uusi käyttäjä tunnistautuu tunnistautumispalvelun kautta, lisätään sovelluksen käyttäjätietokantaan uusi rivi. Mitään tunnistautumisinformaatiota (salasana) ei tallenneta tietokantaan, vaan jokaisen käyttäjän tietokantariville lisätään tunnistautumispalvelussa yksilöivä tieto, esimerkiksi käyttäjätunnus tai id-numero.

Tunnistautumispalvelu toimii siis eräänlaisena ''palomuurina'' web-sovelluksen ja käyttäjähallinnan välillä. Sillä on oma rajapinta web-sovellukseen päin, joten organisaatiossa on mahdollista siirtyä esimerkiksi SQL-tietokannasta LDAP-tietokantaan ilman muutoksia web-sovelluksiin. Myös useiden käyttäjähallintajärjestelmien käyttö on mahdollista. Esimerkiksi Helsingin yliopiston tietojenkäsittelytieteen laitoksen intranetissä suoritettava tunnistautumispalvelu voi ensin hakea käyttäjää tietojenkäsittelytieteen laitoksen järjestelmästä ja jos käyttäjää ei löydy, haetaan käyttäjää yliopiston järjestelmästä.

Haittapuolena keskitetyssä ratkaisussa on käytetyn tunnuksen ja salasanan ''kalastelun'' käyminen houkuttelevaksi, koska niiden avulla hyökkääjä pääsee käyttäjän nimissä useaan palveluun tai jopa luomaan käyttäjän identiteetillä tunnuksia uusin palveluihin. Tästä syystä keskitettyyn tunnistautumispalvelun tietoturvan täytyy olla kunnossa, joten siinä käytettyjen teknologioidenkin täytyy olla luotettavia. Web-sovellusten ja tunnistautumispalvelun väliseen rajapintaan on kehitetty useita tunnistautumisprotokollia, joilla tunnistautuminen voidaan tehdä turvalliseksi ja käyttäjälle helpoksi \cite{open_identity}. Tunnistautumisprotokollat ovat käytössä avoimen Internetin puolella, ja esimerkiksi Facebookilla ja Googlella on ollut merkittävä rooli näiden protokollien syntyhistoriassa \cite{open_identity}. Näiden protokollien toteutusta käsitellään seuraavassa luvussa.