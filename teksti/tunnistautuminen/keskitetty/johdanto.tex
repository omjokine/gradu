Keskitetyn tunnistautumisen tarkoituksena on tarjota palvelu, jonka avulla käyttäjä voidaan tunnistaa erillisessä web-sovelluksessa ilman uusien tunnistautumistietojen (käyttäjätunnus ja salasana) luontia \cite{facebook}. Tunnistautumispalvelun ja sitä käyttävien web-sovellusten välillä on luottamussuhde, jolloin web-sovelluksen ei tarvitse säilyttää tunnistautumistietoja, vaan se voi luottaa tunnistautumispalvelun tunnistamiin käyttäjiin.

Keskitettyyn tunnistautumispalveluun on määritelty web-sovellukset, jotka käyttävät sitä tunnistamiseen. Tunnistautumispalvelu varmistaa pelkästään käyttäjän identiteetin, mutta sillä ei ole mitään tietoa yksittäisen web-sovelluksen toiminnasta. Esimerkiksi Facebook-tunnistautumista käyttävä valokuvien jakopalvelu ei suinkaan tallenna valokuvia Facebookiin, vaan omaan tietokantaansa \cite{web_resources}. Tällöin web-sovelluksissa täytyy olla käyttäjätietokanta, joka pitää sisällään käyttäjän id-numeron ulkopuolisessa tunnistautumispalvelussa ja yksittäiseen käyttäjään liittyviä sisäisiä resursseja kuten valokuvia. Kun uusi käyttäjä tunnistautuu tunnistautumispalvelun kautta, lisätään sovelluksen käyttäjätietokantaan uusi rivi. Mitään tunnistautumisinformaatiota (salasana) ei tallenneta tietokantaan, vaan jokaisen käyttäjän tietokantariville lisätään tunnistautumispalvelussa yksilöivä tieto, esimerkiksi käyttäjätunnus tai id-numero.

Monet web-sovellukset ovat ulkoistaneet käyttäjien tunnistautumisen Facebookin tai LinkedInin tapaisille toimijoille keskittyen vain oman sovelluksensa toteuttamiseen \cite{facebook}. Järjestely helpottaa sovelluskehittäjän työtä, koska kaikkea ei tarvitse tehdä itse. Myös loppukäyttäjän käyttökokemus paranee, sillä samalla käyttäjätunnuksella pääsee moneen eri palveluun.

Organisaatioiden sisäisissä palveluissa Facebookin tai LinkedInin kaltaisten palveluiden käyttö ei välttämättä tule kysymykseen, koska intranet-järjestelmien ylläpitäjät eivät halua siirtää käyttäjähallintaansa ulkopuolisen yrityksen haltuun. Tällöin vaihtoehtona on tarjota Facebookia tai LinkedIniä vastaava keskitetty tunnistautumispalvelu, joka on integroitu organisaation olemassa olevaan käyttäjähallintajärjestelmään.