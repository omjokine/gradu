Yksittäiset web-palvelut vaativat joissakin tapauksissa käyttäjän tunnistamisen, jotta ne voivat palvella kyseistä käyttäjää. Koska palveluperustaisten arkkitehtuurien mukaisesti toteutetun web-palveluiden on tarkoitus olla riippumattomia ympäristöstään, on tärkeää löytää yleiskäyttöinen ratkaisu tunnistautumisongelmaan.

Web-sovellukset sisältävät usein sellaista dataa, johon halutaan asettaa pääsyrajoituksia \cite{inside_the_identity_management_game}. Rajoitukset voivat koskea koko järjestelmää ja sen dataa, jolloin vain tietyt henkilöt pääsevät käyttämään järjestelmää. Tällaisia rajoituksia voidaan toteuttaa esimerkiksi rajaamalla pääsy dataan vain organisaation sisäverkosta. Usein kuitenkaan koko palvelun käytön rajaaminen organisaation sisälle ei riitä, vaan tarvitaan tarkempia pääsyrajoituksia. Käyttäjän tunnistaminen on tärkeää esimerkiksi sähköpostijärjestelmässä, koska sähköpostit saavat näkyä vain lähettäjälle ja vastaanottajalle.

Aliluvussa 3.1 käydään läpi tunnistautumisen osatekijät: mitä tarkoitetaan puhuttaessa tunnistautumisesta tai laajemmin pääsynvalvonnasta. Luku 3.2 käsittelee ongelmia joita erityisesti intranet-palveluita ylläpitävillä organisaatioilla on tunnistautumisen toteuttamisessa. Luku 3.3 käsittelee web-palveluiden tunnistautumisen yleisellä tasolla ja luvussa 3.4 esitellään keskitetty tunnistautumispalvelu, jolla tunnistautuminen voidaan suorittaa palveluperustaisten arkkitehtuurien periaatteiden mukaisesti.