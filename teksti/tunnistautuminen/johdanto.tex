Web-sovelluksia ajetaan tyypillisesti avoimessa Internet-verkossa ja ne sisältää dataa, johon halutaan asettaa pääsyrajoituksia [TODO: lähde]. Rajoitukset voivat koskea koko järjestelmää ja sen dataa, jolloin vain tietyt henkilöt pääsevät järjestelmään. Tällaisia rajoituksia voidaan toteuttaa esimerkiksi rajaamalla pääsy dataan vain organisaation sisäverkosta. Usein kuitenkaan käytön rajaaminen organisaation sisälle ei pelkästään riitä, vaan tarvitaan tarkempia pääsyrajoituksia. Esimerkiksi sähköpostijärjestelmässä sähköpostit saavat näkyä vain lähettäjälle ja vastaanottajalle. Pääsyrajoitusten vuoksi käyttäjä täytyy tunnistaa, jotta hän näkee vain hänelle tarkoitetun datan.

Tavallinen tapa tunnistaa käyttäjä web-sovelluksessa on kysyä tunnusta sekä salasanaa ja verrata niitä järjestelmään tallennettuihin tietoihin. Vaihtoehtoisia menetelmiä on kehitetty ja ne voivat perustua esimerkiksi biometriikkaan (sormenjäljet, silmän iiris) ja näillä menetelmillä voidaan parantaa tunnistamisen luotettavuutta [TODO: lähde]. Tutkimuksen kannalta oleelliset ongelmat ovat kuitenkin yhteisiä tunnistautumismenetelmästä riippumatta, joten yksinkertaisuuden vuoksi tutkielmassa keskitytään vain tunnuksella ja salasanalla tapahtuvaan tunnistautumiseen.

Palveluperustaisten arkkitehtuurien myötä organisaation sisällä saattaa olla useita web-palveluita, jotka haluavat myös tunnistaa käyttäjän. Tässä luvussa pureudutaan tähän liittyvään ongelmakenttään. Vaikka web-sovellusten määrä järjestelmässä kasvaa, halutaan käyttäjien tunnistetiedot pitää keskitettynä, jotta vältytään erilaisilta tietoturva ja synkronointiongelmilta.

Luvun ensimmäisessä aliluvussa käydään läpi tunnistautumiseen liittyvää käsitteistöä ja määritellään mitä tarkoitetaan puhuttaessa tunnistautumisesta tai laajemmin pääsynvalvonnasta. Luvussa 3.2 käydään läpi toimintaympäristöä, jossa tunnistautumista tarvitaan ja johon myöhemmissä luvuissa toteutettava prototyyppi soveltuu. Luku 3.3 käsittelee web-palveluiden tunnistautumisen nykytilannetta ja luvussa 3.4 esitellään keskitetty tunnistautumispalvelu, jolla nykytilanteessa syntyviä ongelmia voidaan ratkaista. Viimeinen luku on yhteenvetoluku.