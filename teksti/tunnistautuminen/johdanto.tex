Yksittäiset web-palvelut vaativat joissakin tapauksissa käyttäjän tunnistautumisen, jotta ne voivat palvella kyseistä käyttäjää. Koska web-palveluiden on tarkoitus olla riippumattomia ympäristöstä, on yleiskäyttöisten ratkaisujen löytäminen tunnistautumisongelmaan tärkeää.

Web-sovelluksia suoritetaan tyypillisesti avoimessa Internet-verkossa, ja ne sisältävät dataa, johon halutaan asettaa pääsyrajoituksia \cite{inside_the_identity_management_game}. Rajoitukset voivat koskea koko järjestelmää ja sen dataa, jolloin vain tietyt henkilöt pääsevät järjestelmään. Tällaisia rajoituksia voidaan toteuttaa esimerkiksi rajaamalla pääsy dataan vain organisaation sisäverkosta. Usein kuitenkaan käytön rajaaminen organisaation sisälle ei pelkästään riitä, vaan tarvitaan tarkempia pääsyrajoituksia. Esimerkiksi sähköpostijärjestelmässä sähköpostit saavat näkyä vain lähettäjälle ja vastaanottajalle. Pääsyrajoitusten vuoksi käyttäjä täytyy tunnistaa, jotta hän näkee vain hänelle tarkoitetun datan.

Aliluvussa 3.1 käydään läpi tunnistautumisen osatekijät: mitä tarkoitetaan puhuttaessa tunnistautumisesta tai laajemmin pääsynvalvonnasta. Luvussa 3.2 käydään läpi toimintaympäristöä, jossa tunnistautumista tarvitaan. Luku 3.3 käsittelee web-palveluiden tunnistautumisen nykytilannetta, ja luvussa 3.4 esitellään keskitetty tunnistautumispalvelu, jolla nykyisin yleisiä ongelmia voidaan ratkaista.