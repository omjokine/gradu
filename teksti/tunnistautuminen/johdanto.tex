Web-sovelluksia ajetaan tyypillisesti avoimessa Internet-verkossa ja ne sisältää dataa, johon halutaan asettaa pääsyrajoituksia [TODO: lähde]. Rajoitukset voivat koskea koko järjestelmää ja sen dataa, jolloin vain tietyt henkilöt pääsevät järjestelmään. Toisaalta järjestelmän käyttöoikeusrajoitukset ei riitä, vaan tarvitaan tarkempia pääsyrajoituksia. Esimerkiksi sähköpostijärjestelmässä sähköpostit saavat näkyä vain lähettäjälle ja vastaanottajalle. Pääsyrajoitusten vuoksi käyttäjä täytyy tunnistaa ja tunnistamisen jälkeen tehdä päätös mitä käyttäjä voi järjestelmän sisällä tehdä.

Tavallinen tapa tunnistaa käyttäjä web-sovelluksessa on kysyä tunnusta sekä salasanaa ja verrata niitä järjestelmään tallennettuihin tietoihin. Vaihtoehtoisia menetelmiä on kehitetty ja ne voivat perustua esimerkiksi biometriikkaan (sormenjäljet, silmän iiris) ja näillä menetelmillä voidaan parantaa tunnistamisen luotettavuutta [TODO: lähde]. Tutkimuksen kannalta oleelliset ongelmat ovat kuitenkin yhteisiä tunnistautumismenetelmästä riippumatta, joten yksinkertaisuuden vuoksi tutkielmassa keskitytään vain tunnuksella ja salasanalla tapahtuvaan tunnistautumiseen.

Tämän luvun ensimmäisessä aliluvussa käydään läpi tunnistautumiseen liittyvää käsitteistöä. Luvussa 3.2 käydään läpi toimintaympäristöä, jossa tunnistautumista tarvitaan ja johon myöhemmissä luvuissa toteutettava prototyyppi on tarkoitettu. Luku 3.3 käsittelee käyttäjädatan tallennukseen teknologioita ja luku 3.4 on yhteenvetoluku.