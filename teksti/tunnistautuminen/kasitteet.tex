Palveluiden turvallisuus koostuu kolmesta osatekijästä: tunnistautumisesta (authentication), pääsynvalvonnasta (access control) ja auditoinnista (audit) \cite{sandhu}. Tunnistautumisessa käyttäjän identiteetti varmistetaan esimerkiksi käyttäjätunnuksen ja salasanan avulla. Tämän jälkeen pääsynvalvonta tarkistaa, onko kyseisellä käyttäjällä oikeutta tehdä pyytämänsä toiminto. Auditoinnissa analysoidaan järjestelmän tuottamaa dataa, esimerkiksi lokitiedostoja, aktiivisesti ja käyttäjän pääsy järjestelmään voidaan estää, jos luvatonta käyttöä esiintyy. Tämän tutkielman pai\-no\-pis\-te on käyttäjän tunnistautumisessa ja osittain myös pääsynvalvonnassa, mutta auditointi ei kuulu tutkielman aihepiiriin.

Tunnistautumista ja pääsynvalvontaa ei voida täysin erottaa toisistaan, koska käyttäjään liittyy attribuutteja, joiden perusteella pääsynvalvonta voidaan tehdä järjestelmän sisällä. Pääsynvalvonta voidaan tehdä esimerkiksi roolipohjaisena, jolloin käyttäjän rooli organisaatiossa (esimerkiksi esimies/alainen) vaikuttaa siihen, mitä palveluita tai palveluiden osia hänellä on oikeus käyttää.