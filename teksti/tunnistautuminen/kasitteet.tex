Palveluiden turvallisuus koostuu kolmesta tekijästä: tunnistautumisesta (authentication), pääsynvalvonnasta (access control) ja auditoinnista (audit) \cite{sandhu}. Tunnistautumisessa käyttäjän identiteetti varmistetaan, esimerkiksi käyttäjätunnuksen ja salasanan avulla. Tämän jälkeen pääsynvalvonta tarkistaa onko kyseisellä käyttäjällä oikeutta tehdä pyytämäänsä toimintoa. Auditoinnissa analysoidaan järjestelmän tuottamaa dataa, esimerkiksi lokitiedostoja, aktiivisesti ja käyttäjän pääsy järjestelmään voidaan estää, jos luvatonta käyttöä esiintyy \cite{sandhu}.

Tämän tutkielman painopiste on käyttäjän tunnistautumisessa ja osittain myös pääsynvalvonnassa, auditointi ei kuulu tutkielman aihepiiriin. Tutkielmassa käytetään termiä pääsynvalvonta sen suppeassa merkityksessä, toisinaan kirjallisuudessa pääsynvalvontaa pidetään kattoterminä, joka pitää sisällään tunnistautumisen, valtuutuksen (authorization) ja auditoinnin [TODO: joku lähde, jossa näin on].

Tunnistautumista ja pääsynvalvontaa ei voida täysin erottaa toisistaan. Käyttäjään liittyy attribuutteja, joiden perusteella pääsynvalvonta voidaan tehdä järjestelmässä. Pääsynvalvonta voidaan tehdä esimerkiksi roolipohjaisena pääsynvalvontana, jolloin käyttäjän rooli organisaatiossa (esimerkiksi esimies/alainen) vaikuttaa siihen mitä palveluita, tai palveluiden osia, hänellä on oikeus käyttää. Tutkielmassa ei mennä pääsynvalvonnan yksityiskohtiin, mutta toteutettavan prototyypin täytyy ottaa huomioon myös käyttäjiin liittyvä metadata, jota käytetään pääsynvalvonnassa.