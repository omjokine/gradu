Palveluiden turvallisuus koostuu kolmesta osatekijästä: tunnistautumisesta (authentication), pääsynvalvonnasta (access control) ja auditoinnista (audit) \cite{sandhu}. Tunnistautumisessa käyttäjän identiteetti varmistetaan esimerkiksi käyttäjätunnuksen ja salasanan avulla. Tämän jälkeen pääsynvalvonta tarkistaa, onko kyseisellä käyttäjällä oikeutta suorittaa pyydetty toiminto. Auditoinnissa analysoidaan aktiivisesti järjestelmän tuottamaa dataa, kuten lokitiedostoja. Käyttäjän pääsy järjestelmään voidaan estää, jos luvatonta käyttöä esiintyy. Tämän tutkielman pai\-no\-pis\-te on käyttäjän tunnistautumisessa ja osittain myös pääsynvalvonnassa. Auditointi ei kuulu tutkielman aihepiiriin.

Tunnistautumista ja pääsynvalvontaa ei voida täysin erottaa toisistaan, koska käyttäjään liittyy attribuutteja, joiden perusteella pääsynvalvonta voidaan tehdä järjestelmän sisällä. Pääsynvalvonta voidaan tehdä esimerkiksi roolipohjaisena, jolloin käyttäjän rooli organisaatiossa (esimerkiksi esimies/alainen) vaikuttaa siihen, mitä palveluita tai palveluiden osia hänellä on oikeus käyttää \cite{sandhu_rbac}.

Tunnistautumismenetelmät voidaan jakaa kolmeen ryhmään: jotain, jonka käyttäjä tuntee, käyttäjä omistaa tai joka on käyttäjän fyysinen ominaisuus \cite{nisti}. Käyttäjän tuntema asia on yleensä salasana, jonka hän määrittelee itselleen. Käyttäjän omistama asia voi olla matkapuhelin, johon lähetettävää tunnuslukua voidaan käyttää tunnistautumisessa \cite{5336918}. Käyttäjän fyysinen ominaisuus tarkoittaa biometristä dataa, kuten sormenjälkeä tai silmän iiristä, joka on jokaisella ihmisellä yksilöllinen. Mitä useampaa menetelmää käytetään, sitä luotettavampi tunnistautuminen on \cite{nisti}. Toisaalta taas usean tunnistautumismenetelmän käyttö lisää tunnistautumisen monimutkaisuutta ja esimerkiksi sormenjälkien tai silmän iiriksen lukeminen on usein mahdotonta puuttellisten päätelaitteiden takia.