Monet Internet-palvelut ovat ulkoistaneet käyttäjien tunnistautumisen Facebookin ja LinkedInin tapaisille toimijoille, keskittyen vain oman palvelun toteuttamiseen. Järjestely helpottaa toisaalta sovelluskehittäjän arkea, koska kaikkea ei tarvitse tehdä itse, mutta myös loppukäyttäjän käyttökokemus paranee, sillä samalla verkkoidentiteetillä pääsee moneen eri palveluun. Muistettavien tunnusten ja salasanojen määrä vähenee ja käyttäjä kokee mielekkäänä tunnistautumisen turvallisuustason noston. Yksittäisen palvelun kohdalla vaikkapa matkapuhelinvarmennuksen käyttöönotto on hankalaa, mutta kun tunnusta käytetään useampaan palveluun, sen käyttö on helpommin perusteltavissa.

Organisaatiot eivät monista syistä voi tai halua ulkoistaa käyttäjähallintaa, vaan sisäisten palveluiden käyttö halutaan rajoittaa vain henkilöille, joilla on tunnus organisaation omassa järjestelmässä. Ratkaisuksi esitetään palvelusuuntautuneiden arkkitehtuurien mukaista ratkaisua, jossa tunnistautuminen keskitetään erilliselle tunnistautumispalvelulle. Tunnistautumispalvelu integroituu organisaation käyttäjähallintaan, jolloin yksittäisten sovellusten ei tarvitse sitä tehdä. Koko järjestelmän tietoturvaa paranee, koska käyttäjätunnukset ja salasanat pysyvät yhdessä paikassa, eikä niitä tarvitse kopioida ympäri järjestelmää.

Seuraavassa luvussa keskitytään keskitetyssä tunnistautumispalvelussa käytettyihin tekniikoihin. Tekniikat ovat samoja, jotka ovat tälläkin hetkellä käytössä avoimessa Internetissä. Tekniikoiden soveltuvuutta organisaatioiden sisäiseen käyttöön tutkitaan ja annetaan arvio eri protokollien soveltuvuudesta toteutettavaan prototyyppiin.