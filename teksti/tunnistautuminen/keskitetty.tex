Monet web-palvelut ovat ulkoistaneet käyttäjien tunnistautumisen Facebookin ja LinkedInin tapaisille toimijoille, keskittyen vain oman sovelluksensa toteuttamiseen \cite{facebook}. Järjestely helpottaa toisaalta sovelluskehittäjän arkea, koska kaikkea ei tarvitse tehdä itse, mutta myös loppukäyttäjän käyttökokemus paranee, sillä samalla verkkoidentiteetillä pääsee moneen eri palveluun.

Organisaatioiden sisäisissä palveluissa Facebookin tai LinkedInin kaltaisten palveluiden käyttö ei välttämättä tule kysymykseen. Intranet-järjestelmien ylläpitäjät eivät mahdollisesti halua siirtää käyttäjähallintaansa ulkopuolisen yrityksen haltuun. Tällöin vaihtoehtona on tarjota Facebookia tai LinkedIniä vastaava keskitetty tunnistautumispalvelu, joka on integroitu organisaation olemassa olevaan käyttäjähallintajärjestelmään.

Keskitetyn tunnistautumisen tarkoituksena on tarjota palvelu, jota vasten käyttäjä voidaan tunnistaa erillisestä web-sovelluksesta ilman uuden identiteetin luontia \cite{facebook}. Tunnistautumispalvelun ja sitä käyttävien web-sovellusten välillä on luottamussuhde, jolloin web-sovellukseen ei tarvitse luoda omaa käyttäjille tunnistautumistietoja (käyttäjätunnus/salasana), vaan se voi luottaa tunnistautumispalvelun tunnistamiin käyttäjiin.

Etuna tunnistautumispalvelusta on järjestelmän tietoturvan ja tunnistautumisen luotettavuuden paraneminen. Keskitetyssä palvelussa käyttäjän tunnistautumisen luotettavuutta voidaan parantaa esimerkiksi vaatimalla tavallista web-sovellusta vahvempia salasanoja. Toisaalta on havaittu, että käyttäjät valitsevat vahvempia salasanoja palveluihin, jotka he kokevat tärkeiksi verrattuna vähemmän tärkeisiin web-palveluihin \cite{password_habits}.

Käyttäjien tunnistautumista voidaan vahvistaa myös lisävarmistuksilla, kuten puhelimen kautta tehtävällä todennuksella tai biometriikalla \cite{nisti}. Yksittäisen palvelun kohdalla vaikkapa matkapuhelinvarmennuksen käyttöönotto on hankalaa, mutta kun tunnusta käytetään useampaan palveluun, sen käyttö on helpommin perusteltavissa. Tunnistautumiseen tehtävät parannukset eivät vaadi muutoksia yksittäiseen web-sovellukseen, koska web-sovelluksen ja tunnistautumispalvelun välinen rajapinta ei muutu.

Keskitetyn ratkaisun myötä erillisten web-sovellusten ei tarvitse integroitua suoraan organisaation käyttäjähallintajärjestelmiin, vaan pelkästään tunnistautumispalvelulla on pääsy sinne. Pelko käyttäjiä koskevan datan joutumisesta vääriin käsiin vähenee, koska vain tunnistautumispalvelu integroituu käyttäjähallintajärjetelmään. Tällöin kriittisen käyttäjähallinnan integraatiopisteiden määrä vähenee, kun web-sovellukset eivät ota suoraan yhteyttä käyttäjähallintaan.

Jos käyttäjähallinta on esimerkiksi SQL-tietokanta, joka ei tue tunnistautumista, keskitetyllä tunnistautumispalvelulla ratkaistaan tilanne, jossa käyttäjädata ja käyttäjiin liittyviä tunnistautumistietoja käyttäjätunnuksineen ja salasanoineen on kopioitu jokaiseen erilliseen web-sovellukseen. Tällöin ei synny aiemmin mainittuja synkronointiongelmia esimerkiksi henkilön jättäessä organisaation.

Tunnistautumispalvelu toimii siis eräänlaisena ''palomuurina'' web-sovelluksen ja käyttäjähallinnan välillä. Sillä on oma rajapinta web-sovellukseen päin, joten organisaatiossa on mahdollista siirtyä esimerkiksi SQL-tietokannasta LDAP-tietokantaan ilman muutoksia web-sovelluksiin. Myös useiden käyttäjähallintajärjestelmien käyttö on mahdollista. Esimerkiksi tietojenkäsittelytieteen laitoksen intranetissä suoritettava tunnistautumispalvelu voi ensin hakea käyttäjää tietojenkäsittelytieteen laitoksen järjestelmästä ja jos käyttäjää ei löydy, haetaan käyttäjää yliopiston järjestelmästä.

Kääntöpuolena keskitetyssä ratkaisussa on käytetyn tunnuksen ja salasanan kalastelun käyminen houkuttelevaksi, koska niiden avulla hyökkääjä pääsee käyttäjän nimissä useaan palveluun tai jopa luomaan käyttäjän identiteetillä tunnuksia uusin palveluihin. Tästä syystä keskitettyyn tunnistautumispalveluun kohdistuu normaalia web-palvelua suuremmat odotukset tietoturvalle, joten siinä käytettyjen teknologioiden täytyy olla luotettavia.

Web-sovellusten ja tunnistautumispalvelun väliseen rajapintaan on ehitetty useita tunnistautumisprotokollia, joilla tunnistautuminen voidaan tehdä turvalliseksi ja käyttäjälle helpoksi \cite{open_identity}. Tunnistautumisprotokollat ovat käytössä avoimen Internetin puolella, ja esimerkiksi Facebookilla ja Googlella on ollut merkittävä rooli näiden protokollien syntyhistoriassa \cite{open_identity}. Näitä protokollia käsitellään tarkemmin seuraavassa luvussa.