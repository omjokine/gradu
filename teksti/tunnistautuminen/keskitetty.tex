Keskitetyn tunnistautumisen tarkoituksena on tarjota palvelu, jota vasten käyttäjä voidaan tunnistaa erillisestä web-palvelusta ilman uuden identiteetin luontia. Tunnistautumispalvelun ja sitä käyttävien web-palveluiden välillä on luottamussuhde, jolloin web-palveluun ei tarvitse luoda omaa käyttäjille tunnistautumistietoja (käyttäjätunnus/salasana), vaan se voi luottaa tunnistautumispalvelun tunnistamiin käyttäjiin.

Tunnistautumispalvelulla voidaan parantaa järjestelmien tietoturvaa ja parantaa tunnistautumisen luotettavuutta. Käyttäjät valitsevat vahvempia salasanoja palveluihin, jotka he kokeavat tärkeiksi, kuten sähköpostiin, verrattuna vähemmän tärkeisiin web-palveluihin \cite{password_habits}. Keskitetyssä palvelussa käyttäjän tunnistautumisen luotettavuutta voidaan parantaa esimerkiksi vaatimalla normaalia web-palvelua vahvempia salasanoja. Käyttäjien todennusta voidaan vahvistaa myös lisävarmistuksilla, kuten erilaisilla tunnuslukulistoilla tai puhelimen kautta tehtävällä todennuksella. Käyttäjä ei tarvitse organisaation jokaiseen järjestelmään eri tunnistautumistietoja, vaan yksi riittää koko organisaation laajuudessa.

Kääntöpuolena keskitetyssä ratkaisussa on käytetyn tunnuksen ja salasanan kalastelun käyminen houkuttelevaksi, koska sen avulla pääsee käyttäjän nimissä useaan palveluun tai jopa luomaan käyttäjän identiteetillä tunnuksia uusin palveluihin. Tästä syystä palveluun kohdistuu normaalia web-palvelua suuremmat odotukset tietoturvalle, joten käytettyjen teknologioiden täytyy olla luotettavia. Keskitetyn tunnistautumisen teknologioita käydään läpi seuraavassa luvussa ja pohditaan niiden soveltamista prototyypissä.