Palveluperustaisten arkkitehtuurien myötä organisaation sisällä saattaa olla useita web-palveluita, jotka haluavat myös tunnistaa käyttäjän. Tässä luvussa pureudutaan tähän liittyvään ongelmakenttään. Vaikka web-sovellusten määrä järjestelmässä kasvaa, halutaan käyttäjien tunnistetiedot pitää keskitettynä, jotta vältytään erilaisilta tietoturva- ja synkronointiongelmilta.

Tavallinen tapa tunnistaa käyttäjä web-sovelluksessa on kysyä tunnusta sekä salasanaa ja verrata niitä järjestelmään tallennettuihin tietoihin. Vaihtoehtoisia menetelmiä on kehitetty ja ne voivat perustua esimerkiksi biometriikkaan (sormenjäljet, silmän iiris) ja näillä menetelmillä voidaan parantaa tunnistamisen luotettavuutta [TODO: lähde]. Tutkimuksen kannalta oleelliset ongelmat ovat kuitenkin yhteisiä tunnistautumismenetelmästä riippumatta, joten yksinkertaisuuden vuoksi tutkielmassa keskitytään vain tunnuksella ja salasanalla tapahtuvaan tunnistautumiseen.

Web-palveluiden tunnistautuminen tehdään sovelluksen väliohjelmakerroksessa (kuva \ref{dispatcher}). Tällöin varsinaisen sovelluslogiikan ei tarvitse tietää tunnistautumisen tekniikasta mitään, sovellusohjelmalle välitetään vain kirjautuneen käyttäjän tiedot. Lähestymistapa mahdollistaa tunnistautumismekanismin vaihtamisen ilman muutosta sovellusohjelman logiikkaan. Väliohjelmistolla ja sovelluslogiikalla on sovittu rajapinta, jonka mukaan käyttäjän tiedot päätyvät sovellusohjelmalle.

Tunnistautumisen yhteydessä käyttäjä ohjataan sivulle, jossa olevaan lomakkeeseen hän syöttää käyttäjätunnuksen ja salasanan. Sivuohjauksen yhteydessä web-palvelu lisää HTTP-pyynnön parametreihin tiedon siitä, minne käyttäjä ohjataan onnistuneen kirjautumisen jälkeen [TODO: lähde]. Kirjautumisen jälkeen web-palvelu saa käyttäjän perustiedot (esim id-numero ja nimi) ja käyttäjään liittyvän metadatan, jota käytetään hyväksi pääsynvalvonnassa.

Tunnistautumispalveluun on määritelty web-palvelut, jotka käyttävät sitä tunnistamiseen. Tunnistautumispalvelu, nimensä mukaisesti, hoitaa pelkästään käyttäjän identiteetin varmistamisen, sillä ei ole mitään tietoa yksittäisen web-palvelun toiminnasta. Esimerkiksi Facebook-tunnistautumista käyttävä valokuvien jakopalvelu ei suinkaan tallenna valokuvia Facebookiin, vaan omaan tietokantaansa. Tällöin web-palvelussa täytyy olla käyttäjätietokanta, joka pitää sisällään käyttäjän id-numeron ulkopuolisessa tunnistautumispalvelussa ja käyttäjään liittyviä sisäisiä resursseja, kuten valokuvia. Kun uusi käyttäjä tunnistautuu hyväksytysti tunnistautumispalvelun kautta, lisätään sovelluksen käyttäjätietokantaan uusi rivi. Mitään tunnistautumisinformaatiota (tunnus/salasana) ei tallenneta tietokantaan, vaan tunnistautumispalvelulta tulevaa id-numeroa verrataan tietokannassa olevaan.

Organisaatioiden sisäisissä palveluissa Facebookin tai LinkedInin kaltaisten palveluiden käyttö ei välttämättä tule kysymykseen. Intranet-järjestelmien ylläpitäjät eivät mahdollisesti halua siirtää käyttäjähallintaansa ulkopuolisen yrityksen haltuun, vaan olemassaolevia hallintajärjestelmiä halutaan käyttää. Seuraavassa luvussa käydään läpi ongelmia, joita nykyisten käyttäjähallintajärjestelmien integrointi palvelusuuntautuneisiin arkkitehtuureihin tuottaa ja esitetään ratkaisuksi keskitettyä tunnistautumispalvelua.