Web-palveluissa käyttäjä tunnistetaan sovelluksen väliohjelmakerroksessa (kuva \ref{dispatcher}). Tällöin varsinaisen sovelluslogiikan ei tarvitse tietää tunnistautumisen tekniikasta mitään, vaan sovellusohjelmalle välitetään vain kirjautuneen käyttäjän tiedot. Lähestymistapa mahdollistaa tunnistautumismekanismin vaihtamisen ilman muutosta sovellusohjelman logiikkaan. Väli- ja sovellusohjelmistoilla on tällöin sovittu rajapinta, jonka mukaan käyttäjän tiedot päätyvät sovellusohjelmalle \cite{django}.

Tunnistautumisen yhteydessä käyttäjä ohjataan sivulle, jossa olevaan lomakkeeseen hän syöttää käyttäjätunnuksensa ja salasanansa, joita verrataan järjestelmään tallennettuihin tietoihin. Sivuohjauksen yhteydessä web-palvelu lisää HTTP-pyynnön parametreihin tiedon siitä, minne käyttäjä ohjataan onnistuneen kirjautumisen jälkeen \cite{oauth2_0}. Kirjautumisen jälkeen web-palvelu saa käyttäjän perustiedot (esim id-numeron ja nimen) ja käyttäjään liittyvän metadatan, jota käytetään hyväksi pääsynvalvonnassa.

Web-sovelluksen kannalta ei ole väliä minne käyttäjän tunnistetiedot on tallennettu tai missä tunnistaminen tehdään. Yksinkertaisin ratkaisu on tallentaa käyttäjän tunnistetiedot samaan tietokantaan kuin web-sovelluksen muukin tieto. Tästä seuraa kuitenkin edellisessä luvussa mainittuja ongelmia, kun samoja tunnistetietoja tarvitsevia web-sovelluksia on useampia. Ongelma voidaan ratkaista keskittämällä käyttäjän tunnistaminen omaan tunnistautumispalveluun.