Web-palveluissa tunnistautuminen tehdään sovelluksen väliohjelmakerroksessa (kuva \ref{dispatcher}). Tällöin varsinaisen sovelluslogiikan ei tarvitse tietää tunnistautumisen tekniikasta mitään, vaan sovellusohjelmalle välitetään vain kirjautuneen käyttäjän tiedot. Lähestymistapa mahdollistaa tunnistautumismekanismin vaihtamisen ilman muutosta sovellusohjelman logiikkaan. Väli- ja sovellusohjelmalla on tällöin sovittu rajapinta, jonka mukaan käyttäjän tiedot päätyvät sovellusohjelmalle \cite{django}.

Tunnistautumisen yhteydessä käyttäjä ohjataan sivulle, jolla olevaan lomakkeeseen hän syöttää käyttäjätunnuksen ja salasanan, joita verrataan järjestelmään tallennettuihin tietoihin. Sivuohjauksen yhteydessä web-palvelu lisää HTTP-pyynnön parametreihin tiedon siitä, minne käyttäjä ohjataan onnistuneen kirjautumisen jälkeen \cite{oauth2_0}. Kirjautumisen jälkeen web-palvelu saa käyttäjän perustiedot (esim id-numeron ja nimen) ja käyttäjään liittyvän metadatan, jota käytetään hyväksi pääsynvalvonnassa.

Tunnistautumismenetelmät voidaan jakaa kolmeen ryhmään \cite{nisti}. Tunnistautumiseen voidaan käyttää ensinnäkin jotain käyttäjän tuntemaa asiaa, yleisesti salasanaa. Toinen vaihtoehto on käyttää jotain, mitä käyttäjä omistaa. Tällainen on esimerkiksi matkapuhelin, johon lähetettävää tunnuslukua voidaan käyttää tunnistautumisessa \cite{5336918}. Kolmas vaihtoehto on käyttää jotain käyttäjän fyysistä ominaisuutta, käytännössä biometristä dataa, kuten sormenjälkeä tai silmän iiristä, jotka ovat jokaisella ihmisellä yksilölliset. Näillä menetelmillä voidaan parantaa tunnistamisen luotettavuutta \cite{nisti}.

Tunnistautumispalveluun on määritelty web-palvelut, jotka käyttävät sitä tunnistamiseen. Tunnistautumispalvelu hoitaa, nimensä mukaisesti, pelkästään käyttäjän identiteetin varmistamisen, mutta sillä ei ole mitään tietoa yksittäisen web-palvelun toiminnasta. Esimerkiksi Facebook-tunnistautumista käyttävä valokuvien jakopalvelu ei suinkaan tallenna valokuvia Facebookiin, vaan omaan tietokantaansa \cite{web_resources}. Tällöin web-palvelussa täytyy olla käyttäjätietokanta, joka pitää sisällään käyttäjän id-numeron ulkopuolisessa tunnistautumispalvelussa ja käyttäjään liittyviä sisäisiä resursseja, kuten valokuvia. Kun uusi käyttäjä tunnistautuu hyväksytysti tunnistautumispalvelun kautta, lisätään sovelluksen käyttäjätietokantaan uusi rivi. Mitään tunnistautumisinformaatiota (salasana) ei tallenneta tietokantaan, vaan jokaisen käyttäjän tietokantariville lisätään tunnistautumispalvelussa yksilöivä tieto, esimerkiksi käyttäjätunnus tai id-numero.