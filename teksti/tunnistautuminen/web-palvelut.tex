Web-palveluissa käyttäjän tunnistautuminen tehdään sovelluksen väliohjelmakerroksessa (kuva \ref{dispatcher}). Tällöin varsinaisen sovelluslogiikan ei tarvitse tietää tunnistautumisen tekniikasta mitään, vaan sovellusohjelmalle välitetään vain kirjautuneen käyttäjän tiedot. Lähestymistapa mahdollistaa tunnistautumismekanismin vaihtamisen ilman muutosta sovellusohjelman logiikkaan. Väli- ja sovellusohjelmistoilla on tällöin sovittu rajapinta, jonka mukaan käyttäjän tiedot päätyvät sovellusohjelmalle \cite{django}.

Tunnistautumisen yhteydessä käyttäjä ohjataan sivulle, jolla olevaan lomakkeeseen hän syöttää käyttäjätunnuksen ja salasanan, joita verrataan järjestelmään tallennettuihin tietoihin. Sivuohjauksen yhteydessä web-palvelu lisää HTTP-pyynnön parametreihin tiedon siitä, minne käyttäjä ohjataan onnistuneen kirjautumisen jälkeen \cite{oauth2_0}. Kirjautumisen jälkeen web-palvelu saa käyttäjän perustiedot (esim id-numeron ja nimen) ja käyttäjään liittyvän metadatan, jota käytetään hyväksi pääsynvalvonnassa.

Tunnistautumismenetelmät voidaan jakaa kolmeen ryhmään: jotain, jonka käyttäjä tuntee, käyttäjä omistaa tai joka on käyttäjän fyysinen ominaisuus \cite{nisti}. Käyttäjän tuntemaa asia on yleensä salasana, jonka hän määrittelee itselleen. Käyttäjän omistama asia voi olla matkapuhelin, johon lähetettävää tunnuslukua voidaan käyttää tunnistautumisessa \cite{5336918}. Käyttäjän fyysisinen ominaisuus tarkoittaa biometristä dataa, kuten sormenjälkeä tai silmän iiristä, jotka ovat jokaisella ihmisellä yksilölliset. Mitä useampaa menetelmää käytetään, sitä luotettavampi tunnistautuminen on \cite{nisti}. Toisaalta taas usean tunnistautumismenetelmän käyttö lisää tunnistautumisen monimutkaisuutta ja esimerkiksi sormenjälkien tai silmän iiriksen lukeminen on usein mahdotonta puuttellisten päätelaitteiden takia.