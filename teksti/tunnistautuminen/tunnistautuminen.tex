Miten tunnistetaan perinteisesti?

Käyttäjädatan tallennus
- passwd (ts tiedosto levyllä tms)
- tietokanta
- AD-kannat (LDAP)

---> käyttäjän datan abstraktoinnin tarve, ehkä mainita, että käyttäjädatan backendejä voi käytännössä olla monta erilaista. Esim. LDAP:n rinnalla tietokannat.

Tunnistautumiseen liittyvien käsitteiden läpikäynti ennen protokollien yksityiskohtaista esittelyä auttaa tunnistautumiseen liittyvien periaatteiden hahmottamista. Käsitteet ovat yleisluontoisia ja eivätkä kosketa vain tiettyjä protokollaa. Protokollien yhteydessä käytetään käsitteitä asiakasohjelma, tunnistautumispalvelu, suojattu resurssi, valtuutustieto (credentials), valtuutusavain (authorization code) ja pääsyvaltuutus (access token) \cite{nisti}.

Asiakasohjelmalla tarkoitetaan web-palvelun käyttäjän pääteohjelmaa, jolla hän kirjautuu web-palveluun käyttäen keskitettyä tunnistautumispalvelua. Käytännössä asiakasohjelma on web-palvelun tapauksessa käyttäjän WWW-selain, joka pystyy tekemään uudelleenohjauksia sivustolta toiselle. Uudelleenohjaus on HTTP-protokollan perustoiminnallisuutta, joten mikä tahansa HTTP/1.1-standardin WWW-selain käy asiakasohjelmaksi \cite{rfc2616}.

Tunnistautumispalvelu on web-palvelu, johon käyttäjä ohjataan tekemään tunnistautuminen. Onnistuneen tunnistautumisen jälkeen tunnistautumispalvelu ohjaa asi\-a\-kas\-oh\-jel\-man takaisin tunnistautumista pyytäneen palvelun määrittelemään osoitteeseen \cite{nisti}. Avoimen Internetin puolella tunnistautumispalvelu voi olla esimerkiksi Facebook tai LinkedIn.

Tunnistautumisprotokollien yhteydessä suojatulla resurssilla tarkoitetaan resurssia, jonka käyttö vaatii tunnistautumisen ja käyttöoikeuden. Yleisessä tapauksessa suojatulla resurssilla tarkoitetaan yksittäistä resurssia (käyttäjän valokuvaa), johon halutaan asettaa pääsyrajoituksia \cite{nisti}. Tämän tutkielman puitteissa suojatulla resurssilla tarkoitetaan tunnistautumista vaativaa web-palvelua.

Valtuutustieto koostuu yksilöivästä tunnisteesta ja siihen liittyvästä salaisesta avaimesta. Tämän tutkielman puitteissa valtuutustiedolla tarkoitetaan käyttäjän tunnusta ja salasanaa.

Kirjauduttuaan sisään tunnistautumispalvelimelle, käyttäjä saa valtuutusavaimen, jonka hän lähettää eteenpäin suojatun resurssin omistajalle. Valtuutusavain ei pidä sisällään käyttäjän valtuutustietoja, vaan ainoastaan tunnistautumispalvelin osaa lukea sen \cite{nisti}. Saatuaan valtuutusavaimen käyttäjältä voi suojatun resurssin omistaja hakea pääsyvaltuuden käyttäjän tietoihin tunnistautumispalvelusta.

Pääsyvaltuutus on tunnistautumispalvelimelta saatava yksilöivä tunniste, jonka avulla suojatun resurssin omistaja voi pyytää käyttäjän tiedot tunnistautumispalvelulta. Pääsyvaltuutus on voimassa tietyn ajan, jonka jälkeen se täytyy uusia tunnistautumispalvelimella \cite{nisti}. Pääsyvaltuutusta voidaan käyttää myös tunnistautumispalvelusta erillään olevien resurssien valtuuttamiseen. Esimerkiksi web-sovellus voi hakea tunnistautumispalvelulta pääsyvaltuuden, jolla hän hakee valokuvia valokuvien jakopalvelusta \cite{facebook}.
\subsection{Ympäristön kuvaus}
Lähteet:\\
- inside the identity management game \cite{inside_the_identity_management_game}\\
- Decentralization: The Future of Online Social Networking \cite{decentralisations}

Tyypillisesti web-palvelun toimintakenttä on Internet, jossa palvelut toimivat itsenäisesti. Näiden palveluiden välinen integraatio on kasvussa ja palveluiden kesken halutaan jakaa tietoa, jolloin niiden täytyy pystyä identifioimaan käyttäjä keskenään. Yleisen identeettitarjoajan rakentaminen Internettiin on tutkimuksen alla ja OpenID ja mitä näitä nyt on. 

Usein ei ole tarpeen tehdä palveluista julkisia, vaan käyttöoikeus niihin voidaan rajata tietylle osajoukolle kaikista Internetin käyttäjistä. Tällaisia osajoukkoja voi olla esimerkiksi yrityksen työntekijät, joilla on pääsy intranet-palveluihin tai tietyn sivuston käyttäjät, joilla on pääsy sivuston palveluihin. Tällöin voi olla järkevää eriyttää käyttäjähallinta omaksi palveluksi ja keskittää osapalveluiden tunnistautuminen siihen. Tämän tutkielman pääpaino on tunnetulle osajoukolle, esimerkiksi yrityksen työntekijöille, suunnatuissa palveluissa.

Tutkielmassa pyritään selvittämään kuinka yritys voi rakentaa keskitetyn tunnistautumispalvelun valmiin käyttäjädatan päälle. Lähtökohtaisesti tunnistautumista vaativat palvelut ovat web-pohjaisia, mutta myös työasemalla tai puhelimella käytettävät asiakasohjelmat pyritään ottamaan huomioon.

\subsection{Käyttötapauksia}
Millaisia käyttötapauksia kyseisessä järjestelmässä on. Eli avataan käyttötapauskaavioilla ongelmakenttää.

\subsection{Käyttäjädata}
Tunnistautumiseen liittyvien käsitteiden läpikäynti ennen protokollien yksityiskohtaista esittelyä auttaa tunnistautumiseen liittyvien periaatteiden hahmottamista. Käsitteet ovat yleisluontoisia ja eivätkä kosketa vain tiettyjä protokollaa. Protokollien yhteydessä käytetään käsitteitä asiakasohjelma, tunnistautumispalvelu, suojattu resurssi, valtuutustieto (credentials), valtuutusavain (authorization code) ja pääsyvaltuutus (access token) \cite{nisti}.

Asiakasohjelmalla tarkoitetaan web-palvelun käyttäjän pääteohjelmaa, jolla hän kirjautuu web-palveluun käyttäen keskitettyä tunnistautumispalvelua. Käytännössä asiakasohjelma on web-palvelun tapauksessa käyttäjän WWW-selain, joka pystyy tekemään uudelleenohjauksia sivustolta toiselle. Uudelleenohjaus on HTTP-protokollan perustoiminnallisuutta, joten mikä tahansa HTTP/1.1-standardin WWW-selain käy asiakasohjelmaksi \cite{rfc2616}.

Tunnistautumispalvelu on web-palvelu, johon käyttäjä ohjataan tekemään tunnistautuminen. Onnistuneen tunnistautumisen jälkeen tunnistautumispalvelu ohjaa asi\-a\-kas\-oh\-jel\-man takaisin tunnistautumista pyytäneen palvelun määrittelemään osoitteeseen \cite{nisti}. Avoimen Internetin puolella tunnistautumispalvelu voi olla esimerkiksi Facebook tai LinkedIn.

Tunnistautumisprotokollien yhteydessä suojatulla resurssilla tarkoitetaan resurssia, jonka käyttö vaatii tunnistautumisen ja käyttöoikeuden. Yleisessä tapauksessa suojatulla resurssilla tarkoitetaan yksittäistä resurssia (käyttäjän valokuvaa), johon halutaan asettaa pääsyrajoituksia \cite{nisti}. Tämän tutkielman puitteissa suojatulla resurssilla tarkoitetaan tunnistautumista vaativaa web-palvelua.

Valtuutustieto koostuu yksilöivästä tunnisteesta ja siihen liittyvästä salaisesta avaimesta. Tämän tutkielman puitteissa valtuutustiedolla tarkoitetaan käyttäjän tunnusta ja salasanaa.

Kirjauduttuaan sisään tunnistautumispalvelimelle, käyttäjä saa valtuutusavaimen, jonka hän lähettää eteenpäin suojatun resurssin omistajalle. Valtuutusavain ei pidä sisällään käyttäjän valtuutustietoja, vaan ainoastaan tunnistautumispalvelin osaa lukea sen \cite{nisti}. Saatuaan valtuutusavaimen käyttäjältä voi suojatun resurssin omistaja hakea pääsyvaltuuden käyttäjän tietoihin tunnistautumispalvelusta.

Pääsyvaltuutus on tunnistautumispalvelimelta saatava yksilöivä tunniste, jonka avulla suojatun resurssin omistaja voi pyytää käyttäjän tiedot tunnistautumispalvelulta. Pääsyvaltuutus on voimassa tietyn ajan, jonka jälkeen se täytyy uusia tunnistautumispalvelimella \cite{nisti}. Pääsyvaltuutusta voidaan käyttää myös tunnistautumispalvelusta erillään olevien resurssien valtuuttamiseen. Esimerkiksi web-sovellus voi hakea tunnistautumispalvelulta pääsyvaltuuden, jolla hän hakee valokuvia valokuvien jakopalvelusta \cite{facebook}.
\subsubsection{passwd}
Vanha kunnon /etc/passwd, tästä tunnistautuminen on varmaan lähtenyt käyntiin. Ikävää, kun webiin tunnistautuessa täytyy olla tunnus kyseisellä koneella ja muutenkin ei ole hyvä kun tunnukset siirtyy verkkoa pitkin.
\subsubsection{Relaatiotietokannat}
Käyttäjätietokannat, relaatiokannat lähinnä, ehkä NoSQL.
\subsubsection{LDAP}
Lähteet: Howes, T. A., The Lightweight Directory Access Protocol: X.500 Lite. CITI
Technical Report 95–8, University of Michigan, 1995. \cite{howes} \\
rfc:t 4510-4513 (ainakin 4513 "Authentication Methods and Security Mechanisms" kiinnostaa)

Lightweight Directory Access Protocol (LDAP) on X.500 OSI-standardiin perustuva hakemistopalvelu, jota käytetään yleisesti käyttäjätiedon tallennukseen [TODO: lähde]. 1990-luvulla TCP/IP-mallin syrjäytettyä OSI-mallin, myös DAP kävi vanhanaikaiseksi \cite{howes}. Korvaajaksi on noussut LDAP, josta käytetään myös nimeä X.500 Lite \cite{howes}.

LDAP:ssa asiakassovellukset (directory user agent, DUA) keskustelevat puumalliin perustuvan hakemistopalvelimen (directory system agent, DSA) kanssa käyttäen määriteltyä protokollaa (directory access protocol, DAP) \cite{howes}. Asiakassovellukset voivat hakea hakemistopalvelimesta tietoa suodattimiin (filter) perustuvalla lukuoperaatiolla. Suodattimessa voidaan määritellä raja-arvot attribuutin arvolle tai hakea avainsanoilla attribuuteista.

LDAP-tietuille voidaan määritellä pakollisten attribuuttien (esim. etu- ja sukunimi) lisäksi valinnaisia attribuutteja. Tietueet on järjestetty puuhun niiden yksilöivän nimen (distinguished name, DN) mukaan ja ne voi olla hajautettu usealle palvelimelle. Suhteellinen nimi (relative distinguished name, RDN) identifioi tietueen omalla hierarkiatasollaan.

LDAP-tietueella voi olla tunnus sekä salasana ja LDAP-palvelinta voidaan käyttää käyttäjän tunnistautumiseen [TODO: lähde, ehkä rfc4513].

TODO: lisää tekstiä

\subsubsection{Käyttäjädatan abstraktointi}
Tutkimuksen kannalta abstraktointi on oleellista, oikeastaan sillä ei ole ison kuvan kannalta merkitystä, että onko siellä taustalla tietokanta, tiedosto, ldap vai mikä.
\subsection{Yhteenveto}
Käytetään OAuthia protossa, koska OpenID:ssä kaikkea tarpeetonta mukana. SAML taas skipataan, koska...? Tätä pitäisi pohtia jossain kohtaa.
