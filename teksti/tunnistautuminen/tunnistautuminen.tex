Miten tunnistetaan perinteisesti?

Käyttäjädatan tallennus
- passwd (ts tiedosto levyllä tms)
- tietokanta
- AD-kannat (LDAP)

---> käyttäjän datan abstraktoinnin tarve, ehkä mainita, että käyttäjädatan backendejä voi käytännössä olla monta erilaista. Esim. LDAP:n rinnalla tietokannat.

Pituus n. 5 sivua.

Lähteitä:\\
A Guide to Computer Network Security \cite{authentication}\\
Authentication in distributed systems: theory and practice \cite{lampson}

LDAP:

Howes, T. A., The Lightweight Directory Access Protocol: X.500 Lite. CITI
Technical Report 95–8, University of Michigan, 1995. \cite{howes} \\
rfc:t 4510-4513 (ainakin 4513 "Authentication Methods and Security Mechanisms" kiinnostaa)

-----------------------------------------------------------------------------------------------------------

Miten käyttäjädataa onko käsitelty ja käsitellään. Kehitys paikallisesti käytetyistä tiedostopohjaisista systeemeistä kohti tietokantoja ja asiaan räätälöihin palveluihin (LDAP). LDAP oleelisin, mutta tutkimuksen kannalta abstraktointi on tärkeä juttu.

Johdanto puoli sivua, alaluvut 0.5-1 sivu.
\subsection{Käsitteistö}
Palveluiden turvallisuus koostuu kolmesta osatekijästä: tunnistautumisesta (authentication), pääsynvalvonnasta (access control) ja auditoinnista (audit) \cite{sandhu}. Tunnistautumisessa käyttäjän identiteetti varmistetaan esimerkiksi käyttäjätunnuksen ja salasanan avulla. Tämän jälkeen pääsynvalvonta tarkistaa, onko kyseisellä käyttäjällä oikeutta suorittaa pyydetty toiminto. Auditoinnissa analysoidaan aktiivisesti järjestelmän tuottamaa dataa, kuten lokitiedostoja, aktiivisesti. Käyttäjän pääsy järjestelmään voidaan estää, jos luvatonta käyttöä esiintyy. Tämän tutkielman pai\-no\-pis\-te on käyttäjän tunnistautumisessa ja osittain myös pääsynvalvonnassa, mutta auditointi ei kuulu tutkielman aihepiiriin.

Tunnistautumista ja pääsynvalvontaa ei voida täysin erottaa toisistaan, koska käyttäjään liittyy attribuutteja, joiden perusteella pääsynvalvonta voidaan tehdä järjestelmän sisällä. Pääsynvalvonta voidaan tehdä esimerkiksi roolipohjaisena, jolloin käyttäjän rooli organisaatiossa (esimerkiksi esimies/alainen) vaikuttaa siihen, mitä palveluita tai palveluiden osia hänellä on oikeus käyttää \cite{sandhu_rbac}.
\subsection{Ympäristön kuvaus}
Yrityksen tai yhteisön web-sovellukset ovat avoimia pienemmälle osajoukolle käyttäjiä, esimerkiksi yrityksen intranet-järjestelmään on pääsy vain yrityksen työntekijöillä, jotka on kirjattu tietokantaan. Erilaisin palomuuri-asetuksin pääsy intranet-järjestelmään voidaan rajata vain yrityksen sisäverkkoon, mutta sekään ei poista tunnistautumisen tarvetta. Aivan kuin avoimissa sovelluksissa, myös intranet-järjestelmässä halutaan tietää kuka yrityksen työntekijöistä sitä kulloinkin käyttää, jotta työntekijälle osataan näyttää vain häntä koskevaa dataa.

Joissakin tapauksissa intranet-järjestelmä pitää sisällään myös käyttäjähallinnan, jolloin yrityksessä ei ole erillistä tietokantaa käyttäjille, vaan jokaiselle työntekijälle luodaan erillinen tunnus intranetiin. Toisissa tapauksissa taas halutaan hyödyntää erillistä käyttäjähallintaa, jolloin käyttäjän tiedot ovat  esimerkiksi erillisellä LDAP-palvelimella, jota vasten intranet-järjestelmä tunnistaa käyttäjät.

Palvelusuuntautuneissa arkkitehtuureissa samaa käyttäjätietokantaa käyttäviä sovelluksia, tai palveluita, voi olla useita. Tällöin ns. pääkäyttäjäkannasta voidaan luoda oma paikallinen kopio jokaista palvelua varten. Tästä seuraa monenlaisia synkronointiongelmia [TODO: lähde]. Esimerkiksi työntekijän irtisanoutuessa joudutaan tunnus poistamaan kaikista tietokannoista erikseen. Myös osoitteen yms. tietojen muutokset täytyy päivittää kaikkiin tietokantoihin. Lisäksi käyttäjälle syntyy saman järjestelmän sisällä monia tunnuksia, joihin saattaa liittyä erilliset salasanat. Käyttäjän kannalta on myöskin ikävää kirjautua jokaiseen osapalveluun erikseen.

Koko käyttäjäkantaa ei ole kuitenkaan tarve kopioida jokaiselle sovellukselle, vaan yksittäiset sovellukset voivat tunnistautua erillistä tunnistautumispalvelua vasten [TODO: lähde]. Tällöin käyttääkseen yrityksen intranet-palvelua, täytyy käydä tunnistautumassa tunnistautumispalvelussa. Nykypäivänä monet web-sovellukset toimivat juuri ulkoisen tunnistaumispalvelun kautta. Viihdesivustoille voi luoda tunnuksen kirjautumalla Facebookin tai LinkedInin kaltaisten sivustojen kautta [TODO: lähde?]. Myös esimerkiksi Kelan sivuston käyttöä varten tunnistaudutaan pankkitunnuksilla Tupas-järjestelmän avulla [TODO: lähde].

Palvelusuuntauneen arkkitehtuurin kannalta tunnistautumisen keskittäminen yhdelle palvelulle on kiinnostava idea. Tällöin yksittäinen palvelu käyttää tunnistautumiseen erillistä palvelua, joka integroituu yrityksen tai yhteisön käyttäjäkantaan. Keskitettyä tunnistautumista käsitellään seuraavassa luvussa.
\subsection{Käyttäjädata}
Miten käyttäjädataa onko käsitelty ja käsitellään. Kehitys paikallisesti käytetyistä tiedostopohjaisista systeemeistä kohti tietokantoja ja asiaan räätälöihin palveluihin (LDAP). LDAP oleelisin, mutta tutkimuksen kannalta abstraktointi on tärkeä juttu.

Johdanto puoli sivua, alaluvut 0.5-1 sivu.
\subsubsection{passwd}
Vanha kunnon /etc/passwd, tästä tunnistautuminen on varmaan lähtenyt käyntiin. Ikävää, kun webiin tunnistautuessa täytyy olla tunnus kyseisellä koneella ja muutenkin ei ole hyvä kun tunnukset siirtyy verkkoa pitkin.
\subsubsection{Relaatiotietokannat}
Käyttäjätietokannat, relaatiokannat lähinnä, ehkä NoSQL.
\subsubsection{LDAP}
Lightweight Directory Access Protocol (LDAP) on X.500 OSI-standardiin perustuva hakemistopalvelu, jota käytetään yleisesti käyttäjätiedon tallennukseen [TODO: lähde]. 1990-luvulla TCP/IP-mallin syrjäytettyä OSI-mallin, myös DAP kävi vanhanaikaiseksi \cite{howes}. Korvaajaksi on noussut LDAP, josta käytetään myös nimeä X.500 Lite \cite{howes}.

LDAP:ssa asiakassovellukset (directory user agent, DUA) keskustelevat puumalliin perustuvan hakemistopalvelimen (directory system agent, DSA) kanssa käyttäen määriteltyä protokollaa (directory access protocol, DAP) \cite{howes}. Asiakassovellukset voivat hakea hakemistopalvelimesta tietoa suodattimiin (filter) perustuvalla lukuoperaatiolla. Suodattimessa voidaan määritellä raja-arvot attribuutin arvolle tai hakea avainsanoilla attribuuteista.

LDAP-tietuille voidaan määritellä pakollisten attribuuttien (esim. etu- ja sukunimi) lisäksi valinnaisia attribuutteja. Tietueet on järjestetty puuhun niiden yksilöivän nimen (distinguished name, DN) mukaan ja ne voi olla hajautettu usealle palvelimelle. Suhteellinen nimi (relative distinguished name, RDN) identifioi tietueen omalla hierarkiatasollaan.

LDAP-tietueella voi olla tunnus sekä salasana ja LDAP-palvelinta voidaan käyttää käyttäjän tunnistautumiseen [TODO: lähde, ehkä rfc4513].

TODO: lisää tekstiä

\subsubsection{Käyttäjädatan abstraktointi}
Tutkimuksen kannalta abstraktointi on oleellista, oikeastaan sillä ei ole ison kuvan kannalta merkitystä, että onko siellä taustalla tietokanta, tiedosto, ldap vai mikä.
\subsection{Yhteenveto}
Keskitetyn tunnistautumispalvelun käyttö on perusteltua, jos ympäristössä on useita käyttäjän tunnistamista vaativia web-sovelluksia. Tunnistautumispalvelun käytöllä ehkäistään käyttäjätietojen kopiointiin liittyviä synkronointiongelmia, kun käyttäjädata on keskitetty yhteen paikkaan. Erityisesti arkaluontoinen data, kuten salasanatiivisteet, kannattaa keskittää, jolloin ne eivät päädy vääriin käsiin yksittäisiin web-sovelluksiin kohdistuneiden tietomurtojen yhteydessä.

Tunnistautumispalvelu mahdollistaa myös muiden kuin järjestelmän ylläpitäjien tuottamien sovellusten käytön organisaation sisäisillä käyttäjätunnuksilla. Käyttämällä luotettavaksi todettuja rajapintoja, voi ylläpito antaa kolmannen osapuolen toteuttamalle web-sovellukselle oikeuden käyttää tunnistautumispalvelua käyttäjän tunnistamiseen. Tällöin käyttäjä ohjataan tunnistautumispalveluun tunnistamisen ajaksi ja web-sovellus saa vain pääsyvaltuuden, jolla sovellus voi hakea käyttäjän tiedot. Käyttäjän tunnistetiedot eivät tule missään vaiheessa tunnistamista tarvitsevan web-sovelluksen tietoon. Näin ollen esimerkiksi Facebook-tunnuksilla voi kirjautua useaan web-sovellukseen, vaikka Facebookilla ei ole tarkkaa tietoa sovelluksien sisäisestä toimintalogiikasta.

Tutkielmassa esitellyn Kapsi ry:n hallintatyökalujen tapauksessa palvelu tullaan toteuttamaan Django-sovelluskehyksellä, mutta myös valmiita toteutuksia on olemassa eri intranet-ympäristöihin. Teknologivalinta tulee tehdä sovellusympäristön mukaan, eikä yksi ratkaisu sovi kaikkiin ympäristöihin. Web-sovelluksen ja tunnistautumispalvelun välinen rajapinta pitää huolen palveluiden yhteensopivuudesta. Rajapinnoiksi on valittavissa useita eri protokollia, kuten Web Services -standardin SAML tai avoimen lähdekoodin yhteisössä syntyneet OpenID ja OAuth. Protokollien käyttö rinnakkain on myös mahdollista, jolloin tunnistautuminen voidaan tehdä esimerkiksi SAML- tai OAuth-protokollalla riippuen web-sovelluksesta.

Tunnistautumiseen liittyvien tehtävien erottaminen yksittäisiltä web-sovelluksilta erillisen palvelun tehtäväksi on palvelusuuntauneiden arkkitehtuurien periaatteiden mukaista. Tällaisten arkkitehtuurien mukaan toteutetuissa sovellusympäristöissä jokaisella web-sovelluksella on oma tarkasti määritelty tehtävä. Kun arkkitehtuurissa on oma palvelu tunnistautumiselle, on pienten yksittäisten komponenttien toteutus helpompaa, koska jokaisen komponentin kohdalla ei tarvitse huolehtia tunnistautumisen toteutuksesta.

Web-sovelluksen ja tunnistamisen erottaminen toisistaan mahdollistaa myös tunnistautumisen tehostamisen ilman muutoksia web-sovellusten toimintaan. Järjestelmässä voidaan ottaa salasanan lisäksi käyttöön toiseen tekijään perustuva tunnistaminen, jolloin esimerkiksi käyttäjän täytyy salasanan lisäksi syöttää matkapuhelimeen lähetetty tunnistekoodi. Web-sovelluksen ja tunnistautumispalvelun välinen rajapinta ei tässä tapauksessa muutu, joten tunnistamisen parantaminen ei edellytä web-sovelluksien muuttamista.

Keskitetyn pääsynvalvonnan toteuttaminen tutkielmassa kuvatuilla periaatteilla on mahdollista. Tällöin tiedot käyttäjien pääsyoikeuksista ympäristön sisällä ovat yhdessä paikassa, joka helpottaa niiden hallinnointia. Esimerkiksi henkilön siirtyessä tehtävästä toiseen, voidaan hänen pääsyvaltuudet muuttaa samasta paikasta. Pääsynvalvontaan voidaan käyttää esimerkiksi tutkielmassa esiteltyjä SAML- tai OAuth-protokollia.

Käyttäjän tunnistamista vaativissa web-sovelluksissa käyttäjän tunnistaminen kannattaa toteuttaa erillisenä komponenttina. Käytettyjen teknologioiden suhteen valinta täytyy tehdä web-sovellusympäristön mukaan, koska kaikkiin ympäristöihin sopivaa ratkaisua ei ole. Tässä tutkielmassa esitellyt periaatteet ovat kuitenkin sovellettavissa eri teknologioita käytettäessä.
