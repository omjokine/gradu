Palveluiden turvallisuus koostuu kolmesta tekijästä: tunnistautumisesta (authentication), pääsynvalvonnasta (access control) ja auditoinnista (audit) \cite{sandhu}. Tunnistautumisessa käyttäjän identiteetti varmistetaan, esimerkiksi käyttäjätunnuksen ja salasanan avulla. Tämän jälkeen pääsynvalvonta tarkistaa onko kyseisellä käyttäjällä oikeutta tehdä pyytämäänsä toimintoa. Auditoinnissa analysoidaan järjestelmän tuottamaa dataa, esimerkiksi lokitiedostoja, aktiivisesti ja käyttäjän pääsy järjestelmään voidaan estää, jos luvatonta käyttöä esiintyy \cite{sandhu}.

Tämän tutkimuksen painopiste on käyttäjän tunnistautumisessa ja osittain myös pääsynvalvonnassa, auditointi ei kuulu tutkimuksen aihepiiriin. Tutkimuksessa käytetään termiä pääsynvalvonta sen suppeassa merkityksessä, toisinaan kirjallisuudessa pääsynvalvontaa pidetään kattoterminä, joka pitää sisällään tunnistautumisen, valtuutuksen (authorization) ja auditoinnin [TODO: joku lähde, jossa näin on].
