Mitä rajoituksia toteutuksella on? Kuinka voisi laajentaa? Kertakirjautuminen ja autorisointi hyviä suuntia. SSO aika peruskamaa OAuthin kanssa, autorisoinnista sen sijaan voi saada ihan mielenkiintoista pohdintaa aikaan.

LDAP:n hyödyntäminen autorisoinnissa, voisiko esimerkiksi käyttäjäryhmät olla tallennettu jotenkin kätevästi LDAP:in ja niiden perusteella voidaan autorisoida pääsy resurssiin.

Sinällään pääsynhallinta tiettyihin palveluihin on jonkin tason autorisointia ja kuuluu ehdottomasti gradun aihepiiriin (toteutus tunnistaa x:n palvelun käyttäjiä samaa käyttäjädataa vasten, kaikille käyttäjille ei saa antaa oikeutta kaikkiin palveluihin).

Pituus yhteensä pari-kolme sivua. Ei sinällään haittaa, vaikka menisi tuostakin yli, tämä on kuitenkin mielenkiintoista ja oleellista asiaa.

Tunnistautumiseen liittyvien käsitteiden läpikäynti ennen protokollien yksityiskohtaista esittelyä auttaa tunnistautumiseen liittyvien periaatteiden hahmottamista. Käsitteet ovat yleisluontoisia ja eivätkä kosketa vain tiettyjä protokollaa. Protokollien yhteydessä käytetään käsitteitä asiakasohjelma, tunnistautumispalvelu, suojattu resurssi, valtuutustieto (credentials), valtuutusavain (authorization code) ja pääsyvaltuutus (access token) \cite{nisti}.

Asiakasohjelmalla tarkoitetaan web-palvelun käyttäjän pääteohjelmaa, jolla hän kirjautuu web-palveluun käyttäen keskitettyä tunnistautumispalvelua. Käytännössä asiakasohjelma on web-palvelun tapauksessa käyttäjän WWW-selain, joka pystyy tekemään uudelleenohjauksia sivustolta toiselle. Uudelleenohjaus on HTTP-protokollan perustoiminnallisuutta, joten mikä tahansa HTTP/1.1-standardin WWW-selain käy asiakasohjelmaksi \cite{rfc2616}.

Tunnistautumispalvelu on web-palvelu, johon käyttäjä ohjataan tekemään tunnistautuminen. Onnistuneen tunnistautumisen jälkeen tunnistautumispalvelu ohjaa asi\-a\-kas\-oh\-jel\-man takaisin tunnistautumista pyytäneen palvelun määrittelemään osoitteeseen \cite{nisti}. Avoimen Internetin puolella tunnistautumispalvelu voi olla esimerkiksi Facebook tai LinkedIn.

Tunnistautumisprotokollien yhteydessä suojatulla resurssilla tarkoitetaan resurssia, jonka käyttö vaatii tunnistautumisen ja käyttöoikeuden. Yleisessä tapauksessa suojatulla resurssilla tarkoitetaan yksittäistä resurssia (käyttäjän valokuvaa), johon halutaan asettaa pääsyrajoituksia \cite{nisti}. Tämän tutkielman puitteissa suojatulla resurssilla tarkoitetaan tunnistautumista vaativaa web-palvelua.

Valtuutustieto koostuu yksilöivästä tunnisteesta ja siihen liittyvästä salaisesta avaimesta. Tämän tutkielman puitteissa valtuutustiedolla tarkoitetaan käyttäjän tunnusta ja salasanaa.

Kirjauduttuaan sisään tunnistautumispalvelimelle, käyttäjä saa valtuutusavaimen, jonka hän lähettää eteenpäin suojatun resurssin omistajalle. Valtuutusavain ei pidä sisällään käyttäjän valtuutustietoja, vaan ainoastaan tunnistautumispalvelin osaa lukea sen \cite{nisti}. Saatuaan valtuutusavaimen käyttäjältä voi suojatun resurssin omistaja hakea pääsyvaltuuden käyttäjän tietoihin tunnistautumispalvelusta.

Pääsyvaltuutus on tunnistautumispalvelimelta saatava yksilöivä tunniste, jonka avulla suojatun resurssin omistaja voi pyytää käyttäjän tiedot tunnistautumispalvelulta. Pääsyvaltuutus on voimassa tietyn ajan, jonka jälkeen se täytyy uusia tunnistautumispalvelimella \cite{nisti}. Pääsyvaltuutusta voidaan käyttää myös tunnistautumispalvelusta erillään olevien resurssien valtuuttamiseen. Esimerkiksi web-sovellus voi hakea tunnistautumispalvelulta pääsyvaltuuden, jolla hän hakee valokuvia valokuvien jakopalvelusta \cite{facebook}.
\subsection{Kertakirjautuminen}
Kertakirjautumisella (SSO) tarkoitetaan arkkitehtuuria, jossa yhdellä kirjautumisella tunnistaudutaan useaan eri palveluun \cite{sso}. Käytetyissä kertakirjautumistekniikoissa käytetään identiteetintarjoajaa (identity provider), joka varmentaa kirjautumisen ja luo valtuutuksen web-palveluihin. Käytetyt tekniikat jaetaan kahteen osaan: näennäiseen- (pseudo SSO) ja tosikertakirjautumiseen (true SSO) \cite{sso}. Näennäisessä kertakirjautumisessa identiteetintarjoaja luo palveluun sopivan valtuutuksen, jolloin web-palvelu ei tiedä kertakirjautumisesta mitään. Tosikertakirjautumisessa taas käytetään yleisiä järjestelmänlaajuisia valtuutuksia, jolloin web-palveluiden täytyy olla tietoisia kertakirjautumisesta.

Kapsin nykyisessä arkkitehtuurissa käyttäjä voi kirjautua samalla tunnuksella eri palveluihin, mutta kyseessä ei ole kertakirjautuminen, koska kirjautumisen täytyy tehdä erikseen jokaiseen palveluun. Kertakirjautuminen on mahdollista tehdä myös nykyisessä arkkitehtuurissa, esimerkiksi tallentamalla tunnistetiedot selaimen evästeeseen (cookie). Toinen web-palvelu voi taas lukea tunnistetiedot evästeestä ja tunnistaa käyttäjän tätä kautta. Koska eri web-palvelut toimivat samassa domain-osoitteessa, pystyvät ne lukemaan toistensa kirjoittamia evästeitä \cite{rfc6265}. Tällainen ratkaisu on kuitenkin ongelmallinen, koska tunnistetiedot tallennetaan käyttäjän koneelle ja ne on luettavissa evästeestä. Jos taas tunnistetietojen kirjoittamiseen ja lukemiseen käytetään salausta, joudutaan salausavain jakamaan web-palveluiden kesken, joka lisää ylläpidon tarvetta.

Keskitettyä tunnistautumispalvelua käytettäessä kertakirjautuminen voidaan toteuttaa OAuth-protokollalla \cite{distributed_web_security}. Kontrollin kulku on samanlainen kuin kuvassa \ref{auth_kapsi_fi_flow}, mutta käyttäjän ei tarvitse syöttää tunnusta ja salasanaa, koska hänellä on jo voimassaoleva istunto tunnistautumispalveluun. Toisin sanoen käyttäjäagentti (selain) saa pääsyvaltuuden tunnistautumispalvelusta ilman uutta kirjaumista, jolloin kyseessä on näennäinen kertakirjautuminen. Kuitenkin ensimmäistä kertaa kirjautuessa web-sovellukseen käyttäjän täytyy hyväksyä tietojen haku tunnistautumispalvelusta. Kun tietojen luovutus on hyväksytty, voidaan jatkossa kirjautuminen tehdä ilman käyttäjän syötettä.
\subsection{Pääsynvalvonta}
Pääsynvalvonta ei ole kovin triviaali juttu. Kuitenkin jotain vakavamielistä pohdintaa, miten ratkaisua voisi laajentaa siihen suuntaan.

Käytännössä jonkunlainen pääsynvalvonta tulee, sillä auth.kapsi.fi tarkistaa onko käyttäjällä oikeus käyttää Sikteeriä tms ja palauttaa käyttäjän tiedot vain jos näin on.

