Web-sovellusten arkkitehtuurin kehittyminen palveluperustaiseksi on lisännyt tunnistusta tarvitsevien web-sovellusten määrää. Useiden samoilla käyttäjätunnuksilla käytettävien web-sovellusten ongelma on käyttäjien pääsyvaltuuksien kopioiminen moneen eri tietokantaan. Pääsyvaltuuksien kopioiminen tuottaa synkronointiongelmia, kun samasta datasta on useita eri kopioita. Ratkaisuksi tutkielmassa esitettiin keskitettyä tunnistautumispalvelua.

Keskitetyn tunnistautumispalvelun käyttö on perusteltua, jos ympäristössä on useita käyttäjän tunnistamista vaativia web-sovelluksia. Tunnistautumispalvelun käytöllä ehkäistään käyttäjätietojen kopiointiin liittyviä synkronointiongelmia, kun käyttäjädata on keskitetty yhteen paikkaan. Erityisesti arkaluontoinen data, kuten salasanatiivisteet, kannattaa keskittää, jolloin ne eivät päädy vääriin käsiin yksittäisiin web-sovelluksiin kohdistuneiden tietomurtojen yhteydessä.

Tunnistautumispalvelu mahdollistaa myös muiden kuin järjestelmän ylläpitäjien tuottamien sovellusten käytön organisaation sisäisillä käyttäjätunnuksilla. Käyttämällä luotettavaksi todettuja rajapintoja voi ylläpito antaa kolmannen osapuolen toteuttamalle web-sovellukselle oikeuden käyttää tunnistautumispalvelua käyttäjän tunnistamiseen. Tällöin käyttäjä ohjataan tunnistautumispalveluun tunnistamisen ajaksi ja web-sovellus saa vain pääsyvaltuuden, jolla sovellus voi hakea käyttäjän tiedot. Käyttäjän tunnistetiedot eivät tule missään vaiheessa tunnistamista tarvitsevan web-sovelluksen tietoon. Näin ollen esimerkiksi Facebook-tunnuksilla voi kirjautua useaan web-sovellukseen, vaikka Facebookilla ei ole tarkkaa tietoa sovelluksien sisäisestä toimintalogiikasta.

Tunnistautumiseen liittyvien tehtävien erottaminen yksittäisiltä web-sovelluksilta erillisen palvelun tehtäväksi on palveluperustaisten arkkitehtuurien periaatteiden mukaista. Tällaisten arkkitehtuurien mukaan toteutetuissa sovellusympäristöissä jokaisella web-sovelluksella on oma tarkasti määritelty tehtävä. Kun arkkitehtuurissa on oma palvelu tunnistautumiselle, on pienten yksittäisten komponenttien toteutus helpompaa, koska jokaisen komponentin kohdalla ei tarvitse huolehtia tunnistautumisen toteutuksesta.

Tutkielmassa esitellyn Kapsi ry:n hallintatyökalujen tapauksessa tunnistautumispalvelu tullaan toteuttamaan Django-sovelluskehyksellä. Myös erilaisiin intranet-ympäristöihin on olemassa valmiita tunnistautumispalvelun toteutuksia. Teknologivalinta tulee tehdä sovellusympäristön mukaan, eikä yksi ratkaisu sovi kaikkiin ympäristöihin. Web-sovelluksen ja tunnistautumispalvelun välinen rajapinta pitää huolen palveluiden yhteensopivuudesta. Rajapinnoiksi on valittavissa useita eri protokollia, kuten Web Services -standardin SAML tai avoimen lähdekoodin yhteisössä syntyneet OpenID ja OAuth. Mikäli kaikki intranet-sovellukset eivät tue samaa protokollaa, on protokollien käyttö rinnakkain mahdollista. Tällöin tunnistautuminen voidaan tehdä esimerkiksi SAML- tai OAuth-protokollalla riippuen web-sovelluksesta.

Web-sovelluksen ja tunnistamisen erottaminen toisistaan mahdollistaa myös tunnistautumisen tehostamisen ilman muutoksia web-sovellusten toimintaan. Järjestelmässä voidaan ottaa salasanan lisäksi käyttöön toiseen tekijään perustuva tunnistaminen, jolloin esimerkiksi käyttäjän täytyy salasanan lisäksi syöttää matkapuhelimeen lähetetty tunnistekoodi. Web-sovelluksen ja tunnistautumispalvelun välinen rajapinta ei tässä tapauksessa muutu, joten tunnistamisen parantaminen ei edellytä web-sovelluksien muuttamista.

Keskitetyn pääsynvalvonnan toteuttaminen tutkielmassa kuvatuilla periaatteilla on mahdollista. Tällöin tiedot käyttäjien pääsyoikeuksista ympäristön sisällä ovat yhdessä paikassa ja niiden hallinnointi helpottuu. Esimerkiksi henkilön siirtyessä tehtävästä toiseen voidaan hänen pääsyvaltuutensa muuttaa samasta paikasta. Pääsynvalvontaan voidaan käyttää esimerkiksi tutkielmassa esiteltyjä SAML- tai OAuth-protokollia.

Käyttäjän tunnistamista vaativissa web-sovelluksissa käyttäjän tunnistaminen kannattaa toteuttaa erillisenä komponenttina. Käytettyjen teknologioiden suhteen valinta täytyy tehdä web-sovellusympäristön mukaan, koska kaikkiin ympäristöihin sopivaa ratkaisua ei ole. Tässä tutkielmassa esitellyt periaatteet ovat kuitenkin sovellettavissa eri teknologioita käytettäessä.