Tutkielmassa selvitetään keskitetyn tunnistautumispalvelun toteuttamista palvelusuuntautuneiden arkkitehtuurien mukaisesti toteutetuissa web-sovellusympäristöissä. Pääpainopisteenä ovat sovellusympäristöt, joissa ollaan siirtymässä palvelusuuntautuneisiin arkkitehtuureihin. Esimerkkinä tällaisesta ympäristöstä käytetään Kapsi Internet-käyttäjät ry:n jäsenhallintatyökaluja, joiden tunnistautuminen muutetaan keskitetyn tunnistautumispalvelun tehtäväksi. Myös palvelusuuntautuneiden arkkitehtuurien perusteiden soveltamista Kapsin ympäristöön tutkitaan.

Web-sovellusten kehitys ensimmäisistä CGI-ohjelmista kohti palvelusuuntautuneita arkkitehtuureita ja tunnistautumisen tarpeet tällaisissa arkkitehtuureissa esitellään. Tutkielmassa käydään läpi turvallisuuden osatekijöitä web-sovelluksissa ja hahmotellaan tarve keskitetylle tunnistautumispalvelulle. Tavallisia tunnistautumisongelmia, ja tunnistuspalvelun tuomia ratkaisuja niihin, käydään läpi tutkielmassa.

Tunnistautumispalvelun ja web-sovelluksen väliseen rajapintaan käytettäviä protokollia (SAML, OpenID ja OAuth) tutkitaan ja esitellään OAuth-protokollaa käyttävä arkkitehtuuri Kapsin järjestelmään. Arkkitehtuurin etuja ja haittoja sekä sen soveltamista muihin vastaaviin ympäristöihin tutkitaan. Arkkitehtuurin laajentamista keskitettyyn pääsynvalvontaan ja kertakirjautumiseen pohditaan.