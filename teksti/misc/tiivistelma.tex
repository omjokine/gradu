Tutkielmassa selvitetään keskitetyn tunnistautumispalvelun toteuttamista palvelusuuntautuneiden arkkitehtuurien mukaisissa web-sovellusympäristöissä. Pääpainopisteenä on sovellusympäristöt, joissa ollaan siirtymässä palvelusuuntautuneisiin arkkitehtuureihin. Tutkielmassa käytetään esimerkkinä tällaisesta ympäristöstä Kapsi Internet-käyttäjät ry:n jäsenhallintatyökaluja, joiden tunnistautuminen muutetaan keskitetyn tunnistautumispalvelun tehtäväksi.

Web-sovellusten kehitys ensimmäisistä CGI-ohjelmista kohti palvelusuuntautuneita arkkitehtuureita ja tunnistautumisen tarpeet tällaisissa arkkitehtuureissa esitellään. Tutkielmassa käydään läpi turvallisuuden osatekijöitä ja tavallisia tunnistautumiseen liittyviä ongelmia web-sovelluksissa. Hahmotellaan keskitetty tunnistautumispalvelu ratkaisemaan ongelmia sekä pohditaan sen etuja ja haittoja nykyään käytettyihin tunnistautumismekanismeihin nähden.

Tunnistautumispalvelun ja web-sovelluksen väliseen rajapintaan käytettäviä protokollia (SAML, OpenID ja OAuth) tutkitaan ja esitellään OAuth-protokollaa käyttävä arkkitehtuuri Kapsin järjestelmään. Esitetyn arkkitehtuurin etuja ja haittoja sekä sen soveltamista muihin vastaaviin ympäristöihin tutkitaan. Arkkitehtuurin laajentamista keskitettyyn pääsynvalvontaan ja kertakirjautumiseen pohditaan. Ulkoisen tunnistautumispalvelun käyttöä pohditaan, esimerkkinä Facebookin tarjoama tunnistautumisrajapinta.