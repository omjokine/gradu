Web-sovellusten yleistyessä törmätään yhä useammin käyttäjän tunnistamisen ongelmaan. Kun tietyillä käyttäjillä on pääsy järjestelmään, pitää varmistua, että käyttäjä on se, joka väittää olevansa. Web-sovellusten määrän kasvaessa käyttäjän tunnistamiseen käytettäviä valtuutustietoja (credentials) syntyy useita, jolloin käyttäjän täytyy muistaa useita käyttäjätunnus/salasana -pareja \cite{billion_keys}.

Samaan ongelmaan törmätään usein myös intranet-palveluissa, joiden arkkitehtuuria viedään palveluperustaiseen suuntaan. Tällöin jokaisen palvelun täytyy pystyä tunnistamaan käyttäjänsä, jotta heille voidaan näyttää käyttöoikeuksien mukaista sisältöä. Intranet-palveluilla voi olla oma käyttäjätietokanta, joka synkronoidaan päätietokannan kanssa. Tällöin yksittäisellä palvelulla on ajantasainen tieto käyttäjien oikeuksista käyttää palvelua. Tässä tapauksessa ongelma on, että käyttäjätunnukset ja salasanat siirtyvät useaan järjestelmään ja tämä heikentää järjestelmän tietoturvaa \cite{nisti}.

Web-palveluille voidaan avata myös pääsy organisaation käyttäjähallintaan, jolloin palveluun pääsy edellyttää, että käyttäjän syöttämä tunnus/salasana-pari löytyy käyttäjähallinnasta. Tämä on järjestelmän ylläpitäjän näkökulmasta ongelmallista, sillä käyttäjätietokanta sisältää tietoja, joita ei haluta antaa ulkopuolisille. Tutkielmassa esitellään ratkaisuksi keskitetty tunnistautumispalvelu, jolla on pääsy käyttäjähallintaan ja jonka avulla yksittäiset web-sovellukset tunnistavat käyttäjänsä \cite{nisti}.

Tutkielmassa paneudutaan käyttäjän tunnistamisen ongelmakenttään ja tutkitaan, ratkaiseeko keskitetty tunnistautumispalvelu esitetyt ongelmat ja millaisia mahdollisia uusia ongelmia sen käyttö tuo tullessaan. Tutkielmassa pohditaan myös, millaisia etuja erillisestä tunnistautumispalvelusta on verrattuna nykyisiin tunnistautumismenetelmiin nähden. Keskitetyn tunnistautumispalvelun etuja ja haittoja arvioidaan järjestelmän ylläpitäjän, käyttäjän ja web-sovellusohjelmoijan näkökulmasta.

Tutkielmassa ei keskitytä kaikille avoimiin tunnistautumispalveluihin (kuten Facebook tai Google), vaan organisaatioihin, joilla on oma käyttäjähallinta. Esimerkkinä tällaisesta organisaatiosta käytetään Internet-palveluita tuottavaa Kapsi ry:tä, joka ylläpitää erilaisia web-sovelluksia, joihin vain Kapsin jäsenillä on pääsy. Kirjallisuuskatsauksen pohjalta luodaan esimerkkiarkkitehtuuri, jolla keskitetty tunnistautumispalvelu saadaan osaksi Kapsin web-palveluita. Tutkielmassa pohditaan hyötyjä ja haittoja, joita keskitetyssä tunnistautumisratkaisussa esiintyy. Kapsin tapauksen perusteella arvioidaan myös, olisiko tuloksista hyötyä vastaavanlaisten organisaatioiden web-palveluissa.

Tutkielma jakautuu kahteen osaan. Luvut 2, 3 ja 4 käsittelevät ongelmakenttään liittyvää teoriaa. Ensimmäiseksi käydään läpi web-sovellusten kehittyminen ensimmäisistä CGI-pohjaisista ohjelmista 2000-luvun palveluperustaisiksi web-sovelluksiksi. Kolmannessa luvussa käsitellään tunnistautumisen ja pääsynvalvonnan käsitteitä ja niiden yhteyttä web-palveluihin. Luvussa 4 käsitellään tunnistautumista keskitettynä palveluna palveluperustaisten arkkitehtuurien näkökulmasta.

Tutkielman luvussa 5 esitellään Kapsi ry:n nykyinen järjestelmä ja siihen tehtävä arkkitehtuurin muutos. Luvussa 6 pohditaan, miten uusi arkkitehtuuri ratkaisee esitettyjä ongelmia, ja arvioidaan, millaisia parannuksia ja rajoitteita muutokset tuovat verrattuna nykyiseen toteutukseen. Myös laajennusmahdollisuuksia ja tulosten soveltamista vastaaviin järjestelmiin pohditaan. Luku 7 on yhteenvetoluku.