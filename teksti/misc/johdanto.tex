Ohjelmistoyritys Enemy \& Sons on tilannut selvityksen, jonka avulla yritys saa kattavan kuvan kuinka tunnistauminen voidaan toteuttaa web-sovelluksissa. Yrityksen tyypillisessä projektissa web-sovellus saadaan valmiina kokonaisuutena ja se on tarkoitus muuttaa skaalautuvaksi palveluperustaiseksi arkkitehtuuriksi. Järjestelmää pilkottaessa erillisiin komponentteihin, törmätään usein käyttäjän tunnistamisen ongelmaan.

Yritys haluaa ratkaisun, jossa yksittäisten komponenttien tunnistautumisongelma on selvitetty. Ratkaisun täytyy olla monistettavissa tyypillisiin web-sovellusprojekteihin. Enemy \& Sons käyttää ohjelmistoissaan moderneja web-ohjelmointikieliä ja toteutettu ratkaisu ei saa olla kieli- tai ympäristöriippuvainen. Järjestelmä rakennetaan avoimien rajapintojen ja protokollien päälle. Järjestelmän tulee olla laajennettavissa myös käyttäjän pääsynhallintaan, jolloin pelkkä tunnistautuminen ei riitä, vaan käyttäjällä täytyy olla oikeus käyttää palvelua. Pääsynhallintaa ei toteuteta tutkimuksessa syntyvään prototyyppiin, mutta käytettyjen ratkaisujen täytyy tukea myös sitä.

Tutkielma jakautuu kahteen osaan. Luvuissa 2, 3 ja 4 käsitellään ongelmakenttää yleisellä tasolla. Luvussa 2 esitellään web-sovellukset, erityisesti palveluperustaiset web-sovellukset. Luvussa 3 käydään läpi käyttäjän tunnistautumisen tekniikoita ja ongelmakenttää. Luvussa 4 käsitellään keskitettyä tunnistautumista palveluperustaisissa arkkitehtuureissa. Tutkielman toisessa osassa esitellään ratkaisu yrityksen kohtaamaan ongelmaan. Luku 5 esittelee toteutuksen, joka evaluoidaan luvussa 6. Luvussa 7 pohditaan toteutuksen laajennettavuutta. Luku 8 on yhteenvetokappale.


\subsection{Tutkimustavoitteet}
Tavoitteiden tarkka määrittely:
- tuottaa yritykselle haluttu selvitys
- pystytäänkö tekemään järjestelmä, jonka avulla palveluperustaisessa arkkitehtuurissa pystytään autentikoimaan keskitetysti
- eli tässä kappaleessa yritetään painottaa asioita, jotka tällä hetkellä sisältyy implisiittisesti seuraavien kappaleiden teksteihin
- prototyypin rajaus ja tarkemmat tutkimusongelmat luvussa 5

