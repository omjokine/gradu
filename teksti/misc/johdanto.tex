Yrityksen Enemy \& Sons toteuttamissa palvelusuuntauneissa web-ohjelmistoissa käyttäjien tunnistautuminen on toteutettu monin tavoin. Tyypillisesti kysytään käyttäjältä tunnus ja salasana, joita verrataan ohjelmiston paikalliseen käyttäjätietokantaan. Paikallinen tietokanta on kopioitu asiakasyrityksen varsinaisesta käyttäjätietokannasta ja paikallisiin tietokantoihin on käyttäjille luotu erillinen käyttäjätunnus.

Tästä on seurannut monenlaisia synkronointiongelmia. Esimerkiksi työntekijän irtisanoutuessa joudutaan tunnus poistamaan kaikista tietokannoista erikseen. Myös osoitteen yms. tietojen muutokset täytyy päivittää kaikkiin tietokantoihin. Lisäksi käyttäjälle syntyy saman järjestelmän sisällä monia tunnuksia, joihin saattaa liittyä erilliset salasanat.

Tutkielman tarkoitus on toteuttaa prototyyppi sovelluksesta, jonka avulla erillisten web-sovellusten käyttäjähallinta keskitetään yhteen komponenttiin. Sovellus hoitaa käyttäjän tunnistautumisen ja käyttäjätietojen hallinnan, joka synkronoidaan asiakasyrityksen käyttäjätietokannan kanssa. Tutkielmassa selvitetään myös tarvitaanko erillistä käyttäjätietokantaa ollenkaan vai voidaanko asiakasyrityksissä tyypillisesti käytössä olevia LDAP-hakemistoja käyttää toteutettavan komponentin käyttäjätietokantana.

Enemy \& Sons käyttää ohjelmistoissaan moderneja web-ohjelmointikieliä ja toteutettu ratkaisu ei saa olla kieli- tai ympäristöriippuvainen. Järjestelmä rakennetaan avoimien rajapintojen ja protokollien päälle. Järjestelmän tulee olla laajennettavissa myös käyttäjän autorisointiin, jolloin pelkkä tunnistautuminen ei riitä, vaan käyttäjällä täytyy olla oikeus käyttää palvelua. Autorisointia ei toteuteta tutkimuksessa syntyvään prototyyppiin, mutta käytettyjen ratkaisujen täytyy tukea myös sitä.

Luvussa 2 kuvataan keskitetyn tunnistautumisen ongelmakenttää ja näytetään siihen liittyviä käyttötapauksia. Luvussa 3 käydään läpi tunnistautumiseen liittyviä teknologioita käyttäjien hallinnasta tunnistautumisprotokolliin ja valitaan käytettävät teknologiat. Luvussa 4 esitellään prototyyppitoteutus tunnistautumiselle ja se evaluoidaan luvussa 5. Luvussa 6 pohditaan toteutuksen laajentamista kertakirjautumisen tai autorisoinnin suuntaan. Luku 7 on yhteenvetokappale.
