Web-sovellusten yleistyessä törmätään yhä useammin käyttäjän tunnistautumisen ongelmaan. Sovelluksen tarjoaja haluaa, että vain tietyillä käyttäjillä on pääsy järjestelmään. Tämä vaatii käyttäjän tunnistamisen, ts. sen varmistamisen, että käyttäjä on se, joka väittää olevansa. Kun web-sovelluksia on yhä enemmän, käyttäjän tunnistamiseen käytettäviä valtuutustietoja (credentials) syntyy useita, jolloin käyttäjän täytyy muistaa useita käyttäjätunnus/salasana-pareja \cite{billion_keys}.

Samaan ongelmaan törmätään usein myös intranet-palveluissa, kun niiden arkkitehtuuria viedään palveluperustaiseen suuntaan. Tällöin jokaisen palvelun täytyy pystyä tunnistamaan sen käyttäjät, jotta heille voidaan näyttää käyttöoikeuksien mukaan sisältöä. Intranetin palveluilla voi olla oma käyttäjätietokanta, joka synkronoidaan päätietokannan kanssa, jolloin yksittäisellä palvelulla on ajantasainen tieto kunkin käyttäjän oikeudesta käyttää palvelua. Tässä tapauksessa ongelmallista on, että käyttäjän tunnukset ja salasanat siirtyvät useaan järjestelmään, mikä heikentää järjestelmän tietoturvaa \cite{nisti}. Jos taas eri intranetin palveluihin luodaan oma tunnus, syntyy uusi muistettava tunnus/salasana-pari.

Palveluille voidaan myös avata pääsy organisaation käyttäjähallintaan, jolloin palveluun pääsy edellyttää, että käyttäjän syöttämä tunnus/salasana-pari löytyy käyttäjähallinnasta. Tämä voi olla järjestelmän ylläpitäjän näkökulmasta ongelmallista, sillä käyttäjäkanta sisältää tietoja, joiden ei haluta päätyvän ulkopuolisten käsiin. Yhtenä ratkaisuna on esitetty keskitettyä tunnistautumispalvelua, jolla on pääsy käyttäjähallintaan ja jota vasten yksittäiset web-sovellukset tunnistavat käyttäjänsä \cite{nisti}.

Tässä tutkielmassa paneudutaan tunnistautumisen ongelmakenttään ja tutkitaan, ratkaiseeko keskitetty tunnistautumispalvelu esitetyt ongelmat ja millaisia mahdollisia uusia ongelmia se tuo tullessaan. Tutkielmassa pohditaan myös, millaisia etuja erillisestä tunnistautumispalvelusta on verrattuna suoraan integraatioon käyttäjähallintaan. Etuja ja haittoja arvioidaan järjestelmän ylläpitäjän, käyttäjän ja web-sovellusohjelmoijan näkökulmasta.

Tutkielmassa ei keskitytä kaikille avoimiin tunnistautumispalveluihin (kuten Facebook tai Google), vaan organisaatioihin, joilla on oma käyttäjähallinta. Esimerkki tällaisesta organisaatiosta on Helsingin yliopisto, jolla on oma Active Directory -käyttäjähallinta \cite{tietotekniikkaa}. Myös monet yritykset ylläpitävät omaa käyttäjähallintaa esimerkiksi LDAP-järjestelmässä tai keskitetyssä tietokannassa. Esimerkkinä hyötyjä ja haittoja punnittaessa käytetään Helsingin yliopistoa, mutta myös muunlaiset organisaatiot huomioidaan, mikäli se on perusteltua. TODO: tämä kappale uusiksi, kun 5-7 hahmottuu

Tutkielma jakautuu kahteen osaan. Luvuissa 2, 3 ja 4 käsitellään ongelmakenttään liittyvää teoriaa. Luvussa 2 esitellään web-sovellukset, erityisesti palveluperustaiset web-sovellukset. Luvussa 3 käydään läpi käyttäjän tunnistautumisen tekniikoita ja ongelmakenttää. Luvussa 4 käsitellään keskitettyä tunnistautumista palveluperustaisissa arkkitehtuureissa. Luvut ...
