Web-sovellusten yleistyessä törmätään yhä useammin käyttäjän tunnistautumisen ongelmaan. Sovelluksen tarjoaja haluaa, että vain tietyillä käyttäjillä on pääsy järjestelmään. Tämä vaatii käyttäjän tunnistamisen, ts. sen varmistamisen, että käyttäjä on se, joka väittää olevansa. Kun web-sovelluksia on yhä enemmän, käyttäjän tunnistamiseen käytettäviä valtuutustietoja (credentials) syntyy useita, jolloin käyttäjän täytyy muistaa useita käyttäjätunnus/salasana-pareja \cite{billion_keys}.

Samaan ongelmaan törmätään usein myös intranet-palveluissa, kun niiden arkkitehtuuria viedään palveluperustaiseen suuntaan. Tällöin jokaisen palvelun täytyy pystyä tunnistamaan sen käyttäjät, jotta heille voidaan näyttää käyttöoikeuksien mukaan sisältöä. Intranetin palveluilla voi olla oma käyttäjätietokanta, joka synkronoidaan päätietokannan kanssa, jolloin yksittäisellä palvelulla on ajantasainen tieto kunkin käyttäjän oikeudesta käyttää palvelua. Tässä tapauksessa ongelmallista on, että käyttäjän tunnukset ja salasanat siirtyvät useaan järjestelmään, mikä heikentää järjestelmän tietoturvaa \cite{nisti}. Jos taas eri intranetin palveluihin luodaan oma tunnus, syntyy uusi muistettava tunnus/salasana-pari.

Palveluille voidaan myös avata pääsy organisaation käyttäjähallintaan, jolloin palveluun pääsy edellyttää, että käyttäjän syöttämä tunnus/salasana-pari löytyy käyttäjähallinnasta. Tämä voi olla järjestelmän ylläpitäjän näkökulmasta ongelmallista, sillä käyttäjäkanta sisältää tietoja, joiden ei haluta päätyvän ulkopuolisten käsiin. Yhtenä ratkaisuna on esitetty keskitettyä tunnistautumispalvelua, jolla on pääsy käyttäjähallintaan ja jota vasten yksittäiset web-sovellukset tunnistavat käyttäjänsä \cite{nisti}.

Tässä tutkielmassa paneudutaan tunnistautumisen ongelmakenttään ja tutkitaan, ratkaiseeko keskitetty tunnistautumispalvelu esitetyt ongelmat ja millaisia mahdollisia uusia ongelmia se tuo tullessaan. Tutkielmassa pohditaan myös, millaisia etuja erillisestä tunnistautumispalvelusta on verrattuna suoraan integraatioon käyttäjähallintaan. Etuja ja haittoja arvioidaan järjestelmän ylläpitäjän, käyttäjän ja web-sovellusohjelmoijan näkökulmasta.

Tutkielmassa ei keskitytä kaikille avoimiin tunnistautumispalveluihin (kuten Facebook tai Google), vaan organisaatioihin, joilla on oma käyttäjähallinta. Esimerkkinä tällaisesta organisaatiosta käytetään Kapsi ry:tä, joka ylläpitää erilaisia web-palveluita, joihin jäsenistöllä on pääsy. Kirjallisuuskatsauksen pohjalta luodaan esimerkkiarkkitehtuuri, jolla keskitetty tunnistautumispalvelu saadaan osaksi Kapsin web-palveluita. Tutkielmassa pohditaan hyötyjä ja haittoja, joita keskitetyllä tunnistautumisratkaisulla saadaan Kapsin tapauksessa sekä arvioidaan olisiko tuloksista hyötyä myös vastaavanlaisten organisaatioiden web-palveluissa.

Tutkielma jakautuu kahteen osaan. Luvut 2, 3 ja 4 käsittelevät ongelmakenttään liittyvää teoriaa. Ensin käydään läpi web-sovellusten kehittyminen ensimmäisistä CGI-pohjaisista ohjelmista nykyisiksi palveluperustaisiksi web-sovelluksiksi. Toisessa teorialuvussa käsitellään tunnistautumisen ja pääsynvalvonnan käsitteitä ja niiden yhteyttä web-palveluihin. Viimeisessä teorialuvussa käsitellään tunnistautumisista keskitettynä palveluna palveluperustaisissa arkkitehtuureissa.

Tutkielman jälkimmäisessä osassa esitellään Kapsi ry:n nykyinen järjestelmä ja siihen tehtävä arkkitehtuurin muutos luvussa 5. Luvussa 6 pohditaan miten uusi arkkitehtuuri ratkaisee esitettyjä ongelmia sekä sen tuomia parannuksia ja rajoitteita verrattuna nykyiseen toteutukseen. Myös laajennusmahdollisuuksia ja tulosten soveltamista vastaaviin järjestelmiin pohditaan. Luku 7 on yhteenvetoluku.