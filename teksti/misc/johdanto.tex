Yksittäisten web-sovellusten yleistyessä, törmätään usein käyttäjän tunnistautumiseen liittyvään ongelmaan. Sovelluksen tarjoaja haluaa, että vain tietyillä käyttäjillä on pääsy järjestelmään. Tämä vaatii, että käyttäjä on tunnistettu, ts. voidaan varmistua siitä, että käyttäjä on se, joka väittää olevansa.

Samaan ongelmaan törmätään usein myös suljetuissa yhteisössä, kun web-palveluiden arkkitehtuuria viedään palveluperustaiseen suuntaan. Jos esimerkiksi yrityksen int\-ra\-net-järjestelmään luodaan uutta web-palvelua, johon vain yrityksen työntekijöillä on pääsy, joudutaan tekemään integraatiota olemassa oleviin käyttäjätietokantoihin. Joissakin tapauksissa int\-ra\-net-sovelluksilla on oma käyttäjähallintansa ja käyttäjän oikeus järjestelmään tarkistetaan esimerkiksi rekisteröitymisen yhteydessä. Tällöin käyttäjälle syntyy uusi muistettava tunnus/salasana-pari. Kolmannessa tapauksessa käyttäjätietokannasta voidaan luoda oma kopio sovellusohjelman käyttöön, josta taas seuraa erilaisia synkronointiongelmia, kun sovellusohjelman käyttäjätietokanta täytyy pitää ajantasaisena päätietokantaan verrattuna.

Tämän tutkielman kannalta oleelliset kysymykset ovat: onko yksittäisten web-so\-vel\-lus\-ten liittäminen suoraan nykyiseen käyttäjähallintaan järkevää, vai voitaisiinko tunnistautuminen ja käyttäjän tietojen haku toteuttaa erillisenä web-palveluna, keskitettynä identiteetintarjoajana? Ratkaiseeko keskitetty palvelu edellä listattuja ongelmia ja millaisia uusia ongelmia se mahdollisesti tuo tullessaan? Millainen järjestelmän kokonaisarkkitehtuuri syntyy, kun identiteetintarjoaja toteutetaan palveluna? Millaisia etuja ja haittoja arkkitehtuuri tuo järjestelmän ylläpitäjän, käyttäjän tai web-sovellusohjelmoijan näkökulmasta?

Tutkielmassa ei keskitytä kaikille avoimiin identiteetintarjoajiin (kuten Facebook tai Google), vaan nimenomaan organisaatioihin, joilla on valmis oma käyttäjähallinta, joka tukee myös käyttäjien tunnistautumista. Esimerkki tällaisesta organisaatiosta on Helsingin Yliopisto, jolla on oma Active Directory -käyttäjähallinta [http://www.helsinki.fi/atk/lehdet/110/artikkeli1.html]. Myös monet yritykset ylläpitävät omaa käyttäjähallintaa esimerkiksi LDAP-järjestelmässä tai tietokannassa [lähde?].

Tutkielma jakautuu kahteen osaan. Luvuissa 2, 3 ja 4 käsitellään ongelmakenttään liittyvää teoriaa. Luvussa 2 esitellään web-sovellukset, erityisesti palveluperustaiset web-sovellukset. Luvussa 3 käydään läpi käyttäjän tunnistautumisen tekniikoita ja ongelmakenttää. Luvussa 4 käsitellään keskitettyä tunnistautumista palveluperustaisissa arkkitehtuureissa. Luvut ...
