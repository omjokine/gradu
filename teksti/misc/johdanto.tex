Web-sovellusten yleistyessä törmätään yhä useammin käyttäjän tunnistamisen ongelmaan. Kun halutaan, että vain tietyillä käyttäjillä on pääsy järjestelmään. Tämä vaatii sen varmistamisen, että käyttäjä on se, joka väittää olevansa. Web-sovellusten määrän kasvaessa käyttäjän tunnistamiseen käytettäviä valtuutustietoja (credentials) syntyy useita, jolloin käyttäjän täytyy muistaa useita käyttäjätunnus/salasana-pareja \cite{billion_keys}.

Samaan ongelmaan törmätään usein myös intranet-palveluissa, kun joiden arkkitehtuuria viedään palveluperustaiseen suuntaan. Tällöin jokaisen palvelun täytyy pystyä tunnistamaan sen käyttäjät, jotta heille voidaan näyttää käyttöoikeuksien mukaista sisältöä. Intranet-palveluilla voi olla oma käyttäjätietokanta, joka synkronoidaan päätietokannan kanssa, jolloin yksittäisellä palvelulla on ajantasainen tieto käyttäjien oikeuksista käyttää palvelua. Tässä tapauksessa ongelmallista on, että käyttäjätunnukset ja salasanat siirtyvät useaan järjestelmään, mikä heikentää järjestelmän tietoturvaa \cite{nisti}. Mikäli näin ei tehtäisi, syntyisi jälleen jokaista palvelua kohtaan uusi muistettava tunnus/salasana-pari.

Web-palveluille voidaan myös avata pääsy organisaation käyttäjähallintaan, jolloin palveluun pääsy edellyttää, että käyttäjän syöttämä tunnus/salasana-pari löytyy käyttäjähallinnasta. Tämä voi olla järjestelmän ylläpitäjän näkökulmasta ongelmallista, sillä käyttäjätietokanta sisältää tietoja, joiden ei haluta päätyvän ulkopuolisten käsiin. Yhtenä ratkaisuna on esitetty keskitettyä tunnistautumispalvelua, jolla on pääsy käyttäjähallintaan ja jonka avulla yksittäiset web-sovellukset tunnistavat käyttäjänsä \cite{nisti}.

Tässä tutkielmassa paneudutaan käyttäjän tunnistamisen ongelmakenttään ja tutkitaan, ratkaiseeko keskitetty tunnistautumispalvelu esitetyt ongelmat ja millaisia mahdollisia uusia ongelmia sen käyttö tuo tullessaan. Tutkielmassa pohditaan myös, millaisia etuja erillisestä tunnistautumispalvelusta on verrattuna suoraan integraatioon käyttäjähallintaan. Etuja ja haittoja arvioidaan järjestelmän ylläpitäjän, käyttäjän ja web-sovellusohjelmoijan näkökulmasta.

Tutkielmassa ei keskitytä kaikille avoimiin tunnistautumispalveluihin (kuten Facebook tai Google), vaan organisaatioihin, joilla on oma käyttäjähallinta. Esimerkkinä tällaisesta organisaatiosta käytetään Kapsi ry:tä, joka ylläpitää erilaisia web-palveluita, joihin vain Kapsin jäsenillä on pääsy. Kirjallisuuskatsauksen pohjalta luodaan esimerkkiarkkitehtuuri, jolla keskitetty tunnistautumispalvelu saadaan osaksi Kapsin web-palveluita. Tutkielmassa pohditaan hyötyjä ja haittoja, joita keskitetyssä tunnistautumisratkaisussa  esiintyy. Kapsin tapauksen perusteella arvioidaan myös olisiko tuloksista hyötyä vastaavanlaisten organisaatioiden web-palveluissa.

Tutkielma jakautuu kahteen osaan. Luvut 2, 3 ja 4 käsittelevät ongelmakenttään liittyvää teoriaa. Ensin käydään läpi web-sovellusten kehittyminen ensimmäisistä CGI-pohjaisista ohjelmista 2000-luvun palveluperustaisiksi web-sovelluksiksi. Kolmannessa luvussa käsitellään tunnistautumisen ja pääsynvalvonnan käsitteitä ja niiden yhteyttä web-palveluihin. Luvussa 4 käsitellään tunnistautumista keskitettynä palveluna palveluperustaisten arkkitehtuurien näkökulmasta.

Tutkielman luvussa 5 esitellään Kapsi ry:n nykyinen järjestelmä ja siihen tehtävä arkkitehtuurin muutos. Luvussa 6 pohditaan miten uusi arkkitehtuuri ratkaisee esitettyjä ongelmia ja arvioidaan millaisia parannuksia ja rajoitteita muutokset tuovat verrattuna nykyiseen toteutukseen. Myös laajennusmahdollisuuksia ja tulosten soveltamista vastaaviin järjestelmiin pohditaan. Luku 7 on yhteenvetoluku.