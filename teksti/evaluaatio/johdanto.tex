Edellisessä luvussa esitelty esimerkkiarkkitehtuuri ratkaisee monia nykyisessä Kapsin arkkitehtuurissa olevia ongelmia, mutta se tuo mukanaan myös joitakin uusia ongelmia. Arkkitehtuuria voidaan myös laajentaa tulevaisuudessa esimerkiksi tukemaan kertakirjautumista tai keskitettyä pääsynvalvontaa. Arkkitehtuurin soveltaminen vastaavanlaisissa ympäristöissä on mahdollista. Arkkitehtuurin soveltaminen on mahdollista myös ilman omaa tunnistautumispalvelua, vaan tunnistautumispalveluna voidaan käyttää ulkopuolisen palveluntarjoajan komponenttia.

Tämän luvun ensimmäisessä aliluvussa käydään läpi uuden arkkitehtuurin etuja Kapsin ylläpitäjien, jäsenten ja web-sovellusten toteuttajien näkökulmasta. Toisessa aliluvussa pohditaan mitä ongelmia ja rajoituksia uuteen arkkitehtuuriin liittyy. Aliluvussa 6.3 käydään läpi arkkitehtuurin laajentamista kertakirjautumiseen tai keskitettyyn pääsynvalvontaan. Viimeisessä aliluvussa 6.4 sovelletaan arkkitehtuuria, ja sen periaatteita, vastaavissa ympäristöissä. Esimerkiksi on valittu Helsingin yliopiston tietojenkäsittelytieteen laitos, pk-yrityksen intranet ja yksittäiset web-sovellukset, jotka haluavat tunnistaa käyttäjät, mutta joiden ylläpitäjät eivät välttämättä halua toteuttaa omaa käyttäjähallintaa.