Edellisessä luvussa esitelty esimerkkiarkkitehtuuri ratkaisee monia nykyisessä Kapsin arkkitehtuurissa olevia ongelmia, mutta se tuo mukanaan myös joitakin uusia ongelmia arkkitehtuurin monimutkaistuessa. Toisaalta uusi arkkitehtuuri mahdollistaa kertakirjautumisen tai keskitetyn pääsynvalvonnan toteuttamisen tulevaisuudessa. Esimerkkiarkkitehtuuri noudattaa myös palveluperustaisten arkkitehtuurien periaatetta, jonka mukaan jokaisella komponentilla on oma tarkoin määritelty tehtävänsä.

Arkkitehtuuria voidaan soveltaa myös muissa Kapsin tyyppisissä ympäristöissä joko sellaisenaan tai osia siitä. Esimerkiksi keskitettynä tunnistautumispalveluna voidaan käyttää ulkopuolisen palveluntarjoajan komponenttia, jolloin web-sovellukseen ei tarvita omaa käyttäjähallintaa.

Tämän luvun ensimmäisessä alaluvussa käydään läpi uuden arkkitehtuurin etuja Kapsin ylläpitäjien, jäsenten ja web-sovellusten toteuttajien näkökulmasta. Toisessa alaluvussa pohditaan, mitä ongelmia ja rajoituksia uuteen arkkitehtuuriin liittyy. Alaluvussa 6.3 käydään läpi arkkitehtuurin laajentamista kertakirjautumiseen ja keskitettyyn pääsynvalvontaan. Viimeisessä alaluvussa 6.4 sovelletaan arkkitehtuuria ja sen periaatteita Kapsia vastaavissa ympäristöissä. Esimerkiksi on valittu Helsingin yliopiston tietojenkäsittelytieteen laitoksen web-sovellukset, pk-yrityksen intranet ja yksittäiset web-sovellukset, jotka haluavat tunnistaa käyttäjät, mutta joiden ylläpitäjät eivät välttämättä halua toteuttaa omaa käyttäjähallintaa.