Sikteerin käyttäjän näkökulmasta uuden järjestelmän toiminta ei poikkea merkittävästi vanhasta ja uuden arkkitehtuurin tuomat edut ovat lähinnä epäsuoria. Sikteerin tarjoaman kirjautumislomakkeen sijaan käyttäjä ohjataan tekemään tunnistautuminen ulkoiselle palvelimelle. Tämä parantaa järjestelmän tietoturvaa ja vähentää riskiä käyttäjän tietojen vuotamisesta ulkopuoliselle taholle.

Arkkitehtuurin muutoksen jälkeen Sikteerissä ei käytetä enää erillistä salasanaa, vaan muutoksen jälkeen sama salasana on käytössä kaikissa Kapsin web-sovelluksissa. Myös mahdollisiin tuleviin sovelluksiin ei tarvita enää erillistä salasanaa, vaan samalla salasanalla voi kirjautua kaikkiin Kapsin palveluihin. Tällainen uusi sovellus voisi olla esimerkiksi web-sovellus, jonka kautta käyttäjä voi lisätä jäsenetuihin kuuluvia sähköpostitunnuksia. Myös mahdollisuus saada käyttöön uusia web-sovelluksia on epäsuora etu käyttäjälle.

Jos arkkitehtuuri laajennetaan mahdollistamaan kertakirjautuminen, myös käyttäjät hyötyvät entistä enemmän, koska jokaiseen järjestelmän palveluun ei tarvitse enää kirjautua erikseen. Tämä vaatii jonkun verran laajennuksia arkkitehtuuriin, ja niitä käsitellään tarkemmin alaluvussa 6.3.