Kapsin ylläpidon näkökulmasta suurin etu on järjestelmän yleisen tietoturvan parantuminen. Käyttäjähallintaa voidaan pitää kriittisenä komponenttina ja kun vain tunnistautumispalvelu integroituu siihen, paranee järjestelmän yleinen tietoturva \cite{arkkitehtuurit}. Lisäksi käyttäjien tunnistautumistietoja ei enää tarvitse kopioida sovelluksien käyttöön, koska käyttäjät tunnistetaan OAuth-protokollan avulla \cite{oauth2_0}. Riski salasanatiivisteiden tms tunnistetietojen leviämisestä järjestelmän ulkopuolelle pienenee, koska kriittinen käyttäjädata on keskitetty yhteen paikkaan.

Muutosten tekeminen tunnistautumisen tekniikoihin helpottuu uuden arkkitehtuurin myötä. Esimerkiksi Googlen palveluita käytettäessä on mahdollista käyttää niin kutsuttua kahden tekijän tunnistautumista (two factor authentication) \cite{google_two_factor}. Tällöin tunnuksen ja salasanan lisäksi palvelu pyytää syöttämään tunnistuskoodin, joka lähetetään käyttäjän matkapuhelimeen. Tällöin tiedetään, että tunnuksen ja salasanan lisäksi käyttäjällä on pääsy myös järjestelmään rekisteröityyn matkapuhelimeen, joten tunnistusta voidaan pitää normaalia tunnistautumista tehokkaampana \cite{}. Tällaisen tunnistautumista vahvistavan toimenpiteen tuominen Kapsin järjestelmiin on mahdollista ilman, että web-palveluiden koodia tarvitsee muuttaa, kun esimerkkiarkkitehtuurin mukainen tunnistautumispalvelu on käytössä.