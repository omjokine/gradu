Kapsin ylläpidon näkökulmasta suurin etu on järjestelmän yleisen tietoturvan parantuminen. Käyttäjähallintaa voidaan pitää kriittisenä komponenttina ja kun vain tunnistautumispalvelu integroituu siihen, paranee järjestelmän yleinen tietoturva \cite{arkkitehtuurit}. Lisäksi käyttäjien tunnistautumistietoja ei enää tarvitse kopioida sovelluksien käyttöön, koska käyttäjät tunnistetaan OAuth-protokollan avulla \cite{oauth2_0}. Riski salasanatiivisteiden tai muiden tunnistetietojen leviämisestä järjestelmän ulkopuolelle pienenee, koska kriittinen käyttäjädata on keskitetty yhteen paikkaan.

Muutosten tekeminen tunnistamisen tekniikoihin helpottuu uuden arkkitehtuurin myötä. Esimerkiksi kahden tekijän tunnistaminen (two factor authentication) voidaan ottaa käyttöön. Kahden tekijän tunnistamisessa käyttäjän tunteman (salasana) asian lisäksi käytetään jotain käyttäjän omistamaa \cite{nisti}. Palvelu voi pyytää syöttämään tunnistuskoodin, joka on lähetetty käyttäjän matkapuhelimeen. Kun käyttäjä syöttää oikean koodin, tiedetään käyttäjällä olevan hallussaan järjestelmään aiemmin rekisteröity matkapuhelin. Tällöin tunnistamiseen on käytetty kahta eri tekijää, joten tunnistusta voidaan pitää yhden tekijän tunnistautumista tehokkaampana \cite{nisti}. Tällaisen tunnistautumista vahvistavan toimenpiteen tuominen Kapsin järjestelmiin on mahdollista ilman, että web-palveluiden koodia tarvitsee muuttaa, kun esimerkkiarkkitehtuurin mukainen tunnistautumispalvelu on käytössä.