Kapsin ylläpidon näkökulmasta uuden arkkitehtuurin suurin etu on järjestelmän yleisen tietoturvan parantuminen. Käyttäjähallintaa voidaan pitää kriittisenä komponenttina ja vain tunnistautumispalvelun integroituessa käyttäjähallintaan paranee järjestelmän yleinen tietoturva \cite{arkkitehtuurit}. Lisäksi käyttäjien tunnistautumistietoja ei enää tarvitse kopioida sovelluksien käyttöön, koska käyttäjät tunnistetaan OAuth-protokollan avulla \cite{oauth2_0}. Tällöin riski salasanatiivisteiden tai muiden tunnistetietojen leviämisestä järjestelmän ulkopuolelle pienenee, koska kriittinen käyttäjädata on keskitetty yhteen paikkaan.

Toinen etu uudesta arkkitehtuurista on muutosten tekemisen helpottuminen tunnistamisen tekniikoihin. Esimerkiksi kahden tekijän tunnistaminen (two factor authentication) voidaan ottaa käyttöön. Kahden tekijän tunnistamisessa käyttäjän tunteman (salasana) asian lisäksi käytetään jotain käyttäjän omistamaa \cite{nisti}. Palvelu voi pyytää syöttämään tunnistuskoodin, joka on lähetetty käyttäjän matkapuhelimeen. Kun käyttäjä syöttää oikean koodin, tiedetään käyttäjällä olevan hallussaan järjestelmään aiemmin rekisteröity matkapuhelin. Tällöin tunnistamiseen on käytetty kahta eri tekijää, joten tunnistusta voidaan pitää yhden tekijän tunnistautumista tehokkaampana \cite{nisti}.

Tällaisen tunnistautumista vahvistavan toimenpiteen tuominen Kapsin järjestelmiin on mahdollista ilman, että web-sovellusten koodia tarvitsee muuttaa, kun esimerkkiarkkitehtuurin mukainen tunnistautumispalvelu on käytössä. Muutos on mahdollinen, koska web-sovelluksen ja tunnistautumispalvelun välillä on OAuth-rajapinta. Web-sovellus saa tunnistautumispalvelusta pääsyvaltuuden, jolla käyttäjätiedot voidaan hakea, joten tunnistautumispalvelun toimintalogiikka ei vaikuta web-sovelluksen toimintaan.