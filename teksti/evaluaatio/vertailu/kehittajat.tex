Kehittäjän näkökulmasta suurin etu uudesta arkkitehtuurista on web-sovellusten entistä helpompi liitettävyys osaksi Kapsin järjestelmää. Web-sovelluksen kehittäjän ei tarvitse rakentaa sovellukseensa käyttäjähallintaa, vaan se on käytettävissä ulkoisena palveluna. Sovelluksen ohjelmoijan ei tarvitse myöskään huolehtia käyttäjätietokannan synkronoinnista, vaan käyttäjätiedot ovat aina ajan tasalla.

Kun palveluiden ja tunnistautumispalvelun välillä käytetään luotettavaa protokollaa, voivat ylläpitäjät sallia web-sovellusten pääsyn tunnistautumispalveluun, vaikka eivät tuntisikaan web-sovelluksen toimintaa. Esimerkiksi Facebook hyödyntää tätä ominaisuutta tarjoamalla käyttäjiensä toteuttamille web-sovelluksille mahdollisuuden käyttää Facebook-tunnuksia tunnistautumiseen. Samaan tapaan Kapsin jäsenet voisivat tuottaa omia palveluita muiden jäsenten käyttöön. Uuden arkkitehtuurin myötä myös muut kuin järjestelmän ylläpitäjät voivat Facebookin tapaan tehdä omia palveluita Kapsin jäsenille.