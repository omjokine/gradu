Kehittäjän näkökulmasta suurin etu uudesta arkkitehtuurista on web-sovellusten helpompi liitettävyys osaksi Kapsin järjestelmiä. Web-sovellusten kehittäjän ei tarvitse rakentaa sovellukseensa käyttäjähallintaa, vaan se on käytettävissä ulkoisena palveluna.

Kun palveluiden ja tunnistautumispalvelun välillä käytetään luotettavaa protokollaa, voi ylläpitäjät sallia web-palveluiden pääsyn tunnistautumispalveluun, vaikka eivät tuntisikaan web-palvelun toimintaa. Esimerkiksi Facebook hyödyntää tätä ominaisuutta tarjoamalla käyttäjiensä toteuttamille web-sovelluksille mahdollisuuden käyttää Facebook-tunnuksia tunnistautumiseen. Samaan tapaan Kapsin jäsenet voisivat tuottaa omia palveluita muiden jäsenten käyttöön. Uuden arkkitehtuurin myötä myös muut kuin järjestelmän ylläpitäjät voivat Facebookin tapaan tehdä omia palveluita Kapsin jäsenille.