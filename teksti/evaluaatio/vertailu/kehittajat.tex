Kehittäjän näkökulmasta suurin etu uudesta arkkitehtuurista on web-sovellusten helpompi liitettävyys osaksi Kapsin järjestelmiä. Web-sovellusten kehittäjän ei tarvitse rakentaa sovellukseensa käyttäjänhallintaa, vaan se on käytettävissä ulkoisena palveluna.

Uudella arkkitehtuurilla myös yhä useammat henkilöt, esimerkiksi Kapsin jäsenet, voivat kehittää jäsenille suunnattuja web-sovelluksia, koska jäsenten tunnistetietoja ei ole enää tarvitse kopioida jokaiselle sovellukselle. Kun palveluiden ja tunnistautumispalvelun välillä käytetään luotettavaa protokollaa, voi ylläpitäjät sallia web-palveluiden pääsyn tunnistautumispalveluun, vaikka eivät tuntisikaan web-palvelun toimintaa. Esimerkiksi Facebook hyödyntää tätä ominaisuutta tarjoamalla käyttäjiensä toteuttamille web-sovelluksille mahdollisuuden käyttää Facebook-tunnuksia tunnistautumiseen [TODO: lähde, tarvitaanko?]. Samaan tapaan Kapsin jäsenet voisivat tuottaa omia palveluita muiden jäsenten käyttöön.