Tutkielman (erityisesti luvun 5) kriittinen arviointi, osattiinko esitellyllä arkkitehtuurilla vastata johdannossa esitettyihin kysymyksiin.

Vielä hieman auki, mutta pituus noin 5 sivua.

Tutkielmassa ei keskitytä kaikille avoimiin tunnistautumispalveluihin (kuten Facebook tai Google), vaan organisaatioihin, joilla on oma käyttäjähallinta. Esimerkki tällaisesta organisaatiosta on Helsingin yliopisto, jolla on oma Active Directory -käyttäjähallinta \cite{tietotekniikkaa}. Myös monet yritykset ylläpitävät omaa käyttäjähallintaa esimerkiksi LDAP-järjestelmässä tai keskitetyssä tietokannassa. Esimerkkinä hyötyjä ja haittoja punnittaessa käytetään Helsingin yliopistoa, mutta myös muunlaiset organisaatiot huomioidaan, mikäli se on perusteltua. TODO: tämä kappale uusiksi, kun 5-7 hahmottuu

\subsection{Esimerkkiarkkitehtuurin laajennettavuus}
Kuinka voisi laajentaa? Kertakirjautuminen ja autorisointi hyviä suuntia. SSO aika peruskamaa OAuthin kanssa, autorisoinnista sen sijaan voi saada ihan mielenkiintoista pohdintaa aikaan.

LDAP:n hyödyntäminen autorisoinnissa, voisiko esimerkiksi käyttäjäryhmät olla tallennettu jotenkin kätevästi LDAP:in ja niiden perusteella voidaan autorisoida pääsy resurssiin.

Sinällään pääsynhallinta tiettyihin palveluihin on jonkin tason autorisointia ja kuuluu ehdottomasti gradun aihepiiriin (toteutus tunnistaa x:n palvelun käyttäjiä samaa käyttäjädataa vasten, kaikille käyttäjille ei saa antaa oikeutta kaikkiin palveluihin).

Miten käyttäjädataa onko käsitelty ja käsitellään. Kehitys paikallisesti käytetyistä tiedostopohjaisista systeemeistä kohti tietokantoja ja asiaan räätälöihin palveluihin (LDAP). LDAP oleelisin, mutta tutkimuksen kannalta abstraktointi on tärkeä juttu.

Johdanto puoli sivua, alaluvut 0.5-1 sivu.
\subsection{Kertakirjautuminen}
Kertakirjautumisella (SSO) tarkoitetaan arkkitehtuuria, jossa yhdellä kirjautumisella tunnistaudutaan useaan eri palveluun \cite{sso}. Käytetyissä kertakirjautumistekniikoissa käytetään identiteetintarjoajaa (identity provider), joka varmentaa kirjautumisen ja luo valtuutuksen web-sovellukseen. Käytetyt tekniikat jaetaan kahteen osaan: näennäiseen- (pseudo SSO) ja tosikertakirjautumiseen (true SSO) \cite{sso}. Näennäisessä kertakirjautumisessa identiteetintarjoaja luo palveluun sopivan valtuutuksen, jolloin web-sovellus ei tiedä kertakirjautumisesta mitään. Tosikertakirjautumisessa taas käytetään yleisiä järjestelmänlaajuisia valtuutuksia, jolloin web-sovellusten täytyy olla tietoisia kertakirjautumisesta.

Kapsin nykyisessä arkkitehtuurissa käyttäjä voi kirjautua samalla tunnuksella eri palveluihin, mutta kyseessä ei ole kertakirjautuminen, koska kirjautumisen täytyy tehdä erikseen jokaiseen palveluun. Kertakirjautuminen on mahdollista tehdä myös nykyisessä arkkitehtuurissa, esimerkiksi tallentamalla tunnistetiedot selaimen evästeeseen (cookie). Toinen web-sovellus voi taas lukea tunnistetiedot evästeestä ja tunnistaa käyttäjän tätä kautta. Koska eri web-sovellukset toimivat samassa domain-osoitteessa, ne voivat lukea toistensa kirjoittamia evästeitä \cite{rfc6265}. Tällainen ratkaisu on kuitenkin ongelmallinen, koska tunnistetiedot tallennetaan käyttäjän koneelle ja ne on luettavissa evästeestä. Jos taas tunnistetietojen kirjoittamiseen ja lukemiseen käytetään salausta, joudutaan salausavain jakamaan web-sovellusten kesken. Tämä lisää ylläpidon tarvetta.

Keskitettyä tunnistautumispalvelua käytettäessä kertakirjautuminen voidaan toteuttaa OAuth-protokollalla \cite{distributed_web_security}. Kontrollin kulku on samanlainen kuin kuvassa \ref{auth_kapsi_fi_flow}, mutta käyttäjän ei tarvitse syöttää tunnusta ja salasanaa, koska hänellä on jo voimassaoleva istunto tunnistautumispalveluun. Toisin sanoen käyttäjäagentti (selain) saa pääsyvaltuuden tunnistautumispalvelusta ilman uutta kirjautumista, jolloin kyseessä on näennäinen kertakirjautuminen. Kuitenkin ensimmäistä kertaa kirjautuessa web-sovellukseen käyttäjän täytyy hyväksyä tietojen haku tunnistautumispalvelusta. Kun tietojen luovutus on hyväksytty, voidaan jatkossa kirjautuminen tehdä ilman käyttäjän syötettä.
\subsection{Autorisointi}
Esitetyssä arkkitehtuurissa OAuth-protokollaa käytetään tunnistautumisen toteuttamiseen, mutta sen käyttöä voidaan laajentaa myös muuhun pääsynvalvontaan \cite{distributed_web_security}. Käyttö on varsin perusteltua, sillä OAuth on nimenomaan pääsynvalvontaprotokolla ja tunnistamisen toteuttaminen sitä käyttäen on erikoiskäyttötapaus. OAuthia käyttäen tunnistautumispalvelu voidaan laajentaa suojaamaan Kapsin järjestelmissä käytettäviä resursseja.

Jäsenen tiedot voivat olla yksi resurssi, joiden käyttöä valvotaan auth.kapsi.fi-pal\-ve\-lun kautta. Tällöin Sikteeri saa käyttäjätietokannasta tiedot vain, jos käyttäjällä olisi esittää pääsyvaltuutus, joka oikeuttaa tietojen hakemiseen. Näin Sikteeri ei olisi suoraan yhteydessä käyttäjätietokantaan, vaan arkkitehtuuriin lisätään palvelu, joka mahdollistaa käyttäjän hakemisen, lisäämisen, muokkaamisen tai poistamisen. Tämä palvelu on yhteydessä LDAP-käyttäjätietokantaan ja se antaa lukea tietoja vain jos käyttäjällä on esittää oikea pääsyvaltuutus. Pääsyvaltuuksia hallinnoi tunnistautumispalvelu.

Arkkitehtuurin laajentaminen koskemaan myös pääsynvalvontaa veisi sitä entistä enemmän kohti palveluperusteista arkkitehtuuria, jossa jokaisella komponentilla on yksi tehtävä, jonka ne suorittavat hyvin \cite{soa}. Sikteeri olisi siinä tapauksessa käyttöliittymä komponentille, joka hallinnoi käyttäjiä. Päätös käyttäjän oikeudesta hallinnoida käyttäjiä ei ole Sikteerin vastuulla vaan vastuu on keskitetty omaan palveluun.

Pääsynvalvonnan keskittäminen on perusteltua, koska silloin käyttäjien oikeuksiin liittyviä toimenpiteitä on helpompi tehdä. Esimerkiksi toimihenkilön siirtyessä rivijäseneksi, ei hänen käyttöoikeuksia tarvitse poistaa jokaisesta Kapsin palvelusta, vaan pelkästään keskitetystä pääsynvalvonnasta. Toistaiseksi ongelma ei ole kovin suuri, koska palveluita on vain kaksi, mutta varsinkin kun arkkitehtuuria viedään kohti kuvassa \ref{kapsi_uusi} esitettyä palveluperustaista arkkitehtuuria, tarve korostuu.

\subsection{Toteutuksen rajoitukset}
Kun Kapsin tietojärjestelmissä siirrytään keskitetyn tunnistautumispalvelun käyttöön, joudutaan tunnistautumista vaativat web-sovellukset muuttaa käyttämään uutta tunnistautumisjärjestelmää. Ohjelman toiminnan kannalta tällä ei saavuteta varsinaisesti mitään hyötyä, joten kyseessä on epätoivottu heijastusvaikutus (ripple effect). Heijastusvaikutuksella tarkoitetaan muutoksia, joita joudutaan tekemään ohjelmaan, joita alkuperäiset muutokset eivät suoraan koske \cite{arkkitehtuurit}. Toisaalta ratkaisulla vähennetään heijastusvaikutuksia tulevaisuudessa, kun web-sovelluksen ja tunnistautumisen välinen vahva riippuvuussuhde puretaan \cite{arkkitehtuurit}. Tällöin esimerkiksi kaksivaiheisen kirjautumisen toteuttaminen järjestelmään ei vaadi muutoksia kirjautumista käyttäviin web-sovelluksiin.

Esimerkkiarkkitehtuuri kasvattaa myös jonkun verran järjestelmän kompleksisuutta, kun käytössä olevien komponenttien määrä kasvaa. Myös tunnistautumispalvelun toteutuksessa käytetty tekniikka saattaa poiketa sitä käyttävien web-sovellusten tekniikasta, joka kasvattaa kompleksisuutta entisestään \cite{arkkitehtuurit}. Tällä hetkellä ongelmaa ei ole, sillä Kapsin järjestelmissä käytetään yleisesti Django-ohjelmistokehystä, jolla myös tunnistautumispalvelu toteutetaan. Jos jatkossa siirrytään toisiin web-ohjelmointikehyksiin, voi tunnistautumispalvelusta tulla perinnejärjestelmä, jonka muuttamiseen yhdistyksessä ei ole ammattitaitoa. Tämä on kuitenkin erittäin epätodennäköinen skenaario, koska web-palvelun ja tunnistautumispalvelun välinen rajapinta varmistaa, että tunnistautumispalvelu voidaan kirjoittaa helposti uusiksi, ilman heijastusvaikutuksia web-sovelluksiin, ennen kuin Django on muuttunut perinnetiedoksi.

\subsection{Soveltamisalueet}
Miten käyttäjädataa onko käsitelty ja käsitellään. Kehitys paikallisesti käytetyistä tiedostopohjaisista systeemeistä kohti tietokantoja ja asiaan räätälöihin palveluihin (LDAP). LDAP oleelisin, mutta tutkimuksen kannalta abstraktointi on tärkeä juttu.

Johdanto puoli sivua, alaluvut 0.5-1 sivu.

\subsubsection{Keskitetty tunnistautuminen Helsingin yliopiston tietojenkäsittelytieteen laitoksella}
Case-esimerkistä saadun kokemuksen perusteella keskitettyä tunnistautumisratkaisua voitaisiin hyödyntää myös Helsingin yliopiston tietojenkäsittelytieteen laitoksella. Tällä hetkellä laitoksen opiskelijoiden käytössä on erilaisia web-palveluita, kuten kurssi-ilmoittautuminen ja sähköposti. Lisäksi henkilökunnalla on palveluita mm. kurssisuoritusten ylläpitoa varten. Nämä palvelut voisivat olla yhteydessä keskitettyyn tunnistautumispalveluun, jolloin niiden ei tarvitsisi päästä suoraan käyttäjähallintaan.

Edellisen lisäksi keskitetty tunnistautumispalvelu mahdollistaisi muiden kuin laitoksen IT-osaston tuottamien web-palveluiden käytön laitoksen käyttäjätunnuksilla. Ohjelmistotuotantoprojekti-kurssin opiskelijat voisivat tuottaa web-palveluita laitoksen opiskelijoiden ja henkilökunnan käyttöön. Esimerkiksi keväällä 2010 toteutettu opiskelijoiden ainejärjestö TKO-älyn rekrytointipalvelu olisi voinut käyttää hyväkseen keskitettyä tunnistautumispalvelua, jolloin käytön rajaaminen vain laitoksen opiskelijoiden käyttöön olisi ollut mahdollista. Laitoksen LDAP-käyt\-tä\-jä\-hal\-lin\-nan käyttö tunnistautumiseen ei tullut kysymykseen, koska se olisi ollut tietoturvariski. Keskitettyä tunnistautumispalvelua käytettäessä riskiä ei ole, sillä web-palvelun tietokantaan tallennetaan vain yksilöivä käyttäjätunnus ja tunnistautuminen tehdään tunnistautumispalvelussa.

Myös keskitetystä pääsynvalvonnasta olisi hyötyä tietojenkäsittelytieteen laitoksen tapauksessa. Laitoksella on erilaisia ryhmiä, esimerkiksi kantahenkilökunta, tutkijat, sivutoimiset tuntiopettajat, vierailevat luennoitsijat, joilla on käytössään erilaisia palveluita. Tutkijat voivat käyttää Ukko-laskentaklusteria, sivutoimiset tuntiopettajat ja muu opetushenkilökunta kirjata arvosanoja kurssikirjanpitojärjestelmään jne. Näiden kaikkien palveluiden pääsynvalvonta olisi perusteltua toteuttaa keskitetysti, jolloin henkilön roolin muuttuessa hänen käyttöoikeuksien muuttaminen olisi helppoa. Keskitetyn tunnistautumispalvelun käyttöönoton jälkeen myös pääsynvalvonta voitaisiin keskittää luvussa 6.3.1 mainitulla tavalla.

\subsubsection{Keskitetty tunnistautuminen yrityksen intranetissä}
TODO: tähän on tosi vaikea löytää mitään viitteitä

Yritysten intranet-järjestelmät muistuttavat usein Kapsin arkkitehtuuria: järjestelmässä on erilaisia web-palveluita, joihin kirjaudutaan sisäisellä käyttäjätunnuksella. Keskitetyn tunnistautumispalvelun tuominen tällaiseen järjestelmään on perusteltua samoista syistä kuin Kapsin tapauksessa.

Usein yrityksissä ei toteuteta web-palveluita itse, vaan käytetään valmiita ratkaisuja. Tällöin keskitetyn tunnistautumispalvelun teknologiavalinnat täytyy tehdä ympäristön mukaan. Esimerkiksi jos intranet-palvelu on rakennettu Microsoftin SharePoint-teknologialla, on tunnistautumispalvelu toteutettavissa Microsoftin teknologialla \cite{sharepoint}. Myös muilla teknologioilla toteutettuihin intranet-järjestelmiin on tarjolla kaupallisia keskitettyjä tunnistautumisratkaisuja, esimerkiksi suomalaisen Ubisecuren CustomerID \cite{ubisecure}. [TODO: maksettu mainos? joku open source tilalle?]

Jos taas yrityksessä tuotetaan intranetin palveluita itse, voi oman tunnistautumispalvelun toteuttaminen Kapsin tapaan olla perusteltua. Usein omien palveluiden lisäksi järjestelmässä käytetään muiden tuottamia palveluita esimerkiksi sähköpostia varten. Tällöin täytyy selvittää teknologioiden yhteensopivuus valmiiden palveluiden kanssa, jolloin esimerkiksi SAML voi osoittautua OAuthia paremmaksi ratkaisuksi tunnistautumisen rajapintaprotokollaksi.

\subsubsection{Tunnistautuminen web-sovelluksissa ilman omaa käyttäjähallintaa}
Keskitetyn tunnistamisen periaatteita voidaan soveltaa myös ilman omaa käyttäjähallintaa. Tällöin web-sovellukseen kirjautuminen vaatii, että käyttäjällä on tunnus tunnistautumisen tarjoavan kolmannen osapuolen palveluun, esimerkiksi Facebookiin tai Googleen. Näillä ja monilla muilla palveluntarjoajilla on käytössään OAuth-pohjainen tunnistautumisrajapinta \cite{inside_the_identity_management_game}.

Ulkoistetun käyttäjähallinnan hyvänä puolena voidaan pitää sen vaivattomuutta. Tunnistamista tarjoavilla palveluilla on tapa varmistaa käyttäjän tiedot, joten web-sovelluksen ylläpitäjän ei tarvitse varmistaa niiden oikeellisuutta. Käyttäjät päivittävät tietonsa mielummin Facebookiin tms. kuin yksittäisiin web-palveluihin, joten tieto pysyy tuoreempana \cite{inside_the_identity_management_game}. Lisäksi web-sovelluksen ylläpitäjän pelko joutua salasanamurtojen kohteeksi vähenee, koska palveluun ei tallenneta salasanoja (tai edes niiden tiivisteitä).

Haittapuolena ulkoisesta tunnistamispalvelusta on, että käyttäjällä täytyy olla tunnus kolmannen osapuolen palveluun, jotta hän voi kirjautua web-sovellukseen. Vaikka Facebookilla ja Googlella on merkittävä määrä rekisteröityneitä käyttäjiä, jää niiden ulkopuolelle useita ihmisiä. Esimerkiksi Helsingin yliopisto ei voi edellyttää, että sen palveluiden käyttäjillä on tunnus Facebookiin tai Googleen, vaan palveluiden täytyy olla kaikkien opiskelijoiden käytössä. Tällöin oman käyttäjähallinnan rakentaminen ja ylläpito on tarpeen.

Ulkoinen käyttäjähallinta on kuitenkin hyvä ratkaisu monissa tapauksissa. Esimerkiksi monissa verkkolehdissä ja -blogeissa kommentointi on mahdollista vain Fa\-ce\-book-käyt\-tä\-jil\-le, jolloin keskusteluun voivat osallistua vain Facebookin varmentamat käyttäjät. Tällöin anonyymi kommentointi jää pois ja kommentointi tapahtuu vain ihmisten oikeilla nimillä. Ulkoisen käyttäjähallinnan käyttö lisää myös käyttäjien aktiivisuutta palvelussa: yksittäisen web-sovelluksen salasana unohtuu helposti, mutta yleisesti käytetyn sovelluksen salasana muistetaan paremmin \cite{password_habits}.