Tutkielman (erityisesti luvun 5) kriittinen arviointi, osattiinko esitellyllä arkkitehtuurilla vastata johdannossa esitettyihin kysymyksiin.

Vielä hieman auki, mutta pituus noin 5 sivua.

Tutkielmassa ei keskitytä kaikille avoimiin tunnistautumispalveluihin (kuten Facebook tai Google), vaan organisaatioihin, joilla on oma käyttäjähallinta. Esimerkki tällaisesta organisaatiosta on Helsingin yliopisto, jolla on oma Active Directory -käyttäjähallinta \cite{tietotekniikkaa}. Myös monet yritykset ylläpitävät omaa käyttäjähallintaa esimerkiksi LDAP-järjestelmässä tai keskitetyssä tietokannassa. Esimerkkinä hyötyjä ja haittoja punnittaessa käytetään Helsingin yliopistoa, mutta myös muunlaiset organisaatiot huomioidaan, mikäli se on perusteltua. TODO: tämä kappale uusiksi, kun 5-7 hahmottuu

\subsection{Esimerkkiarkkitehtuurin laajennettavuus}
Esitettyä esimerkkiarkkitehtuuria voidaan laajentaa tarvittaessa. Tässä luvussa esitetään miten valittua OAuth-protokollaa voidaan käyttää tunnistautumisen lisäksi myös pääsynvalvontaan. Myös kertakirjautumisen toteuttaminen OAuth-protokolla käydään läpi. Viimeisessä aliluvussa käydään läpi tunnettuja rajoituksia toteutetussa esimerkkiarkkitehtuurissa.

\subsubsection{Pääsynvalvonta}
Pääsynvalvonta ei ole kovin triviaali juttu. Kuitenkin jotain vakavamielistä pohdintaa, miten ratkaisua voisi laajentaa siihen suuntaan.

Käytännössä jonkunlainen pääsynvalvonta tulee, sillä auth.kapsi.fi tarkistaa onko käyttäjällä oikeus käyttää Sikteeriä tms ja palauttaa käyttäjän tiedot vain jos näin on.

\subsubsection{Kertakirjautuminen}
Kertakirjautumisella (SSO) tarkoitetaan arkkitehtuuria, jossa yhdellä kirjautumisella tunnistaudutaan useaan eri palveluun \cite{sso}. Käytetyissä kertakirjautumistekniikoissa käytetään identiteetintarjoajaa (identity provider), joka varmentaa kirjautumisen ja luo valtuutuksen web-palveluihin. Käytetyt tekniikat jaetaan kahteen osaan: näennäiseen- (pseudo SSO) ja tosikertakirjautumiseen (true SSO) \cite{sso}. Näennäisessä kertakirjautumisessa identiteetintarjoaja luo palveluun sopivan valtuutuksen, jolloin web-palvelu ei tiedä kertakirjautumisesta mitään. Tosikertakirjautumisessa taas käytetään yleisiä järjestelmänlaajuisia valtuutuksia, jolloin web-palveluiden täytyy olla tietoisia kertakirjautumisesta.

Kapsin nykyisessä arkkitehtuurissa käyttäjä voi kirjautua samalla tunnuksella eri palveluihin, mutta kyseessä ei ole kertakirjautuminen, koska kirjautumisen täytyy tehdä erikseen jokaiseen palveluun. Kertakirjautuminen on mahdollista tehdä myös nykyisessä arkkitehtuurissa, esimerkiksi tallentamalla tunnistetiedot selaimen evästeeseen (cookie). Toinen web-palvelu voi taas lukea tunnistetiedot evästeestä ja tunnistaa käyttäjän tätä kautta. Koska eri web-palvelut toimivat samassa domain-osoitteessa, pystyvät ne lukemaan toistensa kirjoittamia evästeitä \cite{rfc6265}. Tällainen ratkaisu on kuitenkin ongelmallinen, koska tunnistetiedot tallennetaan käyttäjän koneelle ja ne on luettavissa evästeestä. Jos taas tunnistetietojen kirjoittamiseen ja lukemiseen käytetään salausta, joudutaan salausavain jakamaan web-palveluiden kesken, joka lisää ylläpidon tarvetta.

Keskitettyä tunnistautumispalvelua käytettäessä kertakirjautuminen voidaan toteuttaa OAuth-protokollalla \cite{distributed_web_security}. Kontrollin kulku on samanlainen kuin kuvassa \ref{auth_kapsi_fi_flow}, mutta käyttäjän ei tarvitse syöttää tunnusta ja salasanaa, koska hänellä on jo voimassaoleva istunto tunnistautumispalveluun. Toisin sanoen käyttäjäagentti (selain) saa pääsyvaltuuden tunnistautumispalvelusta ilman uutta kirjaumista, jolloin kyseessä on näennäinen kertakirjautuminen. Kuitenkin ensimmäistä kertaa kirjautuessa web-sovellukseen käyttäjän täytyy hyväksyä tietojen haku tunnistautumispalvelusta. Kun tietojen luovutus on hyväksytty, voidaan jatkossa kirjautuminen tehdä ilman käyttäjän syötettä.


\subsection{Toteutuksen rajoitukset}
Kun Kapsin tietojärjestelmissä siirrytään keskitetyn tunnistautumispalvelun käyttöön, joudutaan tunnistautumista vaativat web-sovellukset muuttaa käyttämään uutta tunnistautumisjärjestelmää. Ohjelman toiminnan kannalta tällä ei saavuteta varsinaisesti mitään hyötyä, joten kyseessä on epätoivottu heijastusvaikutus (ripple effect). Heijastusvaikutuksella tarkoitetaan muutoksia, joita joudutaan tekemään ohjelmaan, joita alkuperäiset muutokset eivät suoraan koske \cite{arkkitehtuurit}. Toisaalta ratkaisulla vähennetään heijastusvaikutuksia tulevaisuudessa, kun web-sovelluksen ja tunnistautumisen välinen vahva riippuvuussuhde puretaan \cite{arkkitehtuurit}. Tällöin esimerkiksi kaksivaiheisen kirjautumisen toteuttaminen järjestelmään ei vaadi muutoksia kirjautumista käyttäviin web-sovelluksiin.

Tunnistautumispalvelu on erittäin kriittinen komponentti arkkitehtuurissa ja sen kaatuminen estää web-sovellusten täysipainoisen käytön, koska käyttäjien kirjautuminen ei ole mahdollista. Web-sovellukset tulevat siis riippuvaiseksi ulkoisesta (tunnistautumis)palvelusta, jolloin sovelluksen ylläpitäjällä ei ole mahdollisuutta vaikuttaa palvelunsa toimintaan. Vikatilanteessä hänen täytyy vain luottaa siihen, että ylläpito saa korjattua vian mahdollisimman pikaisesti.

Esimerkkiarkkitehtuuri kasvattaa myös järjestelmän kompleksisuutta, kun käytössä olevien komponenttien määrä kasvaa. Tunnistautumispalvelun toteutuksessa käytetty tekniikka saattaa poiketa sitä käyttävien web-sovellusten tekniikasta, joka kasvattaa kompleksisuutta entisestään \cite{arkkitehtuurit}. Tällä hetkellä ongelmaa ei ole, sillä Kapsin järjestelmissä käytetään yleisesti Django-ohjelmistokehystä, jolla myös tunnistautumispalvelu toteutetaan. Jos jatkossa siirrytään toisiin web-ohjelmointikehyksiin, voi tunnistautumispalvelusta tulla perinnejärjestelmä, jonka muuttamiseen yhdistyksessä ei ole ammattitaitoa. Tämä on kuitenkin erittäin epätodennäköinen skenaario, koska web-palvelun ja tunnistautumispalvelun välinen rajapinta varmistaa, että tunnistautumispalvelu voidaan kirjoittaa helposti uusiksi, ilman heijastusvaikutuksia web-sovelluksiin, ennen kuin Django on muuttunut perinnetiedoksi.

\subsection{Soveltamisalueet}
Johdantoa

Yliopiston lisäksi voisi ottaa myös toisen kohteen? Yritys, jolla intranet-palveluita?

\subsubsection{Keskitetty tunnistautuminen Helsingin yliopiston tietojenkäsittelytieteen laitoksella}
Case-esimerkistä saadun kokemuksen perusteella keskitettyä tunnistautumisratkaisua voitaisiin hyödyntää myös Helsingin yliopiston tietojenkäsittelytieteen laitoksella. Tällä hetkellä laitoksen opiskelijoiden käytössä on erilaisia web-sovelluksia, kuten kurssi-ilmoittautuminen ja sähköposti. Lisäksi henkilökunnalla on palveluita mm. kurssisuoritusten ylläpitoa varten. Nämä palvelut voisivat olla yhteydessä keskitettyyn tunnistautumispalveluun, jolloin niiden ei tarvitsisi päästä suoraan käyttäjähallintaan.

Edellisen lisäksi keskitetty tunnistautumispalvelu mahdollistaisi muiden kuin laitoksen IT-ryhmän tuottamien web-sovellusten käytön laitoksen käyttäjätunnuksilla. Ohjelmistotuotantoprojekti-kurssin opiskelijat voisivat tuottaa web-sovelluksia laitoksen opiskelijoiden ja henkilökunnan käyttöön. Esimerkiksi keväällä 2010 toteutettu opiskelijoiden ainejärjestö TKO-älyn rekrytointipalvelu olisi voinut käyttää hyväkseen keskitettyä tunnistautumispalvelua, jolloin käytön rajaaminen vain laitoksen opiskelijoiden käyttöön olisi ollut mahdollista. Laitoksen LDAP-käyt\-tä\-jä\-hal\-lin\-nan käyttö tunnistautumiseen ei tullut kysymykseen, koska siinä olisi ollut tietoturvariski. Keskitettyä tunnistautumispalvelua käytettäessä riskiä ei olisi, sillä web-sovelluksen tietokantaan tallennetaan vain yksilöivä käyttäjätunnus ja tunnistautuminen tehdään tunnistautumispalvelussa.

Myös keskitetystä pääsynvalvonnasta olisi hyötyä tietojenkäsittelytieteen laitoksen tapauksessa. Laitoksella on erilaisia ryhmiä, esimerkiksi kantahenkilökunta, tutkijat, sivutoimiset tuntiopettajat ja vierailevat luennoitsijat, joilla on käytössään erilaisia palveluita. Tutkijat voivat käyttää Ukko-laskentaklusteria, sivutoimiset tuntiopettajat ja muu opetushenkilökunta kirjata arvosanoja kurssikirjanpitojärjestelmään jne. Näiden kaikkien palveluiden pääsynvalvonta olisi perusteltua toteuttaa keskitetysti, jolloin henkilön roolin muuttuessa hänen käyttöoikeuksiensa muuttaminen olisi helppoa. Keskitetyn tunnistautumispalvelun käyttöönoton jälkeen myös pääsynvalvonta voitaisiin keskittää luvussa 6.3.1 kuvatulla tavalla.