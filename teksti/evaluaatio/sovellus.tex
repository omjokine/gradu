Kapsin lisäksi arkkitehtuuria voidaan hyödyntää muissa samantyyppisissä järjestelmissä. Hyviä sovellutusympäristöjä ovat yritysten tai yhteisöjen web-palvelut, joissa on käytössä omat käyttäjätunnukset. Esimerkkejä tällaisista ympäristöistä on Helsingin yliopiston tietojenkäsittelytieteen laitos ja yrityksen intranet. Arkkitehtuurin soveltamista näissä ympäristöissä esitellään tämän luvun kahdessa ensimmäisessä aliluvussa.

Kolmannessa tilanteessa arkkitehtuuria sovelletaan web-palveluihin ilman omaa käyttäjänhallintaa. Tällöin web-palvelun ylläpitäjä luopuu käyttäjien hallinnasta ja tunnistautumisen edellytys on, että käyttäjällä on tunnus jossain kolmannen osapuolen palvelussa, esimerkiksi Facebookissa tai Googlessa. Tämän ratkaisun etuja ja haittoja käsitellään viimeisessä aliluvussa.

\subsubsection{Keskitetty tunnistautuminen Helsingin yliopiston tietojenkäsittelytieteen laitoksella}
Case-esimerkistä saadun kokemuksen perusteella keskitettyä tunnistautumisratkaisua voitaisiin hyödyntää myös Helsingin yliopiston tietojenkäsittelytieteen laitoksella. Tällä hetkellä laitoksen opiskelijoiden käytössä on erilaisia web-sovelluksia, kuten kurssi-ilmoittautuminen ja sähköposti. Lisäksi henkilökunnalla on palveluita mm. kurssisuoritusten ylläpitoa varten. Nämä palvelut voisivat olla yhteydessä keskitettyyn tunnistautumispalveluun, jolloin niiden ei tarvitsisi päästä suoraan käyttäjähallintaan.

Edellisen lisäksi keskitetty tunnistautumispalvelu mahdollistaisi muiden kuin laitoksen IT-ryhmän tuottamien web-sovellusten käytön laitoksen käyttäjätunnuksilla. Ohjelmistotuotantoprojekti-kurssin opiskelijat voisivat tuottaa web-sovelluksia laitoksen opiskelijoiden ja henkilökunnan käyttöön. Esimerkiksi keväällä 2010 toteutettu opiskelijoiden ainejärjestö TKO-älyn rekrytointipalvelu olisi voinut käyttää hyväkseen keskitettyä tunnistautumispalvelua, jolloin käytön rajaaminen vain laitoksen opiskelijoiden käyttöön olisi ollut mahdollista. Laitoksen LDAP-käyt\-tä\-jä\-hal\-lin\-nan käyttö tunnistautumiseen ei tullut kysymykseen, koska siinä olisi ollut tietoturvariski. Keskitettyä tunnistautumispalvelua käytettäessä riskiä ei olisi, sillä web-sovelluksen tietokantaan tallennetaan vain yksilöivä käyttäjätunnus ja tunnistautuminen tehdään tunnistautumispalvelussa.

Myös keskitetystä pääsynvalvonnasta olisi hyötyä tietojenkäsittelytieteen laitoksen tapauksessa. Laitoksella on erilaisia ryhmiä, esimerkiksi kantahenkilökunta, tutkijat, sivutoimiset tuntiopettajat ja vierailevat luennoitsijat, joilla on käytössään erilaisia palveluita. Tutkijat voivat käyttää Ukko-laskentaklusteria, sivutoimiset tuntiopettajat ja muu opetushenkilökunta kirjata arvosanoja kurssikirjanpitojärjestelmään jne. Näiden kaikkien palveluiden pääsynvalvonta olisi perusteltua toteuttaa keskitetysti, jolloin henkilön roolin muuttuessa hänen käyttöoikeuksiensa muuttaminen olisi helppoa. Keskitetyn tunnistautumispalvelun käyttöönoton jälkeen myös pääsynvalvonta voitaisiin keskittää luvussa 6.3.1 kuvatulla tavalla.

\subsubsection{Keskitetty tunnistautuminen yrityksen intranetissä}
Yritysten intranet-järjestelmät muistuttavat usein Kapsin arkkitehtuuria: järjestelmässä on erilaisia web-palveluita, joihin kirjaudutaan sisäisellä käyttäjätunnuksella. Keskitetyn tunnistautumispalvelun tuominen tällaiseen järjestelmään on perusteltua samoista syistä kuin Kapsin tapauksessa. Kapsin järjestelmä on hyvin yhtenäinen teknologialtaan (Python-kieli), mutta yritykset eivät usein tuota palveluita itse, vaan intranet-järjestelmissä käytetään valmiita eri toimittajien tuotteita.

Oman räätälöidyn keskitetyn tunnistautumispalvelun tuottaminen ei ole yrityksessä välttämättä mahdollista, mutta erilaisia valmiita ratkaisuja on olemassa. Näiden ratkaisujen kohdalla teknologiavalinnat täytyy tehdä ympäristön mukaan. Esimerkiksi jos intranet-palvelu on rakennettu Microsoftin SharePoint-teknologialla, on tunnistautumispalvelu toteutettavissa Microsoftin teknologialla \cite{sharepoint}. Myös muilla tekniikoilla toteutetuille intranet-järjestelmille on saatavilla oma keskitetty tunnistautumispalvelu.

Jos taas yrityksessä tuotetaan intranetin palveluita itse, voi oman tunnistautumispalvelun toteuttaminen Kapsin tapaan olla perusteltua. Usein omien palveluiden lisäksi järjestelmässä käytetään muiden tuottamia palveluita esimerkiksi sähköpostia varten. Tällöin täytyy selvittää teknologioiden yhteensopivuus valmiiden palveluiden kanssa, jolloin esimerkiksi SAML voi osoittautua OAuthia paremmaksi ratkaisuksi tunnistautumisen rajapintaprotokollaksi. Teknisesti on myös mahdollista käyttää eri protokollia, kuten OAuth ja SAML, rinnakkain tunnistautumispalvelussa.

\subsubsection{Tunnistautuminen web-palveluissa ilman omaa käyttäjähallintaa}
\input{evaluaatio/facebook}