Esitetyssä arkkitehtuurissa OAuth-protokollaa käytetään tunnistautumisen toteuttamiseen, mutta sen käyttöä voidaan laajentaa myös muuhun pääsynvalvontaan \cite{distributed_web_security}. Käyttö on varsin perusteltua, sillä OAuth on nimenomaan pääsynvalvontaprotokolla ja tunnistamisen toteuttaminen sitä käyttäen on erikoiskäyttötapaus. OAuthia käyttäen tunnistautumispalvelu voidaan laajentaa suojaamaan Kapsin järjestelmissä käytettäviä resursseja.

Jäsenen tiedot voivat olla yksi resurssi, joiden käyttöä valvotaan auth.kapsi.fi-pal\-ve\-lun kautta. Tällöin Sikteeri saa käyttäjätietokannasta tiedot vain, jos käyttäjällä olisi esittää pääsyvaltuutus, joka oikeuttaa tietojen hakemiseen. Näin Sikteeri ei olisi suoraan yhteydessä käyttäjätietokantaan, vaan arkkitehtuuriin lisätään palvelu, joka mahdollistaa käyttäjän hakemisen, lisäämisen, muokkaamisen tai poistamisen. Tämä palvelu on yhteydessä LDAP-käyttäjätietokantaan ja se antaa lukea tietoja vain jos käyttäjällä on esittää oikea pääsyvaltuutus. Pääsyvaltuuksia hallinnoi tunnistautumispalvelu.

Arkkitehtuurin laajentaminen koskemaan myös pääsynvalvontaa veisi sitä entistä enemmän kohti palveluperusteista arkkitehtuuria, jossa jokaisella komponentilla on yksi tehtävä, jonka ne suorittavat hyvin \cite{soa}. Sikteeri olisi siinä tapauksessa käyttöliittymä komponentille, joka hallinnoi käyttäjiä. Päätös käyttäjän oikeudesta hallinnoida käyttäjiä ei ole Sikteerin vastuulla vaan vastuu on keskitetty omaan palveluun.

Pääsynvalvonnan keskittäminen on perusteltua, koska silloin käyttäjien oikeuksiin liittyviä toimenpiteitä on helpompi tehdä. Esimerkiksi toimihenkilön siirtyessä rivijäseneksi, ei hänen käyttöoikeuksia tarvitse poistaa jokaisesta Kapsin palvelusta, vaan pelkästään keskitetystä pääsynvalvonnasta. Toistaiseksi ongelma ei ole kovin suuri, koska palveluita on vain kaksi, mutta varsinkin kun arkkitehtuuria viedään kohti kuvassa \ref{kapsi_uusi} esitettyä palveluperustaista arkkitehtuuria, tarve korostuu.