Esitetyssä arkkitehtuurissa OAuth-protokollaa käytetään tunnistautumisen toteuttamiseen, mutta sen käyttöä voidaan laajentaa myös muuhun pääsynvalvontaan \cite{distributed_web_security}. Käyttö on varsin perusteltua, sillä OAuth on nimenomaan pääsynvalvontaprotokolla ja tunnistamisen toteuttaminen sitä käyttäen on erikoiskäyttötapaus.

OAuthia käyttäen tunnistautumispalvelu voidaan laajentaa suojaamaan Kapsin järjestelmissä käytettäviä resursseja. Jäsenen tiedot voivat olla eräs resurssi, jonka käyttöä valvotaan erillisen pal\-ve\-lun kautta. Tällöin Sikteeri saisi käyttäjätietokannasta tiedot vain, jos käyttäjällä on esittää pääsyvaltuutus, joka oikeuttaa tietojen hakemiseen. Näin Sikteeri ei olisi suoraan yhteydessä käyttäjätietokantaan, vaan arkkitehtuuriin lisättäisiin palvelu, joka mahdollistaa käyttäjän hakemisen, lisäämisen, muokkaamisen tai poistamisen. Tämä palvelu olisi yhteydessä LDAP-käyttäjätietokantaan ja se antaisi lukea tietoja vain, jos käyttäjällä on esittää oikea pääsyvaltuutus. Pääsyvaltuuksia hallinnoisi tunnistautumispalvelu.

Arkkitehtuurin laajentaminen koskemaan myös pääsynvalvontaa veisi sitä entistä enemmän kohti palveluperusteista arkkitehtuuria, jossa jokaisella komponentilla on yksi tehtävä, jonka ne suorittavat hyvin \cite{soa}. Sikteeri olisi siinä tapauksessa käyttöliittymä komponentille, joka hallinnoi käyttäjiä. Päätös käyttäjän oikeudesta hallinnoida käyttäjiä ei olisi Sikteerin vastuulla, vaan vastuu olisi keskitetty omaan palveluun.

Pääsynvalvonnan keskittäminen on perusteltua, koska silloin käyttäjien oikeuksiin liittyviä toimenpiteitä on helpompi tehdä. Esimerkiksi toimihenkilön siirtyessä rivijäseneksi, ei hänen käyttöoikeuksiaan tarvitse poistaa jokaisesta Kapsin palvelusta, vaan vain keskitetystä pääsynvalvonnasta. Toistaiseksi ongelma ei ole kovin suuri, koska palveluita on vain kaksi, mutta kun arkkitehtuuria viedään kohti kuvassa \ref{kapsi_uusi} esitettyä palveluperustaista arkkitehtuuria, tarve korostuu.