Kertakirjautumisella (SSO) tarkoitetaan arkkitehtuuria, jossa yhdellä kirjautumisella tunnistaudutaan useaan eri palveluun \cite{sso}. Käytetyissä kertakirjautumistekniikoissa käytetään identiteetintarjoajaa (identity provider), joka varmentaa kirjautumisen ja luo valtuutuksen web-palveluihin. Käytetyt tekniikat jaetaan kahteen osaan: näennäiseen- (pseudo SSO) ja tosikertakirjautumiseen (true SSO) \cite{sso}. Näennäisessä kertakirjautumisessa identiteetintarjoaja luo palveluun sopivan valtuutuksen, jolloin web-palvelu ei tiedä kertakirjautumisesta mitään. Tosikertakirjautumisessa taas käytetään yleisiä järjestelmänlaajuisia valtuutuksia, jolloin web-palveluiden täytyy olla tietoisia kertakirjautumisesta.

Kapsin nykyisessä arkkitehtuurissa käyttäjä voi kirjautua samalla ja tunnuksella eri palveluihin, mutta kyseessä ei ole kertakirjautuminen, koska kirjautumisen joutuu tekemään erikseen jokaiseen palveluun. Kertakirjautuminen on mahdollista tehdä myös nykyisessä arkkitehtuurissa, esimerkiksi tallentamalla tunnistetiedot selaimen evästeeseen (cookie). Toinen web-palvelu voi taas lukea tunnistetiedot evästeestä ja tunnistaa käyttäjän tätä kautta. Koska eri web-palvelut toimivat samassa domain-osoitteessa, pystyvät ne lukemaan toistensa kirjoittamia evästeitä [TODO: lähde]. Tällainen ratkaisu on kuitenkin ongelmallinen, koska tunnistetiedot tallennetaan käyttäjän koneelle ja ne on luettavissa evästeestä. Jos taas tunnistetietojen kirjoittamiseen ja lukemiseen käytetään salausta, joudutaan salausavain jakamaan web-palveluiden kesken, joka lisää ylläpidon tarvetta [TODO: lähde].

Sen sijaan ehdotetussa arkkitehtuurissa kertakirjautuminen voidaan toteuttaa OAuth-protokollan avulla [TODO: lähde]. OAuthissa x, y ja z.