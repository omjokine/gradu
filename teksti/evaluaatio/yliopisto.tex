Case-esimerkistä saadun kokemuksen perusteella keskitettyä tunnistautumisratkaisua voitaisiin hyödyntää myös Helsingin yliopiston tietojenkäsittelytieteen laitoksella. Tällä hetkellä laitoksen opiskelijoiden käytössä on erilaisia web-palveluita, kuten kurssi-ilmoittautuminen ja webmail. Lisäksi henkilökunnalla on palveluita mm. kurssisuoritusten ylläpitoa varten. Nämä palvelut voisivat olla yhteydessä keskitettyyn tunnistautumispalveluun, jolloin niiden ei tarvitsisi päästä suoraan käyttäjähallintaan.

Edellistä merkittävämpänä etuna keskitetty tunnistautumispalvelu mahdollistaisi muiden, kuin laitoksen IT-osaston, tuottamien web-palveluiden käyttämisen laitoksen käyttäjätunnuksilla. Tällöin esimerkiksi ohjelmistotuotantoprojekti-kurssin opiskelijat voisivat tuottaa web-palveluita laitoksen opiskelijoiden ja henkilökunnan käyttöön. Esimerkiksi keväällä 2010 toteutettu opiskelijoiden ainejärjestö TKO-älyn rekrytointipalvelu olisi voinut käyttää hyväkseen keskitettyä tunnistautumispalvelua, jolloin käytön rajaaminen vain laitoksen opiskelijoiden käyttöön olisi ollut mahdollista. Laitoksen LDAP-käyttäjähallinnan käyttö tunnistautumiseen ei tullut kysymykseen, koska se olisi ollut tietoturvariski. Keskitettyä tunnistautumispalvelua käytettäessä riskiä ei ole, sillä web-palvelun tietokantaan tallennetaan vain yksilöivä käyttäjätunnus ja tunnistautuminen tehdään tunnistautumispalvelussa.

Pääsynvalvontaprotokollia käytettäessä web-palveluiden arkkitehtuuria voisi viedä kohti palvelusuuntautuneita arkkitehtuureita. joobajoo