Kun Kapsin tietojärjestelmissä siirrytään keskitetyn tunnistautumispalvelun käyttöön, joudutaan tunnistautumista vaativat web-sovellukset muuttaa käyttämään uutta tunnistautumisjärjestelmää. Ohjelman toiminnan kannalta tällä ei saavuteta varsinaisesti mitään hyötyä, joten kyseessä on epätoivottu heijastusvaikutus (ripple effect) \cite{arkkitehtuurit}. Heijastusvaikutuksella tarkoitetaan muutoksia, joita joudutaan tekemään ohjelmaan, joita alkuperäiset muutokset eivät suoraan koske. Toisaalta ratkaisulla vähennetään heijastusvaikutuksia tulevaisuudessa, kun web-sovelluksen ja tunnistautumisen välinen vahva riippuvuussuhde puretaan. Tällöin esimerkiksi kaksivaiheisen kirjautumisen toteuttaminen järjestelmään ei vaadi muutoksia kirjautumista käyttäviin web-sovelluksiin.

Tunnistautumispalvelu on erittäin kriittinen komponentti arkkitehtuurissa ja sen kaatuminen estää web-sovellusten täysipainoisen käytön, koska käyttäjien kirjautuminen ei ole mahdollista. Web-sovellukset tulevat siis riippuvaiseksi ulkoisesta (tunnistautumis)palvelusta, jolloin sovelluksen ylläpitäjällä ei ole mahdollisuutta vaikuttaa palvelunsa toimintaan. Vian sattuessa hänen täytyy vain luottaa siihen, että ylläpito saa korjattua vian mahdollisimman pikaisesti.

Esimerkkiarkkitehtuuri kasvattaa myös järjestelmän kompleksisuutta, kun käytössä olevien komponenttien määrä kasvaa. Tunnistautumispalvelun toteutuksessa käytetty tekniikka saattaa poiketa sitä käyttävien web-sovellusten tekniikasta, joka kasvattaa kompleksisuutta entisestään \cite{arkkitehtuurit}. Tällä hetkellä ongelmaa ei ole, sillä Kapsin järjestelmissä käytetään yleisesti Django-ohjelmistokehystä, jolla myös tunnistautumispalvelu toteutetaan. Jos Kapsissa siirrytään käyttämään muita web-oh\-jel\-moin\-ti\-ke\-hyk\-siä, voi tunnistautumispalvelusta tulla perinnejärjestelmä, jonka muuttamiseen yhdistyksessä ei ole ammattitaitoa. Web-sovellusten ja tunnistautumispalvelun välinen rajapinta kuitenkin varmistaa, että tunnistautumispalvelu on mahdollista kirjoittaa uudestaan ilman heijastusvaikutuksia web-sovelluksiin. Tällöin tunnistautumispalvelu voidaan toteuttaa uudestaan, vaikka Django olisi muuttunut jo perinnetiedoksi.

Tunnistautumispalvelun ja web-sovellusten välissä käytettävä OAuth-protokolla tuo myös omat riskinsä. Protokollan versiosta 1.0 löydettiin huhtikuussa 2009 vakava tietoturva-aukko, jonka avulla hyökkääjä pystyy kaappaamaan käyttäjän istunnon \cite{oauth_primer}. Ongelma korjattiin versiossa 1.0a, mutta vastaavien aukkojen löytyminen on mahdollista. On jopa esitetty, että OAuthin arkkitehtuuri on väärin suunniteltu ja siinä on sisäänrakennettuja tietoturvaongelmia, joita ei voida ratkaista ilman protokollan perusteiden muuttamista \cite{oauth_lol}. Suurten Internet-yhtiöiden (Facebook, Twitter yms) osallistuminen protokollan kehitykseen vaikuttaa siihen, että tietoturvaongelmat saadaan ratkaistua, mutta Kapsin ylläpidon on syytä tarkkailla protokollan kehitystä.

OAuth voi aiheuttaa yhteensopivuusongelmia järjestelmään käyttöönotettavien valmiiden web-sovellusten kanssa. Esimerkiksi jäsenistölle tarjottava sähköpostisovellus ei välttämättä tue OAuth-protokollaa, mutta onneksi arkkitehtuuri mahdollistaa muiden protokollien (kuten SAML) käyttämisen OAuthin rinnalla. Rinnakkaisten protokollien lisääminen nostaa järjestelmän kompleksisuutta, mutta mahdollistaa yhä useampien web-sovellusten lisäämisen osaksi järjestelmää.