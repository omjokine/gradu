Kun Kapsin tietojärjestelmissä siirrytään keskitetyn tunnistautumispalvelun käyttöön, joudutaan tunnistautumista vaativat web-sovellukset muuttaa käyttämään uutta tunnistautumisjärjestelmää. Ohjelman toiminnan kannalta tällä ei saavuteta varsinaisesti mitään hyötyä, joten kyseessä on epätoivottu heijastusvaikutus (ripple effect). Heijastusvaikutuksella tarkoitetaan muutoksia, joita joudutaan tekemään ohjelmaan, joita alkuperäiset muutokset eivät suoraan koske \cite{arkkitehtuurit}. Toisaalta ratkaisulla vähennetään heijastusvaikutuksia tulevaisuudessa, kun web-sovelluksen ja tunnistautumisen välinen vahva riippuvuussuhde puretaan \cite{arkkitehtuurit}. Tällöin esimerkiksi kaksivaiheisen kirjautumisen toteuttaminen järjestelmään ei vaadi muutoksia kirjautumista käyttäviin web-sovelluksiin.

Tunnistautumispalvelu on erittäin kriittinen komponentti arkkitehtuurissa ja sen kaatuminen estää web-sovellusten täysipainoisen käytön, koska käyttäjien kirjautuminen ei ole mahdollista. Web-sovellukset tulevat siis riippuvaiseksi ulkoisesta (tunnistautumis)palvelusta, jolloin sovelluksen ylläpitäjällä ei ole mahdollisuutta vaikuttaa palvelunsa toimintaan. Vikatilanteessä hänen täytyy vain luottaa siihen, että ylläpito saa korjattua vian mahdollisimman pikaisesti.

Esimerkkiarkkitehtuuri kasvattaa myös järjestelmän kompleksisuutta, kun käytössä olevien komponenttien määrä kasvaa. Tunnistautumispalvelun toteutuksessa käytetty tekniikka saattaa poiketa sitä käyttävien web-sovellusten tekniikasta, joka kasvattaa kompleksisuutta entisestään \cite{arkkitehtuurit}. Tällä hetkellä ongelmaa ei ole, sillä Kapsin järjestelmissä käytetään yleisesti Django-ohjelmistokehystä, jolla myös tunnistautumispalvelu toteutetaan. Jos jatkossa siirrytään toisiin web-ohjelmointikehyksiin, voi tunnistautumispalvelusta tulla perinnejärjestelmä, jonka muuttamiseen yhdistyksessä ei ole ammattitaitoa. Tämä on kuitenkin erittäin epätodennäköinen skenaario, koska web-palvelun ja tunnistautumispalvelun välinen rajapinta varmistaa, että tunnistautumispalvelu voidaan kirjoittaa helposti uusiksi, ilman heijastusvaikutuksia web-sovelluksiin, ennen kuin Django on muuttunut perinnetiedoksi.