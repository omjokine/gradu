Kapsin lisäksi arkkitehtuuria voidaan hyödyntää muissa samantyyppisissä järjestelmissä. Hyviä sovellutusympäristöjä ovat yritysten tai yhteisöjen web-palvelut, joissa on käytössä omat käyttäjätunnukset. Esimerkkejä tällaisista ympäristöistä on Helsingin yliopiston tietojenkäsittelytieteen laitos tai yrityksen intranet. Arkkitehtuurin soveltamista näissä ympäristöissä esitellään tämän luvun kahdessa ensimmäisessä aliluvussa.

Kolmannessa tilanteessa arkkitehtuuria sovelletaan web-palveluihin ilman omaa käyttäjänhallintaa. Tällöin web-palvelun ylläpitäjä luopuu käyttäjien hallinnasta ja tunnistamisen edellytys on, että käyttäjällä on tunnus jossain kolmannen osapuolen palvelussa, esimerkiksi Facebookissa tai Googlessa. Tämän ratkaisun etuja ja haittoja käsitellään viimeisessä aliluvussa.