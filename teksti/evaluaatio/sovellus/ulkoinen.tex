TODO: sopivia lähteitä tähän

Keskitetyn tunnistamisen periaatteita voidaan soveltaa myös ilman omaa käyttäjähallintaa. Tällöin web-sovellukseen kirjautuminen vaatii, että käyttäjällä on tunnus tunnistautumisen tarjoavan kolmannen osapuolen palveluun, esimerkiksi Facebookiin tai Googleen. Molemmilla on käytössään OAuth-pohjainen tunnistautumisrajapinta [].

Ulkoistetun käyttäjähallinnan hyvänä puolena voidaan pitää sen vaivattomuutta. Tunnistamista tarjoavilla palveluilla on yleensä tapa varmistaa käyttäjän tiedot, joten web-sovelluksen ylläpitäjän ei tarvitse varmistaa niiden oikeellisuutta. Käyttäjät päivittävät tietonsa helpommin Facebookiin tms kuin yksittäisiin web-palveluihin, joten tieto pysyy myös tuoreempana []. Lisäksi web-sivun ylläpitäjän pelko joutua kiusallisten tietomurtouutisten keskiöön vähenee, koska palveluun ei tallenneta salasanoja (tai edes niiden tiivisteitä). [TODO: wat?]

Haittapuolena ulkoisesta tunnistamispalvelusta on se, että käyttäjällä täytyy olla tunnus kolmannen osapuolen palveluun, jotta hän voi kirjautua web-sovellukseen []. Vaikka Facebookilla ja Googlella on merkittävä määrä rekisteröityneitä käyttäjiä, jää niiden ulkopuolelle useita ihmisiä []. Esimerkiksi Helsingin yliopiston palveluiden käytön edellytys ei voi olla Google-tunnus. Tällöin oman käyttäjähallinnan rakentaminen ja ylläpito on tarpeen.