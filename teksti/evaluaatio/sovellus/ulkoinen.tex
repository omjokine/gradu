Keskitetyn tunnistamisen periaatteita voidaan soveltaa myös ilman omaa käyttäjähallintaa. Tällöin web-sovellukseen kirjautuminen vaatii, että käyttäjällä on tunnus tunnistautumisen tarjoavan kolmannen osapuolen palveluun, esimerkiksi Facebookiin tai Googleen. Näillä ja monilla muilla palveluntarjoajilla on käytössään OAuth-pohjainen tunnistautumisrajapinta \cite{inside_the_identity_management_game}.

Ulkoistetun käyttäjähallinnan hyvänä puolena voidaan pitää sen vaivattomuutta. Tunnistamista tarjoavilla palveluilla on tapa varmistaa käyttäjän tiedot, joten web-sovelluksen ylläpitäjän ei tarvitse varmistaa niiden oikeellisuutta. Käyttäjät päivittävät tietonsa mielummin Facebookiin tms. kuin yksittäisiin web-sovelluksiin, joten tieto pysyy tuoreempana \cite{inside_the_identity_management_game}. Lisäksi web-sovelluksen ylläpitäjän pelko joutua salasanamurtojen kohteeksi vähenee, koska palveluun ei tallenneta salasanoja (tai edes niiden tiivisteitä).

Haittapuolena ulkoisesta tunnistamispalvelusta on, että käyttäjällä täytyy olla tunnus kolmannen osapuolen palveluun, jotta hän voi kirjautua web-sovellukseen. Vaikka Facebookilla ja Googlella on merkittävä määrä rekisteröityneitä käyttäjiä, jää niiden ulkopuolelle useita ihmisiä. Esimerkiksi Helsingin yliopisto ei voi edellyttää, että sen palveluiden käyttäjillä on tunnus Facebookiin tai Googleen, vaan palveluiden täytyy olla kaikkien opiskelijoiden käytössä. Tällöin oman käyttäjähallinnan rakentaminen ja ylläpito on tarpeen.

Ulkoinen käyttäjähallinta on kuitenkin hyvä ratkaisu monissa tapauksissa. Esimerkiksi monissa verkkolehdissä ja -blogeissa kommentointi on mahdollista vain Fa\-ce\-book-käyt\-tä\-jil\-le, jolloin keskusteluun voivat osallistua vain Facebookin varmentamat käyttäjät. Tällöin anonyymi kommentointi jää pois ja kommentointi tapahtuu vain ihmisten oikeilla nimillä. Ulkoisen käyttäjähallinnan käyttö lisää myös käyttäjien aktiivisuutta palvelussa: yksittäisen web-sovelluksen salasana unohtuu helposti, mutta yleisesti käytetyn sovelluksen salasana muistetaan paremmin \cite{password_habits}.