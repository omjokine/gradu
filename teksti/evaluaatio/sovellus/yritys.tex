Yritysten intranet-järjestelmät muistuttavat usein Kapsin arkkitehtuuria: järjestelmässä on erilaisia web-sovelluksia, joihin kirjaudutaan yrityksen intranetin sisäisellä käyttäjätunnuksella. Keskitetyn tunnistautumispalvelun tuominen tällaiseen järjestelmään on perusteltua samoista syistä kuin Kapsin tapauksessa.

Oman räätälöidyn keskitetyn tunnistautumispalvelun tuottaminen ei ole yrityksessä välttämättä mahdollista, mutta erilaisia valmiita ratkaisuja on olemassa. Näiden ratkaisujen kohdalla teknologiavalinnat täytyy tehdä ympäristön mukaan. Esimerkiksi jos intranet-palvelu on rakennettu Microsoftin SharePoint-teknologialla, on tunnistautumispalvelu toteutettavissa Microsoftin Active Directory -teknologialla \cite{sharepoint}. Myös muilla tekniikoilla toteutetuille intranet-järjestelmille on saatavilla oma keskitetty tunnistautumispalvelu.

Jos taas yrityksessä tuotetaan intranetin palveluita itse, voi oman tunnistautumispalvelun toteuttaminen Kapsin tapaan olla perusteltua. Usein omien palveluiden lisäksi järjestelmässä käytetään muiden tuottamia palveluita esimerkiksi sähköpostia varten. Tällöin täytyy selvittää teknologioiden yhteensopivuus valmiiden palveluiden kanssa, jolloin esimerkiksi SAML voi osoittautua OAuthia paremmaksi ratkaisuksi tunnistautumisen rajapintaprotokollaksi. Teknisesti on myös mahdollista käyttää eri protokollia kuten OAuth ja SAML rinnakkain tunnistautumispalvelussa.