Case-esimerkistä saadun kokemuksen perusteella keskitettyä tunnistautumisratkaisua voitaisiin hyödyntää myös Helsingin yliopiston tietojenkäsittelytieteen laitoksella. Tällä hetkellä laitoksen opiskelijoiden käytössä on erilaisia web-palveluita, kuten kurssi-ilmoittautuminen ja sähköposti. Lisäksi henkilökunnalla on palveluita mm. kurssisuoritusten ylläpitoa varten. Nämä palvelut voisivat olla yhteydessä keskitettyyn tunnistautumispalveluun, jolloin niiden ei tarvitsisi päästä suoraan käyttäjähallintaan.

Edellisen lisäksi keskitetty tunnistautumispalvelu mahdollistaisi muiden kuin laitoksen IT-osaston tuottamien web-palveluiden käytön laitoksen käyttäjätunnuksilla. Ohjelmistotuotantoprojekti-kurssin opiskelijat voisivat tuottaa web-palveluita laitoksen opiskelijoiden ja henkilökunnan käyttöön. Esimerkiksi keväällä 2010 toteutettu opiskelijoiden ainejärjestö TKO-älyn rekrytointipalvelu olisi voinut käyttää hyväkseen keskitettyä tunnistautumispalvelua, jolloin käytön rajaaminen vain laitoksen opiskelijoiden käyttöön olisi ollut mahdollista. Laitoksen LDAP-käyt\-tä\-jä\-hal\-lin\-nan käyttö tunnistautumiseen ei tullut kysymykseen, koska se olisi ollut tietoturvariski. Keskitettyä tunnistautumispalvelua käytettäessä riskiä ei ole, sillä web-palvelun tietokantaan tallennetaan vain yksilöivä käyttäjätunnus ja tunnistautuminen tehdään tunnistautumispalvelussa.

Myös keskitetystä pääsynvalvonnasta olisi hyötyä tietojenkäsittelytieteen laitoksen tapauksessa. Laitoksella on erilaisia ryhmiä, esimerkiksi kantahenkilökunta, tutkijat, sivutoimiset tuntiopettajat, vierailevat luennoitsijat, joilla on käytössään erilaisia palveluita. Tutkijat voivat käyttää Ukko-laskentaklusteria, sivutoimiset tuntiopettajat ja muu opetushenkilökunta kirjata arvosanoja kurssikirjanpitojärjestelmään jne. Näiden kaikkien palveluiden pääsynvalvonta olisi perusteltua toteuttaa keskitetysti, jolloin henkilön roolin muuttuessa hänen käyttöoikeuksien muuttaminen olisi helppoa. Keskitetyn tunnistautumispalvelun käyttöönoton jälkeen myös pääsynvalvonta voitaisiin keskittää luvussa 6.3.1 mainitulla tavalla.