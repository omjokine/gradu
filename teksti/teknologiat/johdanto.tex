Keskitettyyn tunnistautumispalvelun toteuttamiseen voidaan käyttää verkkotunnistautumiseen kehitettyjä protokollia. Niitä hyödyntämällä käyttäjän tunnistetietoja (esim. käyttäjänimi ja salasana) ei tarvitse siirtää verkon yli ja web-palvelut voivat ulkoistaa tunnistautumisen erillisen palvelun tehtäväksi \cite{nisti}. Tällöin käyttäjä voi samalla identiteetillä tunnistautua useaan eri palveluun ilman, että hänen tarvitsee muistaa palvelukohtaisia tunnus- ja salasanayhdistelmiä \cite{open_identity}.

Tässä luvussa keskitytään keskitetyssä tunnistautumispalvelussa käytettyihin teknologioihin. Teknologiat ovat samoja, jotka ovat tällä hetkellä käytössä Internetin web-palveluissa \cite{facebook}. Ensimmäisessä alaluvussa käydään läpi yleisiä periaatteita koskien keskitettyä tunnistautumista. Toisessa alaluvussa tutustutaan tunnistautumisprotokolliin. Ensin käydään läpi protokollien yleisiä toimintaperiaatteita, minkä jälkeen esitellään tunnistautumisprotokollaksi soveltuvia standardeja. Tämän jälkeen tutkitaan SAML- ja OAuth-protokollien soveltuvuutta tunnistautumispalvelun toteuttamiseen.