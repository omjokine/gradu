Verkkotunnistautumiseen kehitettyjä protokollia hyödyntämällä käyttäjän tunnistetietoja (esim. käyttäjänimi ja salasana) ei tarvitse siirtää verkon yli ja web-palvelut voivat ulkoistaa tunnistautumisen erillisen palvelun tehtäväksi. Tällöin käyttäjä voi samalla identiteetillä tunnistautumia useaan eri palveluun, ilman että hänen tarvitsee muistaa palvelukohtaisia tunnus ja salasanayhdistelmiä.

Tämän luvun ensimmäisessä alaluvussa käydään läpi yleisiä periaatteita koskien keskitettyä tunnistautumista. Toisessa alaluvussa tutustutaan tunnistautumisprotokolliin liittyviin termeihin, joita käytetään yleisesti eri protokollissa. Kolmannessa alaluvussa esitellään rajapintaprotokollien toimintaperiaatteita ja tutkitaan SAML- ja OAuth-protokollien soveltuvuutta prototyyppiin. Viimeinen luku on yhteenvetoluku.