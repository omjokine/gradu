Tunnistautumiseen liittyvien käsitteiden ymmärtäminen ennen protokollien yksityiskohtiin perehtymistä auttaa tunnistautumiseen liittyvien periaatteiden hahmottamista. Käsitteet ovat yleisluontoisia eivätkä kosketa vain tiettyjä protokollia. Protokollien yhteydessä käytetään käsitteitä asiakasohjelma (client), tunnistautumispalvelu (identity provider), suojattu resurssi (protected resource), valtuutustieto (credentials), valtuutusavain (authorization code) ja pääsyvaltuutus (access token) \cite{nisti}.

Asiakasohjelmalla tarkoitetaan web-palvelun käyttäjän pääteohjelmaa, jolla hän kirjautuu web-palveluun käyttäen keskitettyä tunnistautumispalvelua. Käytännössä asiakasohjelma on web-palvelun tapauksessa käyttäjän WWW-selain, joka pystyy tekemään uudelleenohjauksia sivustolta toiselle. Uudelleenohjaus on HTTP-pro\-to\-kol\-lan perustoiminnallisuutta, joten mikä tahansa HTTP/1.1-standardin WWW-selain käy asiakasohjelmaksi \cite{rfc2616}.

Tunnistautumispalvelu on web-palvelu, johon käyttäjä ohjataan tekemään tunnistautuminen. Onnistuneen tunnistautumisen jälkeen tunnistautumispalvelu ohjaa asi\-a\-kas\-oh\-jel\-man takaisin tunnistautumista pyytäneen palvelun määrittelemään osoitteeseen \cite{nisti}. Avoimen Internetin puolella tunnistautumispalvelu voi olla esimerkiksi Facebook tai LinkedIn.

Tunnistautumisprotokollien yhteydessä suojatulla resurssilla tarkoitetaan resurssia, jonka käyttö vaatii tunnistautumisen ja käyttöoikeuden. Yleisessä tapauksessa suojatulla resurssilla tarkoitetaan yksittäistä resurssia (esim. käyttäjän omistamaa valokuvaa), johon halutaan asettaa pääsyrajoituksia \cite{nisti}. Tässä tutkielmassa suojatulla resurssilla tarkoitetaan tunnistautumista vaativaa web-palvelua. TODO: käyttäjä on myös resurssi, ainakin oAuthissa???

Valtuutustieto koostuu yksilöivästä tunnisteesta ja siihen liittyvästä salaisesta avaimesta. Tässä tutkielmassa valtuutustiedolla tarkoitetaan käyttäjän tunnusta ja salasanaa.

Kirjauduttuaan sisään tunnistautumispalvelimelle käyttäjä saa valtuutusavaimen, jonka hän lähettää eteenpäin suojatun resurssin omistajalle. Valtuutusavain ei pidä sisällään käyttäjän valtuutustietoja, vaan ainoastaan tunnistautumispalvelin osaa lukea sen \cite{nisti}. Saatuaan valtuutusavaimen käyttäjältä voi suojatun resurssin omistaja hakea pääsyvaltuuden käyttäjän tietoihin tunnistautumispalvelusta.

Pääsyvaltuutus on tunnistautumispalvelimelta saatava yksilöivä tunniste, jonka avulla suojatun resurssin omistaja voi pyytää käyttäjän tiedot tunnistautumispalvelulta. Pääsyvaltuutus on voimassa tietyn ajan, jonka jälkeen se täytyy uusia tunnistautumispalvelimella \cite{nisti}. Pääsyvaltuutusta voidaan käyttää myös tunnistautumispalvelusta erillään olevien resurssien valtuuttamiseen. Esimerkiksi web-sovellus voi hakea tunnistautumispalvelulta pääsyvaltuuden, jolla se hakee valokuvia valokuvien jakopalvelusta \cite{facebook}.