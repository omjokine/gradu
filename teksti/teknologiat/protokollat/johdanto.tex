Monet tahot tutkivat ja kehittävät tunnistautumis- ja pääsynvalvontaprotokollia. Web Services -teknologioita standardoinut OASIS (Organization for the Advancement of Structured Information Standards) on kehittänyt XML-pohjaista SAML-kieltä tunnistautumisprotokollaksi \cite{saml_spec}. SAML tarjoaa tunnistautumisprotokollan lisäksi joukon muita web-sovellukseen liittyviä tur\-val\-li\-suus\-stan\-dar\-de\-ja \cite{next_saml}.

Myös Microsoft on kehittänyt oman Windows Live ID -standardin, joka tarjoaa tunnistautumisprotokollan \cite{open_identity}. Windows Live ID on osittain suljettu standardi, joka tarjoaa tunnistautumisprotokollan lisäksi täydellisen keskitetyn käyttäjän identiteetin hallinnan \cite{open_identity}. Suljetun lähdekoodin vuoksi Windows Live ID ei ole tämän tutkielman kannalta kiinnostava protokolla.

Avoimen lähdekoodin yhteisössä on syntynyt OpenID, jota mm. Google käyttää tunnistautumisprotokollanaan \cite{open_identity}. OpenID:stä on kehittynyt OAuth-pro\-to\-kol\-la, joka on tarkoitettu nimenomaan pääsynhallintaan, mutta jolla voidaan toteuttaa myös tunnistautuminen \cite{formal_oauth}.

SAML, OpenID ja OAuth ovat tämän tutkielman kannalta kiinnostavia teknologioita, sillä ne ovat avoimia ja niiden kehitystyö on aktiivista \cite{facebook}. OpenID:n, OAuthin ja SAML:n ominaisuuksia ja soveltuvuutta keskitetyn tunnistautumispalvelun rajapintaprotokollaksi tarkastellaan seuraavissa aliluvuissa.