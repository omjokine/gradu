OAuth on avoin tunnistautumisrajapinta hajautetuille web-sovelluksille. Se mahdollistaa käyttäjien resurssien jakamisen palveluiden välillä ilman käyttäjätunnuksen tai salasanan luovuttamista kolmannelle osapuolelle. Se perustuu erilaisten valtuutusavainten (token) välittämiseen palveluiden välillä \cite{oauth2_0}. OAuth on yleisesti käytössä web-sovelluksissa, joissa halutaan näyttää käyttäjälle kuuluvia resursseja (esimerkiksi valokuvia), jotka sijaitsevat toisessa sovelluksessa \cite{web_resources}.

Alun perin OAuthin kehitystyö alkoi marraskuussa 2006 kehitettäessä Twitter-palveluun OpenID-tukea. OAuth on määritelty RFC-dokumentissa numero 5849. Sen ensimmäinen versio (1.0) julkaistiin lokakuussa 2007 ja päivitetty versio (1.0a) kesäkuussa 2009 \cite{oauth2_0}. OAuthin versio 2.0 on myös kehitteillä ja se on tarkoitus julkaista marraskuussa 2012 \cite{oauth2_0}.

Tähän vielä n. sivu-kaksi tekstiä, niin että selviää mitä OpenID ja OAuth on ja mikä niiden suhde toisiinsa on.