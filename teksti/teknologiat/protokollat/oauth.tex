OAuth on avoin tunnistautumisrajapinta hajautetuille web-sovelluksille. Se mahdollistaa käyttäjien resurssien jakamisen palveluiden välillä ilman käyttäjätunnuksen tai salasanan luovuttamista kolmannelle osapuolelle. Se perustuu erilaisten valtuutusavainten (token) välittämiseen palveluiden välillä. OAuth on yleisesti käytössä web-sovelluksissa, joissa halutaan näyttää käyttäjälle kuuluvia resursseja (esimerkiksi valokuvia), jotka sijaitsevat toisessa sovelluksessa [TODO: lähde].

OAuth on määritelty RFC-dokumentissa numero 5849. Sen ensimmäinen versio (1.0) julkaistiin lokakuussa 2007 ja päivitetty versio (1.0a) kesäkuussa 2009 [TODO: viite RFC-dokkariin]. OAuthin versio 2.0 on myös kehitteillä ja se on tarkoitus julkaista marraskuussa 2012 [TODO: viite Oauth 2.0 draft].

Alunperin OAuthin kehitystyö alkoi marraskuussa 2006, kun Blaine Cook kehitty Twitter-palveluun OpenID-tukea.

... tarvitaanko tätä?

\begin{description}
  \item[Asiakas] - HTTP-asiakas, joka tekee OAuth-tunnistaumiskutsuja.

  \item[Palvelin] - HTTP-palvelin, joka ottaa vastaan OAuth-tunnistautumiskutsuja.

  \item[Suojattu resurssi] - Resurssi, joka voidaan saada palvelimelta, jos pyyntö on OAuth-tunnistettu.

  \item[Resurssin omistaja] - Omistaa suojatun resurssin ja valvoo siihen pääsyä.

  \item[Valtuutustieto (credentials)] - Valtuutustieto koostuu yksilöivästä tunnisteesta ja siihen liittyvästä salaisesta avaimesta. OAuthissa käytetään kolmen tasoisia valtuutusavaimia: asiakasavaimia, jotka yksilöi ja varmistaa asiakasohjelmiston, väliaikaisia avaimia, joilla autorisoidaan pyyntö ja valtuutusavaimia, joilla pyyntö hyväksytään.

  \item[Valtuutusavain (token)] - Palvelimelta saatava yksilöivä tunniste, jonka avulla asiakas voi pyytää resurssin omistajalta suojattua resurssia. TODO: yeah?
\end{description}

OAuth 2.0:n on mahdollista tarjota Kerberos-yhteensopivuus. Kerberos taas ei tarjoa OAuth-yhteensopivuutta, joten OAuth > Kerberos.

http://tools.ietf.org/html/draft-hardjono-oauth-kerberos-01
