Security Assertion Markup Language (SAML) on OASIS-komitean kehittämä XML-pohjainen avoin standardi tunnistautumiseen ja pääsynhallintaan \cite{saml_spec}. Standardin versio 1.0 julkaistiin marraskuussa 2002 ja versio 2.0 maaliskuussa 2005. Version 2.0:n viimeisin korjattu versio 5 julkaistiin helmikuussa 2012.

SAML määrittelee XML-pohjaiset työkalut tunnistautumisen ja pääsynhallinnan toteuttamiseen. Varsinainen toteutus, esimerkiksi se, mitä tietoja siirretään ja millä tavalla, jätetään SAML:ssä toteuttajan päätettäväksi \cite{dynamic_saml}. Varsinaiset SAML-viestit voivat kulkea esimerkiksi synkronisesti SOAP- ja HTTP-protokollalla. SAML soveltuu avoimena ja XML-pohjaisena protokollana käytettäväksi Web Services -standardilla toteutetuissa web-sovelluksissa.

SAML on vakiintunut standardi ja siihen on määritelty erilaisia laajennoksia, joiden ansiosta samalla standardilla voidaan toteuttaa koko identiteentinhallinta \cite{saml_spec}. Tästä syystä se on vakiintunut standardina erityisesti yritysten sisäisissä kertakirjautumisratkaisuissa (Single Sign On) \cite{dynamic_saml}. Kuitenkaan niin kutsutun julkisen Internetin puolella SAML ei ole saavuttanut merkittävää asemaa, vaan yritykset kuten Google ja Facebook ovat toteuttaneet omat tunnistautumisrajapintansa kevyemmillä protokollilla, kuten OpenID:llä ja OAuthilla.