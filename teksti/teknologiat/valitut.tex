Tutkielmassa esiteltävän esimerkkiarkkitehtuurin tunnistautumisprotokollana käytetään OAuthia. Tärkeimpänä syynä on sen suosio julkisen Internetin palveluissa ja sitä kautta syntynyt vahva kehittäjäyhteisö. Kun OpenID:n jatkokehitys on epävarma tulevan OpenID Connect -standardin myötä, näyttää OAuthin tulevaisuus varmalta. Toisaalta tiedostetaan myös, että OAuth 2.0:n määritys ei ole vielä lopullinen ja valinnasta saattaa seurata jonkun verran ylläpitotyötä, kun määrityksen lopullinen versio valmistuu.

SAML:in verrattuna OAuth on tarkemmin rajattu protokolla. OAuth määrittelee vain tarvittavan toiminnallisuuden, eikä arkkitehtuurissa ole tarvetta SAML:n kokonaisvaltaisille turvallisuusmäärityksille. Arkkitehtuuri tulee käyttöön ympäristössä, jossa käytetään paljon erilaisilla web-sovelluskehyksillä (Django, Ruby on Rails) toteutettuja sovelluksia. Facebookin yms. toimijoiden käyttäessä OAuthia on myös kaikille käytetyille web-sovelluskehyksille saatavilla tuki OAuth-tunnistautumiselle.

OAuthin käyttöä puoltaa myös mahdollisuus laajentaa arkkitehtuuria toteuttamalla keskitetty pääsynvalvonta. OAuth on pääsynvalvontaprotokolla, joten sen ominaisuuksia voidaan käyttää myöhemmin myös pääsynvalvonnasta. Tällöin järjestelmän resursseja voidaan suojata keskitetyn palvelun kautta ja toteuttaa arkkitehtuuriin esimerkiksi roolipohjainen pääsynvalvonta. OAuthia käyttävä esimerkkiarkkitehtuuri on esitelty seuraavassa luvussa.