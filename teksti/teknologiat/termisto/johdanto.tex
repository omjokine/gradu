Ennen tarkkaa tutustumista tunnistamisen teknologioihin on hyvä käydä läpi muutamia aiheeseen liittyviä termejä. Termit ovat yleisluontoisia ja ne toistuvat eri protokollien kohdalla.

Ehkei tarvita alalukuja kaikista, mutta selitettävä on ainakin (suomi-englanti):

valtuutusavain - authorization code
--> koodi, jonka käyttäjä saa käteen, kun on kirjautunut palveluun. Lähettää koodin web-palvelulle, joka hakee sillä access tokenin tunnistautumispalvelusta

pääsyvaltuutus - access token
--> valtuutus, jolla käyttäjä pääsee hakemaan suojattua resurssia jostain, vanhenee joskus jne.

tunnistautumispalvelu - authorization server
--> mitä tässä nyt rakennetaan. vehje, joka jakaa valtuutusavaimia ja pääsykoodeja

valtuutustieto - credentials
--> käyttäjätunnus ja salasana

asiakas - client
--> WWW-selain (user agent)