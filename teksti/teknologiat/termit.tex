Tunnistautumiseen liittyvien käsitteiden ymmärtäminen ennen protokollien yksityiskohtiin perehtymistä auttaa niiden periaatteiden hahmottamista. Käsitteet ovat yleisluontoisia eivätkä kosketa vain tiettyä protokollaa. Protokollien yhteydessä käytetään käsitteitä asiakasohjelma (client), tunnistautumispalvelu (identity provider), suojattu resurssi (protected resource), valtuutustieto (credentials), valtuutusavain (authorization code) ja pääsyvaltuutus (access token) \cite{nisti}.

Asiakasohjelmalla tarkoitetaan web-sovelluksen käyttäjän pääteohjelmaa, jolla hän kirjautuu web-sovellukseen käyttäen keskitettyä tunnistautumispalvelua \cite{nisti}. Käytännössä asiakasohjelma on web-sovelluksen tapauksessa käyttäjän WWW-se\-lain, joka pystyy tekemään uudelleenohjauksia sivustolta toiselle. Uudelleenohjaus on osa HTTP-pro\-to\-kol\-lan perustoiminnallisuutta, joten mikä tahansa HTTP/1.1-stan\-dar\-din WWW-\-se\-lain sopii asiakasohjelmaksi \cite{rfc2616}.

Tunnistautumispalvelu on web-sovellus, johon käyttäjä ohjataan tekemään tunnistautuminen. Tunnistautumisen jälkeen tunnistautumispalvelu ohjaa asia\-kas\-oh\-jel\-man tunnistamista pyytäneen sovelluksen määrittelemään osoit\-tee\-seen \cite{nisti}. Avoimen Internetin puolella tunnistautumispalvelu voi olla esimerkiksi Facebook tai LinkedIn.

Tunnistautumisprotokollien, kuten OAuth, yhteydessä suojatulla resurssilla tarkoitetaan resurssia, jonka käyttö vaatii tunnistautumisen ja käyttöoikeuden \cite{oauth2_0}. Yleisesti suojatulla resurssilla tarkoitetaan yksittäistä resurssia (esim. käyttäjän omistamaa valokuvaa), johon halutaan asettaa pääsyrajoituksia \cite{nisti}. Keskitetyn tunnistautumispalvelun yhteydessä käyttäjän tiedot voivat olla suojattu resurssi.

Valtuutustieto koostuu yksilöivästä tunnisteesta ja siihen liittyvästä salaisesta avaimesta. Yleensä valtuutustiedolla tarkoitetaan käyttäjän tunnusta ja salasanaa.

Kirjauduttuaan sisään tunnistautumispalvelunn käyttäjä saa valtuutusavaimen, jonka hän lähettää eteenpäin suojatun resurssin omistajalle. Valtuutusavaimeen on kirjattu tieto käyttäjästä ja avaimen voimassaoloaika. Valtuutusavain ei pidä sisällään käyttäjän valtuutustietoja. Ainoastaan tunnistautumispalvelin osaa lukea sen \cite{nisti}. Saatuaan valtuutusavaimen käyttäjältä voi suojatun resurssin omistaja hakea pääsyvaltuuden suojattuun resurssiin.

Pääsyvaltuutus on tunnistautumispalvelimelta saatava yksilöivä tunniste, jonka avulla suojatun resurssin omistaja saa resurssin palvelimelta. Pääsyvaltuutus on voimassa tietyn ajan, jonka jälkeen se täytyy uusia tunnistautumispalvelimella \cite{nisti}. Pääsyvaltuutusta käytetään myös tunnistautumispalvelusta erillään olevien resurssien valtuuttamiseen. Esimerkiksi web-sovellus voi hakea tunnistautumispalvelulta pääsyvaltuuden, jolla se hakee valokuvia kuvien jakopalvelusta \cite{facebook}.