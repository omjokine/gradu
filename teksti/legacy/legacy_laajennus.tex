Mitä rajoituksia toteutuksella on? Kuinka voisi laajentaa? Kertakirjautuminen ja autorisointi hyviä suuntia. SSO aika peruskamaa OAuthin kanssa, autorisoinnista sen sijaan voi saada ihan mielenkiintoista pohdintaa aikaan.

LDAP:n hyödyntäminen autorisoinnissa, voisiko esimerkiksi käyttäjäryhmät olla tallennettu jotenkin kätevästi LDAP:in ja niiden perusteella voidaan autorisoida pääsy resurssiin.

Sinällään pääsynhallinta tiettyihin palveluihin on jonkin tason autorisointia ja kuuluu ehdottomasti gradun aihepiiriin (toteutus tunnistaa x:n palvelun käyttäjiä samaa käyttäjädataa vasten, kaikille käyttäjille ei saa antaa oikeutta kaikkiin palveluihin).

Pituus yhteensä pari-kolme sivua. Ei sinällään haittaa, vaikka menisi tuostakin yli, tämä on kuitenkin mielenkiintoista ja oleellista asiaa.

Miten käyttäjädataa onko käsitelty ja käsitellään. Kehitys paikallisesti käytetyistä tiedostopohjaisista systeemeistä kohti tietokantoja ja asiaan räätälöihin palveluihin (LDAP). LDAP oleelisin, mutta tutkimuksen kannalta abstraktointi on tärkeä juttu.

Johdanto puoli sivua, alaluvut 0.5-1 sivu.
\subsection{Kertakirjautuminen}
Kertakirjautumisella (SSO) tarkoitetaan arkkitehtuuria, jossa yhdellä kirjautumisella tunnistaudutaan useaan eri palveluun \cite{sso}. Käytetyissä kertakirjautumistekniikoissa käytetään identiteetintarjoajaa (identity provider), joka varmentaa kirjautumisen ja luo valtuutuksen web-sovellukseen. Käytetyt tekniikat jaetaan kahteen osaan: näennäiseen- (pseudo SSO) ja tosikertakirjautumiseen (true SSO) \cite{sso}. Näennäisessä kertakirjautumisessa identiteetintarjoaja luo palveluun sopivan valtuutuksen, jolloin web-sovellus ei tiedä kertakirjautumisesta mitään. Tosikertakirjautumisessa taas käytetään yleisiä järjestelmänlaajuisia valtuutuksia, jolloin web-sovellusten täytyy olla tietoisia kertakirjautumisesta.

Kapsin nykyisessä arkkitehtuurissa käyttäjä voi kirjautua samalla tunnuksella eri palveluihin, mutta kyseessä ei ole kertakirjautuminen, koska kirjautumisen täytyy tehdä erikseen jokaiseen palveluun. Kertakirjautuminen on mahdollista tehdä myös nykyisessä arkkitehtuurissa, esimerkiksi tallentamalla tunnistetiedot selaimen evästeeseen (cookie). Toinen web-sovellus voi taas lukea tunnistetiedot evästeestä ja tunnistaa käyttäjän tätä kautta. Koska eri web-sovellukset toimivat samassa domain-osoitteessa, ne voivat lukea toistensa kirjoittamia evästeitä \cite{rfc6265}. Tällainen ratkaisu on kuitenkin ongelmallinen, koska tunnistetiedot tallennetaan käyttäjän koneelle ja ne on luettavissa evästeestä. Jos taas tunnistetietojen kirjoittamiseen ja lukemiseen käytetään salausta, joudutaan salausavain jakamaan web-sovellusten kesken. Tämä lisää ylläpidon tarvetta.

Keskitettyä tunnistautumispalvelua käytettäessä kertakirjautuminen voidaan toteuttaa OAuth-protokollalla \cite{distributed_web_security}. Kontrollin kulku on samanlainen kuin kuvassa \ref{auth_kapsi_fi_flow}, mutta käyttäjän ei tarvitse syöttää tunnusta ja salasanaa, koska hänellä on jo voimassaoleva istunto tunnistautumispalveluun. Toisin sanoen käyttäjäagentti (selain) saa pääsyvaltuuden tunnistautumispalvelusta ilman uutta kirjautumista, jolloin kyseessä on näennäinen kertakirjautuminen. Kuitenkin ensimmäistä kertaa kirjautuessa web-sovellukseen käyttäjän täytyy hyväksyä tietojen haku tunnistautumispalvelusta. Kun tietojen luovutus on hyväksytty, voidaan jatkossa kirjautuminen tehdä ilman käyttäjän syötettä.
\subsection{Pääsynvalvonta}
Esitetyssä arkkitehtuurissa OAuth-protokollaa käytetään tunnistautumisen toteuttamiseen, mutta sen käyttöä voidaan laajentaa myös muuhun pääsynvalvontaan \cite{distributed_web_security}. Käyttö on varsin perusteltua, sillä OAuth on nimenomaan pääsynvalvontaprotokolla ja tunnistamisen toteuttaminen sitä käyttäen on erikoiskäyttötapaus. OAuthia käyttäen tunnistautumispalvelu voidaan laajentaa suojaamaan Kapsin järjestelmissä käytettäviä resursseja.

Jäsenen tiedot voivat olla yksi resurssi, joiden käyttöä valvotaan auth.kapsi.fi-pal\-ve\-lun kautta. Tällöin Sikteeri saa käyttäjätietokannasta tiedot vain, jos käyttäjällä olisi esittää pääsyvaltuutus, joka oikeuttaa tietojen hakemiseen. Näin Sikteeri ei olisi suoraan yhteydessä käyttäjätietokantaan, vaan arkkitehtuuriin lisätään palvelu, joka mahdollistaa käyttäjän hakemisen, lisäämisen, muokkaamisen tai poistamisen. Tämä palvelu on yhteydessä LDAP-käyttäjätietokantaan ja se antaa lukea tietoja vain jos käyttäjällä on esittää oikea pääsyvaltuutus. Pääsyvaltuuksia hallinnoi tunnistautumispalvelu.

Arkkitehtuurin laajentaminen koskemaan myös pääsynvalvontaa veisi sitä entistä enemmän kohti palveluperusteista arkkitehtuuria, jossa jokaisella komponentilla on yksi tehtävä, jonka ne suorittavat hyvin \cite{soa}. Sikteeri olisi siinä tapauksessa käyttöliittymä komponentille, joka hallinnoi käyttäjiä. Päätös käyttäjän oikeudesta hallinnoida käyttäjiä ei ole Sikteerin vastuulla vaan vastuu on keskitetty omaan palveluun.

Pääsynvalvonnan keskittäminen on perusteltua, koska silloin käyttäjien oikeuksiin liittyviä toimenpiteitä on helpompi tehdä. Esimerkiksi toimihenkilön siirtyessä rivijäseneksi, ei hänen käyttöoikeuksia tarvitse poistaa jokaisesta Kapsin palvelusta, vaan pelkästään keskitetystä pääsynvalvonnasta. Toistaiseksi ongelma ei ole kovin suuri, koska palveluita on vain kaksi, mutta varsinkin kun arkkitehtuuria viedään kohti kuvassa \ref{kapsi_uusi} esitettyä palveluperustaista arkkitehtuuria, tarve korostuu.
