Web-sovellukset ovat kokeneet muutoksen jokaisen sivulatauksen yhteydessä ladattavista CGI-ohjelmista jatkuvasti käynnissäoleviksi sovelluksiksi, joiden sovelluskehys pitää huolta perustoiminallisuudesta. Sovelluskehyksen otettua vastuun perusasioista, voivat sovellukset keskittyä entistä enemmän tietyn kokonaisuuden hoitamiseen mahdollisimman hyvin. Tämä muutos on vienyt web-sovelluskehitystä kohti palvelusuuntauneita arkkitehtuureita, joissa yksittäiset autonomiset web-sovellukset tarjoavat web services -rajapinnan, joita muut web-sovellukset tai esimerkiksi JavaScript-pohjaiset asiakasohjelmat käyttävät.

Yksittäiset web-palvelut vaativat joissakin tapauksissa käyttäjän tunnistautumisen, jotta ne voivat palvella kyseistä käyttäjää. Koska web-palveluiden on tarkoitus olla riippumattomia ympäröivästä maailmasta, on yleiskäyttöisten ratkaisujen löytäminen tunnistautumisongelmaan tärkeää. Seuraavassa luvussa paneudutaan syvemmin tunnistautumiseen web-palveluissa ja myöhemmin tutkielmassa luodaan ja evaluoidaan prototyyppi web-sovelluksesta, joka hakee dataa palveluista, jotka osaavat myös tunnistaa käyttäjän.