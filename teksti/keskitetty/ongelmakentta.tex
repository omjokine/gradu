Palvelusuuntautuneissa web-ohjelmistoissa käyttäjien tunnistautuminen on toteutettu monin tavoin. Tyypillisesti kysytään käyttäjältä tunnus ja salasana, joita verrataan ohjelmiston paikalliseen käyttäjätietokantaan. Paikallinen tietokanta on kopioitu palvelun varsinaisesta käyttäjätietokannasta ja paikallisiin tietokantoihin on käyttäjille luotu erillinen käyttäjätunnus.

Tästä seuraa monenlaisia synkronointiongelmia. Esimerkiksi työntekijän irtisanoutuessa joudutaan tunnus poistamaan kaikista tietokannoista erikseen. Myös osoitteen yms. tietojen muutokset täytyy päivittää kaikkiin tietokantoihin. Lisäksi käyttäjälle syntyy saman järjestelmän sisällä monia tunnuksia, joihin saattaa liittyä erilliset salasanat. Käyttäjän kannalta on myöskin ikävää kirjautua jokaiseen osapalveluun erikseen.
