Lähteet:\\
- inside the identity management game \cite{inside_the_identity_management_game}\\
- Decentralization: The Future of Online Social Networking \cite{decentralisations}

Tyypillisesti web-palvelun toimintakenttä on Internet, jossa palvelut toimivat itsenäisesti. Näiden palveluiden välinen integraatio on kasvussa ja palveluiden kesken halutaan jakaa tietoa, jolloin niiden täytyy pystyä identifioimaan käyttäjä keskenään. Yleisen identeettitarjoajan rakentaminen Internettiin on tutkimuksen alla ja OpenID ja mitä näitä nyt on. 

Usein ei ole tarpeen tehdä palveluista julkisia, vaan käyttöoikeus niihin voidaan rajata tietylle osajoukolle kaikista Internetin käyttäjistä. Tällaisia osajoukkoja voi olla esimerkiksi yrityksen työntekijät, joilla on pääsy intranet-palveluihin tai tietyn sivuston käyttäjät, joilla on pääsy sivuston palveluihin. Tällöin voi olla järkevää eriyttää käyttäjähallinta omaksi palveluksi ja keskittää osapalveluiden tunnistautuminen siihen. Tämän tutkielman pääpaino on tunnetulle osajoukolle, esimerkiksi yrityksen työntekijöille, suunnatuissa palveluissa.

Tutkielmassa pyritään selvittämään kuinka yritys voi rakentaa keskitetyn tunnistautumispalvelun valmiin käyttäjädatan päälle. Lähtökohtaisesti tunnistautumista vaativat palvelut ovat web-pohjaisia, mutta myös työasemalla tai puhelimella käytettävät asiakasohjelmat pyritään ottamaan huomioon.
