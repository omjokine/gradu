Web-palvelujen määrä on kasvanut reilusti viime vuosina ja tyypillisesti kaikissa järjestelmissä on oma käyttäjähallintansa \cite{billion_keys}. Tästä seuraa käyttäjän tietojen monistaminen moneen eri järjestelmään ja tyypillisesti käyttäjä kirjautuu näihin järjestelmiin eri identiteetillä. Vuonna 2007 julkaistun tutkimuksen mukaan web-käyttäjällä on n. 25 erillistä verkkoidentiteettiä, joista hän käyttää päivittäin kahdeksaa \cite{password_habits}.

Keskitetyn tunnistautumisen tarkoituksena on tarjota palvelu, jota vasten käyttäjä voidaan tunnistaa erillisestä web-palvelusta ilman uuden identiteetin luontia. Tunnistautumispalvelun ja sitä käyttävien web-palveluiden välillä on luottamussuhde, jolloin erilliseen web-palveluun ei tarvitse luoda omaa käyttäjätietokantaa, vaan se voi luottaa tunnistautumispalvelun tunnistamiin käyttäjiin.

Tunnistautumispalvelulla voidaan parantaa järjestelmien tietoturvaa ja parantaa tunnistautumisen luotettavuutta. Käyttäjät valitsevat vahvempia salasanoja palveluihin, jotka he kokeavat tärkeiksi, kuten sähköpostiin, verrattuna vähemmän tärkeisiin web-palveluihin \cite{password_habits}. Keskitetyssä palvelussa käyttäjän tunnistautumisen luotettavuutta voidaan parantaa esimerkiksi vaatimalla normaalia web-palvelua vahvempia salasanoja. Käyttäjien todennusta voidaan vahvistaa myös lisävarmistuksilla, kuten erilaisilla tunnuslukulistoilla tai puhelimen kautta tehtävällä todennuksella.

Toisaalta keskitetyn tunnistautumiseen käytetyn salasanan kalastelu käy houkuttelevammaksi, koska sen avulla pääsee käyttäjän tietoihin käsiksi useaan palveluun tai jopa luomaan käyttäjän identiteetillä tunnuksia uusin palveluihin. Palveluun kohdistuu myös normaalia web-palvelua suuremmat odotukset tietoturvalle, joten käytettyjen protokollien yms täytyy olla luotettavia.
