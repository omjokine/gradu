Web-palvelujen määrä on kasvanut reilusti viime vuosina ja tyypillisesti kaikissa järjestelmissä on oma käyttäjähallintansa \cite{billion_keys}. Tästä seuraa käyttäjän tietojen monistaminen moneen eri järjestelmään ja tyypillisesti käyttäjä kirjautuu näihin järjestelmiin eri identiteetillä. Vuonna 2007 julkaistun tutkimuksen mukaan web-käyttäjällä on n. 25 erillistä verkkoidentiteettiä, joista hän käyttää päivittäin kahdeksaa \cite{password_habits}.

Toisaalta keskitetyn tunnistautumiseen käytetyn salasanan kalastelu käy houkuttelevammaksi, koska sen avulla pääsee käyttäjän tietoihin käsiksi useaan palveluun tai jopa luomaan käyttäjän identiteetillä tunnuksia uusin palveluihin. Palveluun kohdistuu myös normaalia web-palvelua suuremmat odotukset tietoturvalle, joten käytettyjen protokollien täytyy olla luotettavia.
