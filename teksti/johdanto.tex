Yrityksen toteuttamissa palvelusuuntauneissa web-ohjelmistoissa käyttäjien tunnistautuminen on toteutettu monella eri tavalla. Yrityksellä on tarkoituksena löytää yhteneväinen tunnistautumismenetelmä, jolla eri ohjelmistot voisivat käyttää yrityksen keskitettyä LDAP-palvelinta, jossa käyttäjien tietoja säilytetään. LDAP-palvelimeen talletettujen tietojen perusteella käyttäjälle voidaan joko tarjota tai estää pääsy johonkin web-palveluista.

Yritys käyttää ohjelmistoissa pääsääntöisesti Ruby on Rails -ohjelmistokehystä, mutta tarkoituksena on rakentaa kieliriippumaton järjestelmä, jolloin tunnistauminen voidaan hallita myös esimerkiksi PHP- tai Python-kielten kehyksillä toteutetuissa ohjelmistoissa. Järjestelmän tulee tukea avoimia rajapintoja ja protokollia.
