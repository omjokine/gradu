Servlet- tai FastCGI-tekniikka ei sido web-sovelluksen arkkitehtuuria, vaan molempia tekniikoita käytettäessä sovellusten perusarkkitehtuuri on sama. Sovellukset ovat jatkuvasti pyöriviä prosesseja, jotka saavat syötteenään käyttäjän syöttämän datan lisäksi joukon erilaisia suoritusympäristöön liittyviä resursseja ja muuttujia (mm. tietokantayhteys) \cite{uml}. Syötteen perusteella sovellus tuottaa tulosteen, joka palautetaan käyttäjälle.

Tyypillinen tapa toteuttaa web-sovelluksia nykypäivänä on käyttää sovelluskehystä, joka tuo oman lisänsä sovellusten arkkitehtuuriin. Käytettyjä kehyksiä on esimerkiksi Javalla toteutettu Spring, Pythonin Django ja Ruby-ohjelmointikieleen luottava Ruby on Rails. Näissä sovelluskehyksissä erillinen viestinvälittäjä (dispatcher) ottaa vastaan pyynnön ja välittää sen eteenpäin sovellusohjelmalle. Pyyntö ei siirry suoraan sovellusohjelmalle, vaan se välitetään väliohjelma-kerroksen läpi \cite{ruby2011agile}. Nämä väliohjelmat esimerkiksi tekevät merkintöjä lokitiedostoihin, asettavat ja lukevat käyttäjän selaimen evästeitä tai asettavat ympäristömuuttujia sovellusohjelmaa varten.

TODO: kuva vaikka railsin dispatcherista

Sovellusohjelmasta on pyritty karsimaan paljon yleisiä tehtäviä sovelluskehyksen tai käyttäjän toteuttaman väliohjelman tehtäväksi. Web-sovellusohjelman toteuttaja käyttää hyväkseen sovelluskehyksen tarjoamia ympäristömuuttujia, jotka ovat esimerkiksi viittauksia muuttujiin, joita säilytetään istunnon (session) ajan muistissa. Myös käyttäjän tunnistautuminen on yksi sovelluskehyksen tehtävistä, sovellusohjelma saa ympäristömuuttujana kirjautuneen käyttäjän tiedot (esimerkiksi tunnistenumeron ja käyttöoikeudet), joita se käyttää hyväksi ohjelmalogiikassaan.

Sovelluskehysten otettua vastuun web-sovellusten perustoiminallisuudesta on web-sovellusten arkkitehtuuri mennyt modulaarisempaan suuntaan [TODO: lähde?]. Saman sovelluskehyksen sisällä voi olla useampia ohjelmamoduuleita, joita viestinvälittäjä kutsuu esimerkiksi URL-osoitteen perusteella. Kun ohjelmamoduuleiden ei tarvitse toteuttaa kaikkea perustoiminallisuutta itse, yksinkertaistuu niiden rakenne ja web-sovellusten arkkitehtuureissa voidaan mennä kohti palveluperustaisia arkkitehtuureja.