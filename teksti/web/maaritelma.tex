Web-sovelluksella tarkoitetaan ohjelmaa, jota suoritetaan palvelimella ja jota käytetään WWW-selaimen avulla \cite{uml}. Käyttäjän selaimen kautta tekemät pyynnöt vaikuttavat palvelimen tilaan. Esimerkiksi lomakkeella voidaan lähettää palvelimelle tallennettavaa tai muokata siellä jo olevaa tietoa. Web-sovelluksilla on oma sisäinen logiikkansa, joka erottaa ne tavallisista web-sivuista \cite{uml}. Mikäli web-sivu toimii dynaamisesti, mutta ilman mahdollisuutta vaikuttaa sivuston sisäiseen logiikkaan, ei voida puhua web-sovelluksesta.

Web-sovelluksissa on kolme peruskomponenttia: selain, verkko ja palvelin. Käyttäjä pyytää selaimella verkon yli palvelimelta selaimen ymmärtämällä kielellä ohjelmoitua tiedostoa, jonka selain visualisoi käyttäjän ymmärtämään muotoon \cite{uml}. Tiedostojen koodauksessa käytetään yleisimmin HTML-kuvauskieleltä. Muita käytettyjä kieliä ovat mm. JavaScript ja Flash, joilla laajennetaan HTML-kielen toiminnallisuutta. Pyyntö tehdään verkon yli HTTP- tai HTTPS-protokollalla sen mukaan, käytetäänkö salaamatonta (HTTP) vai salattua (HTTPS) yhteyttä \cite{rfc2818}.

Palvelinarkkitehtuuri voi koostua useista fyysisistä tai virtuaalisista tietokoneista, jotka palvelevat eri osia käyttäjän pyynnöstä \cite{soa}. Puhuttaessa palvelimesta ei viitata yksittäiseen tietokoneeseen, joka palvelee käyttäjän pyyntöjä, vaan pal\-ve\-lin\-ark\-ki\-teh\-tuu\-riin. Tällaisesta hajautetusta palvelinarkkitehtuurista käytetään nimitystä palvelusuuntautunut arkkitehtuuri. Tätä käsitellään luvussa 2.4.