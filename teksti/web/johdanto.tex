Web-sovelluksella tarkoitetaan ohjelmaa, jota suoritetaan palvelimella ja jota käytetään WWW-selaimen avulla \cite{uml}. Käyttäjän selaimen kautta tekemät pyynnöt vaikuttavat palvelimen tilaan. Esimerkiksi lomakkeella voidaan lähettää palvelimelle tallennettavaa tietoa tai muokata palvelimelle aiemmin tallennettua tietoa. Web-sovelluksilla on oma sisäinen logiikka, joka erottaa ne tavallisista web-sivuista \cite{uml}. Web-sivu voi myös toimia dynaamisesti, mutta ilman mahdollisuutta vaikuttaa sivuston sisäiseen logiikkaan, jolloin ei voida puhua web-sovelluksesta.

Web-sovelluksissa on kolme peruskomponenttia: selain, verkko ja palvelin. Käyttäjä pyytää selaimella verkon yli palvelimelta selaimen ymmärtämällä kielellä ohjelmoitua tiedostoa, jonka selain visualisoi käyttäjän ymmärtämään muotoon \cite{uml}. Tiedostojen koodauksessa käytetään yleisimmin HTML-kuvauskielellä. Muita käytettyjä kieliä ovat mm. JavaScript ja Flash, joilla laajennetaan HTML-kielen toiminnallisuutta. Pyyntö tehdään HTTP- tai HTTPS-protokollilla sen mukaan, käytetäänkö salaamatonta (HTTP) vai salattua (HTTPS) yhteyttä \cite{rfc2818}.

Tässä luvussa käydään läpi web-sovellusten historiaa ja niiden perusarkkitehtuurien kehitystä kohti palvelusuuntautuneita järjestelmiä. Luvun loppupuolella avataan web-palveluiden tunnistautumiseen liittyvää ongelmakenttää.