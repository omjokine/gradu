Web-sovelluksella tarkoitetaan ohjelmaa, jota suoritetaan palvelimella ja jota käytetään WWW-selaimen avulla \cite{uml}. Käyttäjän selaimen kautta tekemät pyynnöt vaikuttavat palvelimen tilaan. Esimerkiksi lomakkeella voidaan lähettää palvelimelle tallennettavaa tietoa tai muokata olemassaolevaa. Web-sovellukset toteuttavat business logic (TODO: käännös), joka erottaa ne tavallisista web-sivuista \cite{uml}. Web-sivu voi toimia dynaamisesti, mutta ilman mahdollisuutta vaikuttaa business logiikan (WTF) tilaan, ei voida puhuta web-sovelluksesta.

Web-sovelluksissa on kolme perus komponenttia: selain, verkko ja palvelin. Käyttäjä pyytää selaimella verkon yli palvelimelta selaimen ymmärtämällä kielellä ohjelmoitua tiedostoa, jonka selain visualisoi käyttäjän ymmärtämään muotoon \cite{uml}. Käytetyin kieli on HTML, mutta yleisesti käytettyjä kieliä ovat myös JavaScript ja Flash. Pyyntö tehdään HTTP- tai HTTPS-protokollilla, riippuen käytetäänkö salaamatonta (HTTP) vai salattua (HTTPS) yhteyttä.

TODO: joko kuva tähän, joka kertoo mistä on kyse

TODO: luvun esittely, mitä seuraavissa aliluvuissa?
