Web-sovellusten merkitys on kasvanut Internetin yleistymisen myötä. Sovellukset ovat kehittyneet 90-luvun alun monoliittisista sovelluksista nykypäivän hajautetuiksi sovelluksiksi. Samalla niiden tekninen toteutus on muuttunut merkittävästi ohjelmistokielten ja ohjelmointikehysten kehittyessä.

Tässä luvussa käydään läpi web-sovellusten historiaa ja arkkitehtuuriratkaisuja sekä yksittäisen web-sovelluksen että palvelinarkkitehtuurien näkökulmasta. Ensimmäisessä aliluvussa määritellään web-sovellukset, jonka jälkeen käsitellään niiden teknistä kehitystä. Kahdessa viimeisessä aliluvussa tutustutaan web-sovellusten sisäiseen arkkitehtuuriin ja nykyaikaisiin palvelusuuntautuneisiin palvelinarkkitehtuureihin.