Palveluperustainen arkkitehtuuri (service oriented architecture, SOA) on arkkitehtuurityyli, jossa autonomiset sovellukset tarjoavat rajapinnan muille sovelluksille palveluidensa käyttämiseksi \cite{soa}. Sovellusten tehtävä on rajattu ja muista sovelluksista riippumaton. Esimerkiksi matkatoimiston web-sivulla voidaan käyttää lento- ja hotelliyhtiöiden tarjoamien web-sovellusten rajapintoja matkan varaamiseen ja pankin valuuttakurssi-rajapintaa hintojen muuttamiseen valuutasta toiseen. Palveluperustaiset sovellukset käyttävät toistensa palveluita Internetin yli web services -rajapintojen avulla \cite{soa}.

Palveluperustaisia sovelluksia käytettäessä ohjelmoija ei ole kiinnostunut yksittäisen sovelluksen teknisestä toteutuksesta, vaan sovellusten tarjoamista rajapinnoista eli palveluista \cite{soa}. Rajapintojen muoto on tavallisesti rakenteellinen JSON- tai XML-muotoinen dokumentti, joka on helposti koneen luettavissa. Eri tyyppiset web-sovellukset tai -palvelut käyttävät näitä rajapintoja eri tavalla. Esimerkiksi koko maailman säätiedon tarjoava palvelu saattaa kysyä säätietoja paikallisilta sääpalveluilta ja koostaa niistä JSON-dokumentin, jonka selaimessa pyörivä JavaScript-ohjelma rikastaa käyttäjän ymmärtämäksi sääkartaksi. Yhden sivulatauksen yhteydessä käyttäjä saattaa huomaamattaan käyttää useita eri web-palveluita.

Yksittäiset palvelut, kuten säätilapalvelu, voivat olla kaikille avoimia, kun taas jotkut palvelut haluavat tunnistaa käyttäjän, jotta hänelle voidaan näyttää häntä koskeva data ja pitää muiden data piilossa. Esimerkiksi palvelu, joka pitää kirjaa opiskelijoiden suorittamista kursseista, näyttää vain tunnistetun käyttäjän suoritukset. Käyttäjän tunnistautumiseen voidaan käyttää, muiden palveluiden tapaan, erillistä identiteetintarjoajaa, josta kerrotaan enemmän luvussa 4. Tämä edellyttää sekä käyttäjällä olevaa tunnusta identiteetintarjoajan palveluun (esimerkiksi Facebook, Google tai organisaation oma palvelu) että palvelun ylläpitäjän luottamusta identitetiintarjoajaan.