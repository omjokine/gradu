Palveluperustaisten arkkitehtuurien (service oriented architecture, SOA) on arkkitehtuurityyli, jossa autonomiset sovellukset tarjoavat rajapinnan muille sovelluksille palveluidensa käyttämiseksi [TODO: lähde? puustjärven kalvot]. Sovellukset on pilkottu pieniin osiin ja ne voivat tarjota vain yksittäisen palvelun, esimerkiksi valuuttamuunnoksen tai kertoa säätilan käyttäjän valitsemassa paikassa. Sovellukset käyttävät toistensa palveluita Internetin yli web services -rajapintojen avulla.

Web-sovellusten näkökulmasta palveluperustaiset arkkitehtuurit on johtanut siihen, että yksittäinen web-sovellus saattaa käyttää useaa palvelua saman sivunlatauksen yhteydessä. HTML5-pohjainen Javascriptiä hyväksikäyttävä käyttöliittymä voi hakea näytettävää dataa useasta eri palvelusta ja koostaa siitä käyttäjälle yhtenäiseltä näyttävän sivun. Esimerkiksi .. [TODO: esimerkki lähteineen. esim amazon.comin etusivu?].

Yksittäiset palvelut, kuten säätilapalvelu, voi olla kaikille avoin palvelu, joka ei vaadi käyttäjän tunnistautumista. Toisissa tilanteissa käyttäjä halutaan tunnistaa, jotta hänelle voidaan näyttää oikea data. Esimerkiksi palvelu, joka pitää kirjaa opiskelijoiden käymistä kursseista, näyttää vain tunnistetun käyttäjän suoritukset.