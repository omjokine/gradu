Alunperin CGI-ohjelmissa käytettiin tietovuopohjaista arkkitehtuurityyliä, jossa yksi web-sovellus palveli käyttäjää koko pyynnön ajan. Nykyään on siirrytty modulaarisiin arkkitehtuureihin, jossa web-palvelut voivat koostua useista web-sovelluksista \cite{soa}. Web-sovellukset toimivat autonomisesti ja hoitavat vain tietyn osan käyttäjän pyynnöstä. Tällaisesta useasta hajautetusta web-sovelluksesta koostuvasta web-palvelinarkkitehtuurista käytetään nimitystä palveluperustainen arkkitehtuuri.

Palveluperustainen arkkitehtuuri (service oriented architecture, SOA) on arkkitehtuurityyli, jossa autonomiset sovellukset tarjoavat rajapinnan muille sovelluksille palveluidensa käyttämiseksi \cite{soa}. Sovellusten tehtävä on rajattu ja muista sovelluksista riippumaton. Esimerkiksi matkatoimiston web-sivulla voidaan käyttää lento- ja hotelliyhtiöiden tarjoamien web-sovellusten rajapintoja matkan varaamiseen ja pankin valuuttakurssi-rajapintaa hintojen muuttamiseen valuutasta toiseen. Palveluperustaiset sovellukset voivat käyttää toistensa palveluita Internetin yli web services -rajapintojen avulla \cite{soa}.

Palveluperustaisia sovelluksia käytettäessä ohjelmoijan huomio ei kiinnity yksittäisen sovelluksen tekniseen toteutukseen, vaan sovellusten tarjoamiin rajapintoihin eli palveluihin \cite{soa}. Rajapintojen muoto on tavallisesti rakenteellista tekstiä, esimerkiksi JSON- tai XML-muotoisia dokumentteja, jotka ovat helposti koneen luettavissa. Erityyppiset web-sovellukset tai -palvelut käyttävät näitä rajapintoja eri tavalla. Esimerkiksi koko maailman säätiedon tarjoava palvelu saattaa kysyä säätietoja paikallisilta sääpalveluilta ja koostaa niistä JSON-dokumentin, jonka selaimessa pyörivä JavaScript-ohjelma rikastaa käyttäjän ymmärtämäksi sääkartaksi. Tämä johtaa siihen, että yhden sivulatauksen yhteydessä käyttäjä saattaa huomaamattaan käyttää useita eri web-sovelluksia.

Yksittäiset palvelut, kuten säätilapalvelu, voivat olla kaikille avoimia, kun taas jotkut palvelut vaativat käyttäjän tunnistamisen voidakseen näyttää häntä koskevaa dataa ja pitää muiden data piilossa. Esimerkiksi palvelu, joka pitää kirjaa opiskelijoiden suorittamista kursseista, näyttää vain tunnistetun käyttäjän suoritukset. Käyttäjän tunnistamiseen voidaan käyttää erillistä identiteetintarjoajaa, jollaisista kerrotaan enemmän luvussa 4. Tämä edellyttää sekä käyttäjällä olevaa tunnusta identiteetintarjoajan palveluun (esimerkiksi Facebook, Google tai organisaation oma palvelu) että palvelun ylläpitäjän luottamusta identitetiintarjoajaan.