Palveluperustaisten arkkitehtuurien (service oriented architecture, SOA) on arkkitehtuurityyli, jossa autonomiset sovellukset tarjoavat rajapinnan muille sovelluksille palveluidensa käyttämiseksi [TODO: lähde? puustjärven kalvot]. Sovellukset on pilkottu pieniin osiin ja ne voivat tarjota vain yksittäisen palvelun, esimerkiksi valuuttamuunnoksen tai kertoa säätilan käyttäjän valitsemassa paikassa. Sovellukset käyttävät toistensa palveluita Internetin yli web services -rajapintojen avulla.

Palveluperustaisessa arkkitehtuurityylissä yksittäisen sovelluksen tekninen toteutus abstraktoidaan pois ja keskitytään vain sovellusten tarjoamiin rajapintoihin eli palveluihin. Rajapintojen muoto on tavallisesti rakenteellinen JSON- tai XML-muotoinen dokumentti, joka on helposti koneen luettavissa. Eri tyyppiset web-sovellukset tai -palvelut käyttävät näitä rajapintoja eri tavalla. Koko maailman sään tarjoava palvelu saattaa kysyä säätietoja paikallisilta sääpalveluita, kun taas HTML5-tekniikalla tehdyt web-sovellukset voivat lukea raakaa JSON-muotoista dataa ja rikastaa sen JavaScriptillä tai Flashilla käyttäjän ymmärtämään muotoon, vaikkapa diagrammiksi. Yhden sivulatauksen yhteydessä käyttäjä saattaa huomaamattaan käyttää useita eri web-palveluita.

Yksittäiset palvelut, kuten säätilapalvelu, voi olla kaikille avoin, mutta jotkut palvelut taas haluavat tunnistaa käyttäjän, jotta hänelle voidaan näyttää häntä koskeva data ja muiden data pysyy piilossa. Esimerkiksi palvelu, joka pitää kirjaa opiskelijoiden suorittamista kursseista, näyttää vain tunnistetun käyttäjän suoritukset. Käyttäjän tunnistautumiseen voidaan käyttää, muiden palveluiden tapaan, erillistä tunnistautumispalvelua, josta enemmän luvussa 4.