Tunnistautumispalvelu (auth.kapsi.fi) tarjoaa kaksi erillistä rajapintaa, joista toinen on käyttäjälle näkyvä kirjautumissivu. Se palauttaa valtuutusavaimen, jos käyttäjän antama tunnus/salasana-yhdistelmä vastaa LDAP-käyttäjähallinnassa olevaa tunnusta. Toisen rajapinnan kautta web-palvelut voivat hakea käyttäjän tiedot käyttäjän antaman valtuutusavaimen avulla.

Palvelu on toteutetaan OAuth-rajapinnalla. LDAP-käyttäjähallinnan ja tunnistautumispalvelun väliseen kommunikointiin käytetään nykyisessä Sikteerissä käytössä olevaa LDAPBackend-taustajärjestelmää. LDAP:n ja tunnistautumispalvelun väliseen kommunikointiin ei tämän tutkielman puitteissa oteta kantaa, vaan se toimii kuten tähänkin asti.

Jotta tunnistautumispalvelu toimisi halutulla tavalla, täytyy sitä käyttäville web-palveluille (esim. Sikteeri) määritellä palvelun tunnus ja yhteinen jaettu salainen avain \cite{oauth2_0}. Näiden avulla voidaan varmistua siitä, että tunnistautumispyyntö tulee varmasti hyväksyttävästä lähteestä ja käyttäjä ei pysty väärentämään valtuutusavainta \cite{oauth2_0}.

Palvelun yksilöivän tunnuksen avulla voidaan toteuttaa pääsynvalvonta \cite{oauth2_0}. Esimerkiksi Sikteeriä pääsee käyttämään vain ylläpitoon kuuluvat Kapsin jäsenet, joten tunnistautumispalvelun ei tule antaa käyttäjän tietoja web-palvelulle, jos käyttäjä ei kuulu ylläpito-ryhmään. Toinen mahdollisuus on tehdä pääsynvalvonta web-palvelimen päässä, jolloin tunnistautumispalvelu palauttaa käyttäjän tietojen mukana listan käyttäjän oikeuksista, joiden avulla web-palvelu voi päättää onko käyttäjällä oikeus päästä järjestelmään.