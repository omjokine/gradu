Toteutuksen kuvaus. Arkkitehtuuri yms, koodi liitteenä jos tarvii?

Käytännössä kyseessä on palveluna toimiva tunnistautumisjärjestemä, jolla on joku tietokanta ja rajapinta, jonka kautta tunnistautumispyyntöjä tehdään. Todennäköisesti käyttäjätietokanta on relaatiokanta ja rajapinta Oauth.

Tässä luvussa kuvataan ko. toteutus tarpeellisella tarkkuudella.

Tutkielman tarkoitus on toteuttaa prototyyppi sovelluksesta, jonka avulla erillisten web-sovellusten käyttäjähallinta keskitetään yhteen komponenttiin. Sovellus hoitaa käyttäjän tunnistautumisen ja käyttäjätietojen hallinnan, joka synkronoidaan asiakasyrityksen käyttäjätietokannan kanssa. Tutkielmassa selvitetään myös tarvitaanko erillistä käyttäjätietokantaa ollenkaan vai voidaanko asiakasyrityksissä tyypillisesti käytössä olevia LDAP-hakemistoja käyttää toteutettavan komponentin käyttäjätietokantana.