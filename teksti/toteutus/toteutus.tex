Toteutuksen kuvaus. Arkkitehtuuri yms, koodi liitteenä jos tarvii?

Aluksi vielä vedetään yhteen, että mitä ovat keskeiset ongelmat tällä hetkellä ja miten tämä toteutettu prototyyppi tuo niihin parannusta.

Odotettavissa olevista ongelmista myös jotain. Onko OAuth edes paras mahdollinen juttu tähän, eikö käyttötarkoitus ole vähän toisenlainen?

Käytännössä kyseessä on palveluna toimiva tunnistautumisjärjestemä, jolla on tietokanta ja rajapinta, jonka kautta tunnistautumispyyntöjä tehdään. Todennäköisesti käyttäjätietokanta on relaatiokanta ja rajapinta Oauth. Otetaan joku satunnainen web-sovellus, jossa on tietokantatunnistus ja irrotetaan siitä palanen omaksi palvelukseen ja tehdään toimiva arkkitehtuuri. Tässä luvussa kuvataan ko. toteutus tarpeellisella tarkkuudella. Järjestelmän arkkitehtuuria, viestinkulkua yms kuvataan UML-tekniikalla.

Tutkielman tarkoitus on toteuttaa prototyyppi sovelluksesta, jonka avulla erillisten web-sovellusten käyttäjähallinta keskitetään yhteen komponenttiin (huom, tämä yksi komponentti voi tehdä muutakin kuin tunnistaa, mutta ehkä voisi viedä niinkin pitkälle, että se ei tee muuta kuin tunnistusta). Sovellus hoitaa käyttäjän tunnistautumisen ja käyttäjätietojen hallinnan, joka synkronoidaan asiakasyrityksen käyttäjätietokannan kanssa.

Pituus n. 10sivua.
