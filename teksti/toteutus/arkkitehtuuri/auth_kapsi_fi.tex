Uuteen arkkitehtuuriin liittyvä tunnistautumispalvelu (auth.kapsi.fi) tarjoaa kaksi erillistä rajapintaa, joista ensimäinen on käyttäjälle näkyvä kirjautumissivu. Toisen rajapinnan kautta web-palvelut voivat hakea käyttäjän tiedot valtuutusavaimen avulla. Tunnistautumispalvelun täytyy tarjota molemmat rajapinnat, koska siinä on päätetty käyttää OAuth-protokollaa.

Tunnistautumispalvelu on yhteydessä Kapsin käyttäjähallintaan, joka on LDAP-tietokannassa. Käyttäjähallinnan ja tunnistautumispalvelun väliseen kommunikointiin käytetään LDAPBack\-end-taustajärjestelmää, jonka toimintaperiaate on samankaltainen kuin Sikteerissä käytetyssä ModelBackend-taustajärjestelmästä. Jatkossa järjestelmän salasana on tallennettu vain LDAP-tietokantaan ja sitä ei kopioida yksittäisiin palveluihin, kuten tähän asti.

Jotta tunnistautumispalvelu toimisi halutulla tavalla, täytyy sitä käyttäville web-palveluille (esim. Sikteeri) määritellä palvelun tunnus ja yhteinen jaettu salainen avain \cite{oauth2_0}. Näiden avulla voidaan varmistua siitä, että tunnistautumispyyntö tulee varmasti hyväksyttävästä lähteestä ja käyttäjä ei pysty väärentämään valtuutusavainta. Tällä suojataan käyttäjän tietoja ja estetään niiden joutumisen vääriin käsiin. Myös auditoinnin tekeminen helpottuu, kun tunnistautumispalvelun lokitiedostoista voidaan katsoa mihin palveluihin ja koska käyttäjä on kirjautunut.

Palvelun yksilöivän tunnuksen avulla voidaan toteuttaa pääsynvalvonta \cite{oauth2_0}. Esimerkiksi Sikteeriä pääsevät käyttämään vain ylläpitoon kuuluvat Kapsin jäsenet, joten tunnistautumispalvelun ei tule antaa käyttäjän tietoja web-palvelulle, jos käyttäjä ei kuulu ylläpito-ryhmään. Toinen mahdollisuus on tehdä pääsynvalvonta web-palvelimen päässä, jolloin tunnistautumispalvelu palauttaa käyttäjän tietojen mukana listan käyttäjän oikeuksista, joiden avulla web-palvelu voi päättää, onko käyttäjällä oikeus päästä järjestelmään. Kirjautumispalvelua, ja koko arkkitehtuuria, on arvioitu seuraavassa luvussa.