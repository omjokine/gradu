Tekninen toteutus jakautuu kahteen osaan, nykyiseen Sikteeriin sekä uuteen tunnistautumispalveluun. Tunnistautumispalvelua varten Sikteeristä irrotetaan LDAP-tunnistautumiseen liittyvät osat ja niistä tehdään oma palvelunsa, joka tarjoaa ainakin OAuth-rajapinnan. Sikteeriä taas muutetaan niin, että se käyttää tuota tunnistautumispalvelua rajapinnan kautta.

Seuraavassa aliluvussa käydään läpi Sikteerin nykyistä arkkitehtuuria ja siihen vaadittavia muutoksia, jotta sitä voidaan käyttää yhdessä ulkoisen tunnistautumispalvelun kanssa. Tämän jälkeen määritellään uuden tunnistautumispalvelun toiminta ja hahmotellaan sen arkkitehtuuria.

Ensimmäisessä vaiheessa Sikteerin omaan kantaan on tallennettu lista käyttäjätunnuksista, joilla on pääsy järjestelmään.

Toteutukseen valitulle OAuth-protokollalle on olemassa valmis avoimen lähdekoodin Django-väliohjelmatoteutus.

\subsubsection{Sikteeri}
Sikteerin tärkein uudistus on käyttäjän tunnistamiseen käytettävien tietojen (salasana) siirtäminen pois sen omasta SQL-tietokannasta. Sikteerin omaan tietokantaan tallennetaan jatkossakin niiden käyttäjien tiedot, joilla on pääsy järjestelmään, koska käyttäjällä on vain Sikteeriin liittyviä attribuutteja, joita ei haluta tallentaa käyttäjähallintaan. Samalla Sikteerin omaa tietokantaa käytetään pääsynvalvonnassa: tunnistamisen jälkeen Sikteeriin pääsee vain käyttäjät, joiden käyttäjätunnusta vastaava rivi on kirjattu tietokantaan.

Käyttäjätunnus yksilöi käyttäjän Kapsin järjestelmissä, joten sitä voidaan käyttää avaimena Sikteerin ja tunnistautumispalvelun välillä. Tunnistamisen jälkeen tunnistautumispalvelusta tuleva käyttäjän data sisältää tiedon myös käyttäjätunnuksesta, jota käytetään käyttäjän yksilöivänä tunnisteena.

Käytännön tasolla uutta keskitettyä tunnistautumispalvelua käytettäessä väliohjelmakerroksessa tunnistautumista tekevän väliohjelman toimintaa muutetaan. Nykyisessä toteutuksessa väliohjelma vertaa käyttäjän syöttämää tunnusta ja salasanaa paikallisesta tietokannasta löytyviin. Uudistuksen jälkeen käyttäjätunnus selvitetään ulkoisen palvelun avulla, jonka jälkeen tunnukseen liittyvät attribuutit haetaan paikallisesta kannasta. Väliohjelman muutoksen lisäksi muita muutoksia Sikteerin ohjelmakoodiin ei tarvitse tehdä.

Uudistetun Sikteerin toiminta on esitetty kuvassa \ref{auth_kapsi_fi_flow}. Sikteeri siis ohjaa käyttäjän tunnistautumispalveluun, josta lähetetään onnistuneen tunnistautumisen jälkeen valtuutusavain Sikteerille. Valtuutusavaimella Sikteerin väliohjelmisto hakee käyttäjän tiedot tunnistautumispalvelusta ja etsii tietokannastaan käyttäjätunnusta vastaavan käyttäjän tiedot.
\subsubsection{Tunnistautumispalvelu (auth.kapsi.fi)}
Uuteen arkkitehtuuriin liittyvä tunnistautumispalvelu (auth.kapsi.fi) tarjoaa kaksi erillistä rajapintaa, joista ensimäinen on käyttäjälle näkyvä kirjautumissivu. Toisen rajapinnan kautta web-palvelut voivat hakea käyttäjän tiedot valtuutusavaimen avulla. Tunnistautumispalvelun täytyy tarjota molemmat rajapinnat, koska siinä on päätetty käyttää OAuth-protokollaa.

Tunnistautumispalvelu on yhteydessä Kapsin käyttäjähallintaan, joka on LDAP-tietokannassa. Käyttäjähallinnan ja tunnistautumispalvelun väliseen kommunikointiin käytetään LDAPBack\-end-taustajärjestelmää, jonka toimintaperiaate on samankaltainen kuin Sikteerissä käytetyssä ModelBackend-taustajärjestelmästä. Jatkossa järjestelmän salasana on tallennettu vain LDAP-tietokantaan ja sitä ei kopioida yksittäisiin palveluihin, kuten tähän asti.

Jotta tunnistautumispalvelu toimisi halutulla tavalla, täytyy sitä käyttäville web-palveluille (esim. Sikteeri) määritellä palvelun tunnus ja yhteinen jaettu salainen avain \cite{oauth2_0}. Näiden avulla voidaan varmistua siitä, että tunnistautumispyyntö tulee varmasti hyväksyttävästä lähteestä ja käyttäjä ei pysty väärentämään valtuutusavainta. Tällä suojataan käyttäjän tietoja ja estetään niiden joutumisen vääriin käsiin. Myös auditointi on mahdollista, kun tunnistautumispalvelun lokitiedostoista voidaan katsoa mihin palveluihin ja koska käyttäjä on kirjautunut.

Nykyään pääsynvalvonta toteutetaan Sikteerissä vertaamalla tunnistautumispalvelulta saatavaa käyttäjätunnusta Sikteerin tietokannassa määriteltyihin tunnuksiin, joilla on oikeus käyttää palvelua. Tulevaisuudessa tunnistautumispalvelun voisi palauttaa käyttäjän tietojen mukana listan käyttäjän oikeuksista, joiden avulla web-palvelu voi päättää, onko käyttäjällä oikeus päästä järjestelmään. Pääsynvalvonta voidaan toteuttaa myös tunnistautumispalvelussa, jolloin se paluttaa Sikteerille käyttäjän tiedot vain, jos käyttäjällä on oikeus päästä järjestelmään. Nykytoteutuksessa tiedot palautetaan joka tapauksessa ja Sikteeri tekee päätöksen, onko käyttö sallittu. Arkkitehtuurin pohditaan seuraavassa luvussa, jossa myös arvioidaan täyttääkö uusi arkkitehtuuri sille asetetut vaatimukset.