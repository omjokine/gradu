Nykyisellään Sikteeri käyttää Django-ohjelmistokehyksen tarjoamaa väliohjelmistoa tunnistautumiseen. Django.contrib.auth-niminen väliohjelmisto tarjoaa hyvät laajentamismahdollisuudet \cite{django_auth}. Siinä on mahdollista määritellä eri taustajärjestelmiä (backend), joita käytetään tunnistautumisessa. Sikteeri käyttää tunnistautumiseen Djangon tarjoamaa ModelBackend-taustajärjestelmää, joka mahdollistaa käyttäjän tunnistautumisen vertaamalla käyttäjän syöttämää käyttäjätunnusta ja salasanaa tietokantaan tallennettuihin käyttäjiin \cite{django_auth}. Jos oikealla tunnus/salasana-parilla oleva käyttäjä löytyy, palautetaan se sovellustasolle.

Ulkoista tunnistautumispalvelua käytettäessä ModelBackend korvataan uudella taustajärjestelmällä. Taustajärjestelmän toiminta on esitetty kuvassa \ref{auth_kapsi_fi_flow}. Sikteeri siis ohjaa käyttäjän tunnistautumispalveluun, josta lähetetään onnistuneen tunnistautumisen jälkeen valtuutusavain Sikteerille. Valtuutusavaimella Sikteerin väliohjelmisto hakee käyttäjän tiedot jälleen tunnistautumispalvelusta ja etsii tietokannastaan käyttäjätunnusta vastaavan käyttäjän tiedot.

Uusitussa Sikteerissä käyttäjät tallennetaan edelleen entiseen tapaan sen omaan tietokantaan. Käyttäjätunnus yksilöi käyttäjän Kapsin järjestelmissä, joten se täytyy olla tietokannassa käyttäjän tunnistamista varten. Sen sijaan Sikteerin tietokannasta voi poistaa salasana-sarakkeen, sillä paikallista salasanaa ei käytetä enää jatkossa. Käyttäjien tiedot tallennetaan Sikteeriin, koska käyttäjällä on vain Sikteeriin liittyviä attribuutteja, joita ei haluta tallettaa käyttäjähallintaan.

Sikteerin käyttäjän näkökulmasta järjestelmän toiminta ei poikkea merkittävästi vanhasta. Sikteerin tarjoaman kirjautumislomakkeen sijaan käyttäjä ohjataan tekemään tunnistautuminen ulkoiselle palvelimelle.