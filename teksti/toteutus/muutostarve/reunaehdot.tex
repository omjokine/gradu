Nykyinen toimintaympäristö, sekä Kapsin omat käytännöt, asettavat vaatimuksia ja reunaehtoja toteutukselle. Enimmäkseen nämä vaatimukset johtuvat käytetystä Python-kielestä ja halusta tukea avointa ohjelmistokehitystä.

Järjestelmän ylläpidon helpottamiseksi uudet komponentit halutaan toteuttaa Pythonilla, koska Sikteeri ja admtool on toteutettu sillä. Toteutuksessa halutaan käyttää mahdollisimman paljon avoimen lähdekoodin alaista koodia, joten tunnistautumispalvelussa käytetystä protokollasta täytyy olla Python-kielellä toteutettu ratkaisu. Toisaalta järjestelmän sisällä ei haluta sulkea muiden ohjelmointikielten tai -kehysten käyttöä pois, joten pelkästään Python-kielinen ratkaisu ei riitä, vaan standardejen täytyy olla avoimia ja käytettävissä myös Javan web-kehyksillä tai esimerkiksi Ruby on Rails -ohjelmointikehyksellä.

Tunnistautumispalvelun täytyy tukea myös komentorivityökaluja, kuten nykyistä admtoolia. Tällöin pelkkä käyttäjäagentti-pohjainen tunnistamismenetelmä ei riitä, vaan tunnistamisen täytyy olla mahdollista ilman selainta.

Käyttäjän tunnistaminen ei riitä yksinään, vaan tunnistautumispalvelun täytyy tukea myös käyttäjään liittyvien parametrien välitystä. Tämä on tärkeää, koska järjestelmässä on eri tasoisia käyttäjiä (hallitus, toimihenkilöt ja ylläpito) ja tunnistautumispalvelua käyttävän sovelluksen täytyy tietää, mihin käyttäjäryhmään tunnistettu käyttäjä kuuluu. Myös muita käyttäjään liittyviä attribuutteja, kuten sähköpostiosoitetta tai puhelinnumeroa, tarvitaan sovelluksissa.

Valitun tunnistautumisprotokollan täytyy olla sellainen, että nykyisten sovellusten muuttaminen käyttämään uutta tunnistautumismekanismia on helppoa. Tämä tarkoittaa, että koodiin täytyy tehdä mahdollisimman vähän muutoksia. Nykyisissä toteutuksissa on käytetty Djangon vakio-tunnistautumismekanismia, joka toimii väliohjelmakerroksessa. Käytettävästä tunnistautumisprotokollasta olisi hyvä olla valmis avoimen lähdekoodin Django-valiohjelmatoteutus jo olemassa.