Kapsi ry on vuonna 2003 perustettu vapaaehtoistyöhön perustuva järjestö, jonka tarkoitus on tuottaa jäsenilleen Internet-palveluita \cite{kapsifi}. Kaveriporukalla käynnistyneestä yhdistyksestä on vuosien varrella kasvanut yli 4000 jäsenen yhdistys. Jäsenten tiedot on kirjattu jäsenrekisteriin ja jokaisella Kapsin jäsenellä on käyttäjätunnus, jolla hän voi kirjautua Kapsin eri tietojärjestelmiin.. Yhdistyksellä on erilaisia tarpeita jäsenrekisterin hallintaan, esimerkiksi uusien jäsenhakemusten käsittelyyn ja vuosittain perittävän jäsenmaksun laskutukseen.

Jäsenrekisterin hallintatyökalut on keskitetty Kapsi ry:n toteuttamaan web-so\-vel\-luk\-seen nimeltä Sikteeri sekä komentorivityökaluun admtool. Sikteeri tarjoaa erilaisia palveluita eri käyttäjäryhmille. Hallituksen jäsenet hoitavat erilaisia jäseniin liittyviä toimenpiteitä, kuten uusien jäsenten hyväksymisiä ja vanhojen erottamisia. Ylläpidon täytyy päästä jäsentietoihin käsiksi palvelupyyntöihin vastatessa. Laskutusryhmän jäsenet käsittelevät jäsenmaksuihin liittyviä asioita, eli perivät jäsenmaksut ja tarkkailevat maksusuorituksia. Admtool on ylläpidon työkalu, jolla mm. poistetaan eronneiden jäsenten käyttäjätunnukset.

Tässä luvussa käydään läpi jäsenhallintaan liittyviä työkaluja ja niiden arkkitehtuuriin suunniteltuja muutoksia. Luvun ensimmäisessä alaluvussa tarkastellaan järjestelmän nykytilaa. Toisessa alaluvussa esitellään järjestelmän muutostarve ja näihin muutoksiin liittyviä vaatimuksia sekä järjestelmän asettamia reunaehtoja. Viimeisessä alaluvussa kuvataan järjestelmän uusi arkkitehtuuri.