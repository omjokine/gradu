Kapsi ry on vuonna 2003 perustettu vapaaehtoistyöhön perustuva järjestö, jonka tarkoitus on tuottaa jäsenilleen Internet-palveluita \cite{kapsifi}. Kaveriporukalla käynnistyneestä yhdistyksestä on vuosien varrella kasvanut yli 4000 jäsenen yhdistys. Yhdistyksellä on erilaisia tarpeita jäsenrekisterin hallintaan, esimerkiksi uusien jäsenhakemusten käsittelyyn ja vuosittain perittävän jäsenmaksun laskutusta varten.

Jäsenrekisterin hallintatyökalut on keskitetty Kapsi ry:n toteuttamaan järjestelmään nimeltä Sikteeri. Sikteeri tarjoaa erilaisia palveluita eri käyttäjäryhmälle. Hallituksen jäsenet hoitavat erilaisia jäseniin liittyviä toimenpiteitä, kuten uusien jäsenten hyväksyminen ja vanhojen erottaminen. Ylläpidon täytyy päästä jäsentietoihin käsiksi palvelupyyntöihin vastatessa. Laskutusryhmän jäsenet käsittelee jäsenmaksuihin liittyviä asioita, eli perivät jäsenmaksut ja tarkkailevat suorituksia.

Sikteeri on vuosien varrella paisunut yleiseksi toiminnanohjausjärjestelmäksi, jonka arkkitehtuuria kehittäjät haluaisivat viedä palvelusuuntautuneiden arkkitehtuurien suuntaan, jotta yksittäisten palveluiden kehittäminen olisi kevyempää [TOOD: lähde]. Jotta Sikteerin ulkopuolisten web-palveluiden kehittäminen on mahdollista, täytyy erillisillä palveluilla olla tapa tunnistaa käyttäjä. Varsinainen käyttäjähallinta on Sikteerin ulkopuolisessa LDAP-tietokannassa, jota vasten Sikteeri tunnistaa käyttäjät. Nykyistä LDAP-tietokantaa halutaan käyttää myös keskitetyssä tunnistautumispalvelussa, mutta Sikteerin (tai muun vastaavan palvelun) ei pidä päästä suoraan LDAP-kantaan käsiksi.