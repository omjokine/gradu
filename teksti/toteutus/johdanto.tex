Kapsi ry on vuonna 2003 perustettu vapaaehtoistyöhön perustuva järjestö, jonka tarkoitus on tuottaa jäsenilleen Internet-palveluita \cite{kapsifi}. Kaveriporukalla käynnistyneestä yhdistyksestä on vuosien varrella kasvanut yli 4000 jäsenen yhdistys, jotka on kirjattu jäsenrekisteriin. Yhdistyksellä on erilaisia tarpeita jäsenrekisterin hallintaan, esimerkiksi uusien jäsenhakemusten käsittelyyn ja vuosittain perittävän jäsenmaksun laskutusta varten.

Jäsenrekisterin hallintatyökalut on keskitetty Kapsi ry:n toteuttamaan web-so\-vel\-luk\-seen nimeltä Sikteeri. Sikteeri tarjoaa erilaisia palveluita eri käyttäjäryhmälle. Hallituksen jäsenet hoitavat erilaisia jäseniin liittyviä toimenpiteitä, kuten uusien jäsenten hyväksyminen ja vanhojen erottaminen. Ylläpidon täytyy päästä jäsentietoihin käsiksi palvelupyyntöihin vastatessa. Laskutusryhmän jäsenet käsittelee jäsenmaksuihin liittyviä asioita, eli perivät jäsenmaksut ja tarkkailevat maksusuorituksia.

Luvun ensimmäisessä alaluvussa käydään läpi järjestelmän nykytila. Toisessa alaluvussa esitellään järjestelmän muutostarve ja muutoksiin liittyviä vaatimuksia sekä reunaehtoja. Viimeisessä alaluvussa kuvataan järjestelmän uusi arkkitehtuuri.